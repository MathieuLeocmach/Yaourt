\begin{frame}{Creep and yielding}
Just accepted in PRL, \textit{ArXiv 1401.8234}

\bigskip
\structure{Merciless yoghurt breakers}

\bigskip
\begin{tabu}{X[c]X[c]c}
\includegraphics[height=0.3\textheight]{Chris}&
\includegraphics[height=0.3\textheight]{Thibaut}&
\includegraphics[height=0.3\textheight]{Seb}\\
Christophe Perge & Thibaut Divoux & Sebastien Manneville\\
PhD student & CNRS researcher & Professor\\
E.N.S. Lyon &  CRPP Bordeaux & E.N.S. Lyon\\
\end{tabu}


\bigskip
\begin{description}[Special thanks]
\item[Special thanks] to Alan Parker (Firmenich) for prompting this study and providing casein \& GDL
\end{description} 

\end{frame}

%\tikzset{external/force remake}
\begin{frame}{The yoghurt creep experiment}
\begin{tikzpicture}
\node[anchor=south west] (webcam){\includegraphics[width=0.2\textwidth]{webcam}};
\node[anchor=south] at (0.4\textwidth,0) (spoon) {\includegraphics[width=0.3\textwidth]{spoon}};
\node[anchor=south east] at (\textwidth,0) (ultrasound) {\includegraphics[width=0.4\textwidth]{ultrasound}};

\node[anchor=south west] at ($(ultrasound.north west)+(0,2em)$) (rheometer) {\includegraphics[height=0.4\textheight]{rheometer}};

\begin{scope}[draw]
	\node[below=0of spoon.south] (couette) {A Taylor-Couette cell where the yogurt is made \emph{in situ}};
	\draw[->] (couette) -- ($(spoon.south)+(0,2em)$);
	\node[above=0of ultrasound] {ultrasonic imaging};
	\node[above=0of webcam, text width=0.2\textwidth, align=center] {optical imaging};
	\draw[->] (spoon.north east) +(235:2em) -- (rheometer);
	\node[right=0of rheometer, text width=0.2\textwidth, align=center] {A rheometer imposes a constant stress $\sigma$ and records the strain $\gamma(t)$};
\end{scope}
\end{tikzpicture}
\end{frame}

%\tikzset{external/force remake}
\begin{frame}{The yoghurt creep experiment}
\movie[externalviewer]{\begin{tikzpicture}
\node[inner sep=0] (a) {\includegraphics[height=16\baselineskip]{creep_experiment.jpg}};
\node[below right=\baselineskip of a.north east, inner sep=0] (vmap) {\includegraphics[width=4\baselineskip]{jet_cmap}};

\begin{scope}[font=\footnotesize]
	\node[above=0 of vmap] {velocity map};
	\node[below=0 of vmap.south west] {-50};
	\node[below=0 of vmap.south] (v0) {0};
	\node[below=0 of vmap.south east] {50};
	\node[below=0 of v0, inner sep=0]{\si{\micro\metre\per\second}};
\end{scope}

\begin{scope}[shift={(a.north west)}, shift={(5.8\baselineskip,-1.1\baselineskip)}, ultra thick]
	\draw[Accent1, x radius=3.1\baselineskip, y radius=0.3\baselineskip] ellipse[] ++(3.1\baselineskip,0)  -- ++(0,-7.4\baselineskip)  ++(-3.1\baselineskip,0) ellipse[] ++(-3.1\baselineskip,0)  -- ++(0,7.4\baselineskip);
	\shadedraw[Accent2,ball color=Accent2, x radius=2\baselineskip, y radius=0.1\baselineskip] ellipse[] ++(2\baselineskip,0)  -- ++(0,-7.4\baselineskip)  ++(-2\baselineskip,0) ellipse[] ++(-2\baselineskip,0)  -- ++(0,7.4\baselineskip);
	\fill[Accent2, left color=black,right color=black, middle color=Accent2, ] (-0.3\baselineskip,0) rectangle +(0.6\baselineskip, \baselineskip);
	\begin{scope}[Main]
		\node[draw, minimum height=4.2\baselineskip, minimum width=1.5\baselineskip] at (2.5\baselineskip,-4\baselineskip) (window){};
		\node[draw, minimum height=7.2\baselineskip, minimum width=4\baselineskip] at (8.9\baselineskip,-4.2\baselineskip) (USV){};
		\draw(window.north east) -- (USV.north west) (window.south east) -- (USV.south west);
		\node[right, text width=4\baselineskip] at (USV.east) {ultrasound region of interest};
	\end{scope}

\draw[<->,thin] ++(-2\baselineskip,-\baselineskip) -- +(-1.1\baselineskip,0) node[midway,above] {\SI{2}{\milli\metre}};
\end{scope}
\draw[->] (window.north west) ++(-1em, 0.2em) -- +(1em,0) node[midway,above,inner sep=1pt] {$r$};
\draw[->] (window.north west) ++(-1em, 0.2em) -- +(0,-1em)  node[midway,left] {$z$};

\node[text width=4\baselineskip] at (-3\baselineskip,-2.5\baselineskip) {shear rate response};
\node at (5\baselineskip,-2.5\baselineskip) {strain response};
\end{tikzpicture}%
}{Yaourt141_enhanced.avi}
\end{frame}
%\tikzset{external/force remake=false}

\begin{frame}{Creep rheology: Three regimes}
\begin{tikzpicture}
\begin{loglogaxis}[
   name=g,
	width=\textwidth,
	height=0.8\textheight,
	xmin=2e-2, xmax=4e5, xlabel={time (\si{\second})},
	ymin=0.2,ymax=2,
	ytick={0.2, 0.4,0.8,1.6}, yticklabels={0.2, 0.4,0.8,1.6},
	ylabel={strain $\gamma$},
	extra tick style={grid=major},%
	extra y ticks={1}, extra y tick labels={1.0},
	cycle list name=earthy,
	axis on top,
	]
	\fill[Accent1!20]
		(axis cs:10,0.4) ellipse[rotate=-30, x radius=0.2\textwidth, y radius=0.12\textwidth]
		(axis cs:2e2,1.6) ellipse[x radius=0.35\textwidth, y radius=0.05\textwidth]
		(axis cs:3e3,0.85) ellipse[rotate=-7.5, x radius=0.3\textwidth, y radius=0.07\textwidth]
		;
	\node[above right] at (axis cs:1e1,0.2) {primary};
	\node[below left] at (axis cs:4e5,0.7) {secondary};
	\node[left] at (axis cs:5,1.6) {tertiary};
	\addplot table{Y27_200Pa_gamma_decimated.txt} node (s200){};
	\addplot table{Y38_300Pa_gamma_decimated.txt} node (s300){};
	\addplot table{Y25_400Pa_gamma_decimated.txt}  node (s400){};
	\addplot table{Y32_550Pa_gamma_decimated.txt}  node (s550){};
	\addplot table{Y39_1000Pa_gamma_decimated.txt} node (s1000){};
\end{loglogaxis}
\begin{scope}[anchor=base, every node/.style={yshift=0.2em}]
	\node[red!40!black] at (s200 |- g.outer north) {\SI{200}{\pascal}};
	\node[red!60!black] at (s300 |- g.outer north) {300};
	\node[red!80!black] at (s400 |- g.outer north) {400};
	\node[red] at (s550 |- g.outer north) {550};
	\node[red!80!yellow] at (s1000 |- g.outer north) {1000};
\end{scope}
\end{tikzpicture}

Failure at $\gamma\approx 1$ for a well defined time $\tau_f$
\end{frame}

%\tikzset{external/force remake}
\begin{frame}{Failure time}
\begin{columns}
\column{0.55\textwidth}
\begin{tikzpicture}
\begin{loglogaxis}[
	width=\textwidth,
	height=0.8\textwidth,
	xlabel={$\sigma$ (\si{\pascal})},
	ylabel={$\tau_f$ (\si{\second})},
	xmin=100, ymin=10, ymax=1e6,
	xtick={100, 200,..., 1000}, xticklabels={100, 200,,, 500,,,,, 1000},
	]
	\addplot[only marks, Accent1] table[y index=3]{creep_cas4_gdl1_MCR.txt};
	\addplot[no marks, black, domain={100:1000}] {4.2e17*x^(-5.45)} node [midway, above right] {$\tau_f\sim \sigma^{-5.5}$};
\end{loglogaxis}
\end{tikzpicture}

\begin{block}{Basquin law}
\begin{itemize}
\item fatigue (oscillatory stress)
\item heterogeneous solids
\item no yield stress
\end{itemize}
\end{block}

\column{0.45\textwidth}
\begin{tikzpicture}
\begin{groupplot}[
	group style={
		group name=g, group size=1 by 2,
		},
	width=\textwidth,
	height=0.7\textwidth,
	ymode=log,
	ylabel={$\tau_f$ (\si{\second})},
	ymin=10, ymax=1e6,
	]
	\nextgroupplot[
		xlabel={$\sigma$ (\si{\pascal})},
		xtick={200,600,1000},
		]
		\addplot[only marks, Accent1] table[y index=3]{creep_cas4_gdl1_MCR.txt};
		\addplot[no marks, black, domain={100:1000}] {6.5e5*exp(-x/84)} node [pos=0.25, above right] {$\tau_f\nsim e^{-\frac{\sigma}{\sigma_0}}$};
		
	\nextgroupplot[
		xlabel={$1/\sigma$ (\si{\per\pascal})},
		]
		\addplot[only marks, Accent1] table[x expr=1/\thisrowno{0}, y index=3]{creep_cas4_gdl1_MCR.txt};
		\addplot[no marks, black, domain={1e-3:7e-3}] {20.3*exp(1720*x)} node [pos=0.25, below right] {$\tau_f\nsim e^\frac{\sigma_0}{\sigma}$};
\end{groupplot}
\end{tikzpicture}

Activated processes unlikely (Eyring, Griffith, Taylor, \ldots)
\end{columns}
\end{frame}
\tikzset{external/force remake=false}

%\tikzset{external/force remake}
\begin{frame}{Primary creep}
\begin{columns}
\column{0.55\textwidth}
\begin{tikzpicture}
\begin{loglogaxis}[
	name=g,
	width=\textwidth,
	height=0.8\textwidth,
	xmin=2e-2, xmax=4e5, xlabel={time (\si{\second})},
	ylabel={Strain rate $\dot{\gamma}$ (\si{\per\second})},
	cycle list name=earthy,
	no marks,
	]
	\addplot table[y index=2]{Y27_200Pa_gdot_decimated.txt} node (s200){};
	\addplot table[y index=2]{Y38_300Pa_gdot_decimated.txt} node (s300){};
	\addplot table[y index=2]{Y25_400Pa_gdot_decimated.txt}  node (s400){};
	\addplot table[y index=2]{Y32_550Pa_gdot_decimated.txt}  node (s550){};
	\addplot table[y index=2]{Y39_1000Pa_gdot_decimated.txt} node (s1000){};
	\addplot[Accent2, ultra thick, domain={0.1:1e3}] {0.01*x^(-0.85)} node[pos=0.75, below left] {$\dot{\gamma}(t)\sim t^{-0.85}$};
\end{loglogaxis}
\begin{scope}[anchor=base east, every node/.style={ rotate=90}]
	\node[red!40!black] at (s200 |- g.outer north) {\SI{200}{\pascal}};
	\node[red!60!black] at (s300 |- g.outer north) {300};
	\node[red!80!black] at (s400 |- g.outer north) {400};
	\node[red] at (s550 |- g.outer north) {550};
	\node[red!80!yellow] at (s1000 |- g.outer north) {1000};
\end{scope}
\end{tikzpicture}

\begin{block}{Primary creep $\Rightarrow$ ``Andrade'' creep}
Power-law creep known in \alert{solids} for over a century

Classically explained by disinclination dynamics in crystals
\end{block}

\column{0.45\textwidth}
\includegraphics[width=\textwidth]{Andrade_1910}

\bigskip
\includegraphics[height=7\baselineskip]{Andrade_1910_fig1}
\begin{tikzpicture}
\node[inner sep=0] (a){\includegraphics[height=7\baselineskip]{Andrade_1910_fig2}};
\node[below right=0 and 1em of a.north west, font=\footnotesize,fill=white] {lead \& copper};
\node[above left=1em of a.south east, Accent2] {$\gamma-\gamma_0 \sim t^{1/3}$};
\end{tikzpicture}
\end{columns}
\end{frame}
\tikzset{external/force remake=false}

%\tikzset{external/force remake}
\begin{frame}{Primary creep}
\begin{block}{Local insight from \movie[externalviewer]{ultrasounds}{Yaourt110_primary.avi}}
\begin{itemize}
	\item homogeneous \movie[externalviewer]{strain field}{Yaourt110_profils.avi} (no wall slip, no shear band)
	\item if present, plastic events are below our resolution (a few \si{\micro\metre})
\end{itemize}
\end{block}

\begin{block}{Partial creep and recovery}
\begin{tikzpicture}
\begin{axis}[
	width=\textwidth,
	height=8\baselineskip,
	xlabel={time (\si{\second})},
	ylabel={strain $\gamma$},
	xmin=-1e3, xmax=2.8e4,
	axis background/.style={fill=white},
	]
	\addplot[black, no marks] table {Y261_oscil_creep_recovery_100Pa_decimated.txt};
	\begin{scope}[<-,Main, inner sep=0]
	\draw (axis cs:0,0) -- +(1em,0) node[anchor=west]{$\sigma\leftarrow \SI{100}{\pascal}$};
	\draw (axis cs:600,0.2727007) -- +(1em,0) node[anchor=west]{$\sigma\leftarrow 0$};
	\draw (axis cs:8600,0.0159) -- +(0,1em) node[anchor=base east]{$\gamma\leftarrow 0$};
	\end{scope}
\end{axis}
\end{tikzpicture}
\end{block}

$\Rightarrow$ Primary creep is ``kinetically'' reversible (no detectable structure damage, but viscous dissipation)
\end{frame}
\tikzset{external/force remake=false}




%\tikzset{external/force remake}
\begin{frame}{Dimensionless creep response}
\begin{columns}
\column{0.55\textwidth}
\begin{tikzpicture}
\begin{loglogaxis}[
	width=\textwidth,
	height=0.8\textwidth,
	xmin=1e-5, xmax=2, xlabel={$t/\tau_f$},
	ylabel={$\dot{\gamma}/\dot{\gamma}_\text{min}$},
	cycle list name=earthy,
	clip mode=individual,
	]
	\addplot table[x index=1, y index=3]{Y27_200Pa_gdot_decimated.txt} node (s200){};
	\addplot table[x index=1, y index=3]{Y38_300Pa_gdot_decimated.txt} node (s300){};
	\addplot table[x index=1, y index=3]{Y25_400Pa_gdot_decimated.txt}  node (s400){};
	\addplot table[x index=1, y index=3]{Y32_550Pa_gdot_decimated.txt}  node (s550){};
	\addplot table[x index=1, y index=3]{Y39_1000Pa_gdot_decimated.txt} node (s1000){};
	\addplot[yellow, ultra thick, domain={1e-4:0.1}] {0.378*x^(-0.85)} node[Accent2,pos=0.75, below left] {$\frac{\dot{\gamma}(t)}{\dot{\gamma}_\text{min}}\sim \left(\frac{t}{\tau_f}\right)^{-0.85}$};
\end{loglogaxis}
\end{tikzpicture}

Linear viscoelasticity accounts for the Andrade exponent

\column{0.45\textwidth}
\begin{tikzpicture}
\begin{loglogaxis}[
	height=0.8\textwidth,
	width=\textwidth,
	xlabel={Frequency (\si{\hertz})}, ylabel={\textcolor{Accent1}{$G^\prime$}, \textcolor{Accent2}{$G^{\prime\prime}$}(\si{\pascal})},
	domain={6e-2:70},
	ymin=10,
	]
	\addplot[Accent1, only marks, mark=*] table{freqsweep_Y265_cas4_GDL1.txt};
	\addplot[Accent2, only marks, mark=o] table[y=LossModulus]{freqsweep_Y265_cas4_GDL1.txt};
	\addplot[Main, no marks]{100*x^0.15}  node[midway, below=1em] {$G^\prime\sim G^{\prime\prime}\sim f^{0.15}$};
	
\end{loglogaxis}
\end{tikzpicture}
\textcolor{Main}{
\begin{align*}
\Rightarrow & J(t) \equiv& \frac{\gamma(t)}{\sigma}\sim& t^{0.15}\\
\Rightarrow & \dot{\gamma}(t)\sim& t^{0.15-1} =& t^{-0.85}
\end{align*}
}
\end{columns}
\end{frame}
\tikzset{external/force remake=false}

%\tikzset{external/force remake}
\begin{frame}{Tertiary creep}
\begin{columns}
\column{0.55\textwidth}
Time to failure

\begin{tikzpicture}
\begin{loglogaxis}[
	width=\textwidth,
	height=0.8\textwidth,
	xmin=1e-5, xmax=2, xlabel={$(\tau_f-t)/\tau_f$},
	x dir=reverse,
	ylabel={$\dot{\gamma}/\dot{\gamma}_\text{min}$},
	cycle list name=earthy,
	clip mode=individual,
	]
	\addplot table[x expr=1-\thisrowno{1}, y index=3]{Y27_200Pa_gdot_decimated.txt} node (s200){};
	\addplot table[x expr=1-\thisrowno{1}, y index=3]{Y38_300Pa_gdot_decimated.txt} node (s300){};
	\addplot table[x expr=1-\thisrowno{1}, y index=3]{Y25_400Pa_gdot_decimated.txt}  node (s400){};
	\addplot table[x expr=1-\thisrowno{1}, y index=3]{Y32_550Pa_gdot_decimated.txt}  node (s550){};
	\addplot table[x expr=1-\thisrowno{1}, y index=3]{Y39_1000Pa_gdot_decimated.txt} node (s1000){};
	\addplot[yellow, ultra thick, domain={1e-4:0.1}] {0.187/x} node[Accent2, pos=0.6, below right, inner sep=0] {$\frac{\dot{\gamma}(t)}{\dot{\gamma}_\text{min}}\sim \left(\frac{\tau_f}{\tau_f-t}\right)$};
\end{loglogaxis}
\end{tikzpicture}

Close to failure: 
\begin{itemize}
\item finite time singularity
\item logarithmic divergence
\end{itemize}
$\Rightarrow$ dominated by fracture growth

\column{0.45\textwidth}
\begin{block}{Fracture growth}
\includegraphics[width=0.33\textwidth]{Y110_2013-03-01_03-15-00.jpg}\hfill
\includegraphics[width=0.33\textwidth]{Y110_2013-03-01_03-20-00.jpg}\hfill
\includegraphics[width=0.33\textwidth]{Y110_2013-03-01_03-21-15.jpg}

\begin{tikzpicture}
\begin{axis}[
	width=\textwidth,
	height=0.8\textwidth,
	xmin=1e-3, xmax=1, xlabel={$(\tau_f-t)/\tau_f$},
	xmode=log, x dir=reverse,
	ylabel={$\ell/H$},
	ymin=0, ymax=0.5,
	clip mode=individual,
	axis background/.style={fill=white},
	]
	\addplot[only marks, Accent1, error bars/.cd, y dir=both, y explicit] table[y error index=2]{Y110_fracturelength.txt};
	\addplot[Accent2, ultra thick, no marks, domain={1e-3:1}] {- 0.06 * ln(x)} node [pos=0.25,left=1em] {$\frac{\ell}{H}\sim \ln\frac{\tau_f-t}{\tau_f}$};
\end{axis}
\end{tikzpicture}
\end{block}
\end{columns}
\end{frame}
\tikzset{external/force remake=false}

%\tikzset{external/force remake}
\begin{frame}{Secondary creep: a crossover}
\begin{columns}
\column{0.55\textwidth}
Creep responses in linear scales

\begin{tikzpicture}
\begin{axis}[
	width=\textwidth,
	height=0.8\textwidth,
	xmin=0, xmax=1, xlabel={$t/\tau_f$},
	ylabel={$\dot{\gamma}/\dot{\gamma}_\text{min}$},
	ymin=0.5, ymax=4,
	cycle list name=earthy,
	restrict y to domain=0.5:10,
	clip mode=individual,
	]
	\addplot table[x index=1, y index=3]{Y27_200Pa_gdot_decimated.txt} node (s200){};
	\addplot table[x index=1, y index=3]{Y38_300Pa_gdot_decimated.txt} node (s300){};
	\addplot table[x index=1, y index=3]{Y25_400Pa_gdot_decimated.txt}  node (s400){};
	\addplot table[x index=1, y index=3]{Y32_550Pa_gdot_decimated.txt}  node (s550){};
	\addplot table[x index=1, y index=3]{Y39_1000Pa_gdot_decimated.txt} node (s1000){};
	\addplot[yellow, ultra thick, domain={0.01:0.99}] {0.378*x^(-0.85) + 0.187/(1-x)};
	\draw[<-] (axis cs:0.56,1) -- +(0,1em) node[above] {$\tau_\text{min}$};
\end{axis}
\end{tikzpicture}
\begin{block}{Master curve}
\[\frac{\dot{\gamma}(t)}{\dot{\gamma}_\text{min}} = \underbrace{\lambda \left(\frac{t}{\tau_f}\right)^{-0.85}}_\text{Andrade} + \underbrace{\frac{\mu}{1 - t/\tau_f}}_\text{fractures}\]
\end{block}


\column{0.45\textwidth}
\structure{Monkman-Grant relation}
\begin{tikzpicture}
\begin{loglogaxis}[
	width=\textwidth,
	height=0.8\textwidth,
	xlabel={$\tau_f$ (\si{\second})},
	ylabel={$\tau_\text{min}$ (\si{\second})},
	xmin=10, xmax=1e6, ymin=10, ymax=1e6,
	]
	\addplot[only marks, Accent1] table[x index=3, y index=4]{creep_cas4_gdl1_MCR.txt};
	\addplot[no marks, black, domain={10:1e6}] {0.56*x)};
\end{loglogaxis}
\end{tikzpicture}
\[\tau_\text{min} \approx 0.6 \tau_f\]
``Early time'' response allows to predict failure time
\end{columns}
\end{frame}
\tikzset{external/force remake=false}

\begin{frame}{What about theories and models?}
\begin{columns}
\column{0.35\textwidth}
\begin{block}{Discrete dislocation dynamics (dense)}
\textit{\footnotesize Miguel et al., PRL 2002}

\hspace{1em}\includegraphics[width=\textwidth-2em]{Miguel_2002_img}

\medskip
\hspace{1em}\includegraphics[width=\textwidth-2em]{Miguel_2002_graph}

predicts Andrade creep \alert{but} no fracture
\end{block}
\column{0.65\textwidth}
\begin{columns}
\column{0.5\textwidth+1em}
\structure{Fibre bundle model~\#1}

\textit{\footnotesize Jagla et al., PRE 2011}
\begin{itemize}
\item elastic fibres
\item local yield strain
\item local coupling
\end{itemize}

\hspace{1em}\includegraphics[width=\textwidth-1em]{Jagla_2011_gamma}
\[\tau_f\sim\left(\sigma-\alert{\sigma_c}\right)^{-1.25}\]

\column{0.5\textwidth-1em}
\includegraphics[width=\textwidth]{Jagla_2011_3regimes}
\end{columns}
predicts Andrade creep and failure \alert{but} critical stress
\end{columns}
\end{frame}


\begin{frame}{What about theories and models?}
\structure{Fibre bundle model~\#2}

elastic fibres + local yield strain + damage accumulation
\begin{block}{\textit{\footnotesize Kun et al. J. Stat Mech 2006, PRL 2008}}
\begin{columns}
\column{0.7\textwidth}
\begin{tikzpicture}[inner sep=0]
\node(a){\includegraphics[height=5\baselineskip]{Kun_2006_basquin.png}};
\node[fill=white,yshift=-0.5em] at (a) {$\tau_f \sim \sigma^{-\beta}$};
\end{tikzpicture}\,
\includegraphics[height=5\baselineskip]{Kun_2008_Basquin}\,
\includegraphics[height=5\baselineskip]{asphalt}
\column{0.3\textwidth}
$\beta$ exponent of damage accumulation
\[\dot{\gamma}\sim \left(\tau_f-t\right)^{-1/(1+\beta)}\]
\end{columns}
\end{block}

\begin{columns}
\column{0.65\textwidth}
\begin{block}{\textit{\footnotesize Halász et al., PRE 2012}}
\tikzsetnextfilename{Halasz}
\begin{tikzpicture}[inner sep=0, ultra thick, font=\footnotesize]
\node (g) {\includegraphics[height=7\baselineskip]{Halasz_2012_Basquin}};
\node[draw=green!50!black, below right=0 and \baselineskip of g.north east] (c) {\includegraphics[height=4\baselineskip]{Halasz_2012_crack}};
\node[draw=red, above right=0 and 6\baselineskip of g.south east] (d) {\includegraphics[height=4\baselineskip]{Halasz_2012_disordered}};
\draw[->,green!50!black] (c.west) -- ++(-2\baselineskip,0) node[above] {$\beta=5$} node[below right=0 of c.north east, text width=5\baselineskip]{single crack propagation};
\draw[->, red] (d.south west) -- +(-5\baselineskip,0) --($(g.south east) +(0, \baselineskip)$) node[below=1em] {$\beta=1$} node[above left=0 of d.south west] {diffusive damage};
\end{tikzpicture}
\end{block}
\column{0.35\textwidth}
Predicts Basquin law \& fractures \alert{but} no Andrade creep \& too slow divergence
\end{columns}
\end{frame}

\begin{frame}{Creep and yielding summary}
\begin{itemize}
\item failure involves a single time scale $\tau_f \sim \sigma^{-5.5}$ Basquin law
\item reversible, homogeneous creep $\rightarrow$ irreversible fracture growth
\item a single expression captures all the global rheological response
\end{itemize}
\begin{tikzpicture}
	\begin{groupplot}[%
		group style={
			group name=g, group size=3 by 1,
			horizontal sep=3em,
			},
		width=0.35\textwidth,
		cycle list name=earthy,
		clip mode=individual,
		]
	\nextgroupplot[
		xmin=1e-5, xmax=2, xlabel={$t/\tau_f$},
		ylabel={$\dot{\gamma}/\dot{\gamma}_\text{min}$},
		xmode=log,ymode=log,
		xtick={1e-4, 1e-2, 1},
		]
		\addplot table[x index=1, y index=3]{Y27_200Pa_gdot_decimated.txt};
		\addplot table[x index=1, y index=3]{Y38_300Pa_gdot_decimated.txt};
		\addplot table[x index=1, y index=3]{Y25_400Pa_gdot_decimated.txt};
		\addplot table[x index=1, y index=3]{Y32_550Pa_gdot_decimated.txt};
		\addplot table[x index=1, y index=3]{Y39_1000Pa_gdot_decimated.txt};
		\addplot[yellow, ultra thick, samples at={1e-5,1e-4, 1e-3,1e-2,0.1,0.2,0.3,0.4,0.5,0.6,0.7,0.8,0.9, 0.99, 0.999, 0.9999, 0.99999}] {0.378*x^(-0.85) + 0.187/(1-x)};
		
	\nextgroupplot[
		xmin=0, xmax=1, xlabel={$t/\tau_f$},
		ymin=0.5, ymax=4,
		restrict y to domain=0.5:10,
		]
		\addplot table[x index=1, y index=3]{Y27_200Pa_gdot_decimated.txt};
		\addplot table[x index=1, y index=3]{Y38_300Pa_gdot_decimated.txt};
		\addplot table[x index=1, y index=3]{Y25_400Pa_gdot_decimated.txt};
		\addplot table[x index=1, y index=3]{Y32_550Pa_gdot_decimated.txt};
		\addplot table[x index=1, y index=3]{Y39_1000Pa_gdot_decimated.txt};
		\addplot[yellow, ultra thick, domain={0.01:0.99}] {0.378*x^(-0.85) + 0.187/(1-x)};
		
	\nextgroupplot[
		xmin=1e-5, xmax=2, xlabel={$(\tau_f-t)/\tau_f$},
		x dir=reverse,
		xmode=log,ymode=log,
		xtick={1e-4, 1e-2, 1},
	]
		\addplot table[x expr=1-\thisrowno{1}, y index=3]{Y27_200Pa_gdot_decimated.txt};
		\addplot table[x expr=1-\thisrowno{1}, y index=3]{Y38_300Pa_gdot_decimated.txt};
		\addplot table[x expr=1-\thisrowno{1}, y index=3]{Y25_400Pa_gdot_decimated.txt};
		\addplot table[x expr=1-\thisrowno{1}, y index=3]{Y32_550Pa_gdot_decimated.txt};
		\addplot table[x expr=1-\thisrowno{1}, y index=3]{Y39_1000Pa_gdot_decimated.txt};
		\addplot[yellow, ultra thick, samples at={1e-5,1e-4, 1e-3,1e-2,0.1,0.2,0.3,0.4,0.5,0.6,0.7,0.8,0.9, 0.99, 0.999, 0.9999, 0.99999}] {0.378*(1-x)^(-0.85) + 0.187/x};
	\end{groupplot}
\end{tikzpicture}
\begin{itemize}
\item a model soft solid well captured by fibre-bundle models
\end{itemize}
\end{frame}

