\documentclass[twocolumn,superscriptaddress,showpacs,preprintnumbers,amsmath,amssymb,prl]{revtex4-1}
\usepackage{siunitx}
\usepackage{tikz}
\usetikzlibrary{arrows,shapes,backgrounds, calc, positioning, topaths,chains, intersections, decorations.markings, shapes.geometric, matrix,patterns,mindmap,fit}
%\usetikzlibrary{positioning, patterns,topaths,chains,matrix}

\usepackage{pgfplots}
\pgfplotsset{compat=1.9}
\usepgfplotslibrary{groupplots}
\usepgfplotslibrary{external}
\tikzsetexternalprefix{fig_plis/}
\tikzexternalize
\tikzset{external/force remake}



\definecolor{Main}{rgb}{1, 0.57, 0}
\definecolor{Accent1}{rgb}{1,0.28,0}
\definecolor{Accent2}{rgb}{1,0.74,0}

\begin{document}
\author{Mathieu Leocmach}

\begin{figure}
	\tikzsetnextfilename{dynamics}
	% \begin{figure*}
% 	\begin{tikzpicture}
% 	\matrix[matrix of nodes, inner sep=0, column sep=0.015\textwidth, row sep=0.5em] (m){
% 	33 min & 38 min & 43 min & 48 min & 1h & 1h15 & 2h30\\
% 	\includegraphics[width=0.13\textwidth]{prise_0100_color.jpg}&
% 	\includegraphics[width=0.13\textwidth]{prise_0130_color.jpg}&
% 	\includegraphics[width=0.13\textwidth]{prise_0160_color.jpg}&
% 	\includegraphics[width=0.13\textwidth]{prise_0190_color.jpg}&
% 	\includegraphics[width=0.13\textwidth]{prise_0250_color.jpg}&
% 	\includegraphics[width=0.13\textwidth]{prise_0360_color.jpg}&
% 	\includegraphics[width=0.13\textwidth]{prise_0799_color.jpg}\\
% 	\includegraphics[width=0.13\textwidth]{cas3p2_fluo0p8_GDL4_2_t047_crop_resized.jpg}&
% 	\includegraphics[width=0.13\textwidth]{cas3p2_fluo0p8_GDL4_2_t056_crop_resized.jpg}&
% 	\includegraphics[width=0.13\textwidth]{cas3p2_fluo0p8_GDL4_2_t065_crop_resized.jpg}&
% 	\includegraphics[width=0.13\textwidth]{cas3p2_fluo0p8_GDL4_2_t074_crop_resized.jpg}&
% 	\includegraphics[width=0.13\textwidth]{cas3p2_fluo0p8_GDL4_2_t092_crop_resized.jpg}&
% 	\includegraphics[width=0.13\textwidth]{cas3p2_fluo0p8_GDL4_2_t125_crop_resized.jpg}&
% 	\includegraphics[width=0.13\textwidth]{cas3p2_fluo0p8_GDL4_2_t260_crop_resized.jpg}\\
% 	};
% 	\draw[ultra thick] ++(m-2-1.south west) -- ++(0.023\textwidth,0);
% 	\draw[ultra thick] ++(m-3-1.south west) -- ++(0.1\textwidth,0);
% 	\end{tikzpicture}
% 	\caption{Dynamics of pattern formation for $e\approx\SI{100}{\micro\metre}$. (top) By light transmission microscopy. (bottom) Reconstructed from fluorescent confocal microscopy (corresponds to the squared area in Fig.~\ref{fig:acidgel}e). Scale bars are \SI{1}{\milli\metre}.}
% 	\label{fig:dynamics}
% \end{figure*}
%\begin{figure}
\begin{tikzpicture}
\matrix[matrix of nodes, inner sep=0, column sep=0.05\columnwidth, row sep=0.5em, column 1/.style={base left},nodes={anchor=center},] (m){
	\SI{33}{\minute} & \includegraphics[height=4em]{prise_0100_color.jpg}&
	\includegraphics[height=4em]{cas3p2_fluo0p8_GDL4_2_t047_crop_resized.jpg}\\
	\SI{38}{\minute} & \includegraphics[height=4em]{prise_0130_color.jpg}&
	\includegraphics[height=4em]{cas3p2_fluo0p8_GDL4_2_t056_crop_resized.jpg}\\
	\SI{43}{\minute} & \includegraphics[height=4em]{prise_0160_color.jpg}&
	\includegraphics[height=4em]{cas3p2_fluo0p8_GDL4_2_t065_crop_resized.jpg}\\
	\SI{48}{\minute} & \includegraphics[height=4em]{prise_0190_color.jpg}&
	\includegraphics[height=4em]{cas3p2_fluo0p8_GDL4_2_t074_crop_resized.jpg}\\
	\SI{1}{\hour} & \includegraphics[height=4em]{prise_0250_color.jpg}&
	\includegraphics[height=4em]{cas3p2_fluo0p8_GDL4_2_t092_crop_resized.jpg}\\
	\SI{1}{\hour} 15 & \includegraphics[height=4em]{prise_0360_color.jpg}&
	\includegraphics[height=4em]{cas3p2_fluo0p8_GDL4_2_t125_crop_resized.jpg}\\
	\SI{2}{\hour} 30 & \includegraphics[height=4em]{prise_0799_color.jpg}&
	\includegraphics[height=4em]{cas3p2_fluo0p8_GDL4_2_t260_crop_resized.jpg}\\
};
\newdimen\mydima
\newdimen\mydimb
\pgfextractx{\mydima}{\pgfpointanchor{m-1-2}{east}}
\pgfextractx{\mydimb}{\pgfpointanchor{m-1-2}{west}}
 	\draw[ultra thick] ++(m-1-2.south west) -- ++(0.17\mydima-0.17\mydimb,0);
\newdimen\mydimc
\newdimen\mydimd
\pgfextractx{\mydimc}{\pgfpointanchor{m-1-3}{east}}
\pgfextractx{\mydimd}{\pgfpointanchor{m-1-3}{west}}
 	\draw[ultra thick] ++(m-1-3.south west) -- ++(0.76\mydima-0.76\mydimb,0);
% 	\draw[ultra thick] ++(m-1-3.south west) -- ++(0.1\textwidth,0);
%\draw (m.north west) rectangle +(\columnwidth, -\textheight);
\begin{scope}[every node/.style={anchor=north east, text height=0.8em, text depth=0.2em}]
\node at (m-1-2.north west) {(a)};
\node at (m-1-3.north east) {(b)};
\end{scope}
\end{tikzpicture}

% 	\caption{Dynamics of pattern formation. Left: light transmission microscopy. Right: Reconstructed from fluorescent confocal microscopy. Scale bars are \SI{1}{\milli\metre}.}
%  	\label{fig:dynamics}
% \end{figure}
	\caption{Dynamics of pattern formation. Left: light transmission microscopy. Right: Reconstructed from fluorescent confocal microscopy. Scale bars are \SI{1}{\milli\metre}.}
	\label{fig:dynamics}
\end{figure}

\begin{figure}
	\tikzsetnextfilename{acidification}
	%\begin{figure}
\begin{tikzpicture}
	\begin{groupplot}[%
		group style={
			group name=g, group size=1 by 3,
			x descriptions at=edge bottom,
			vertical sep=0.5em,
			},
		xlabel={time (\si{\hour})},
		xmin=0,xmax=4, ymin=0,
		scale only axis,
		width=\columnwidth-4em,
		height=0.3\columnwidth,
		extra tick style={grid=major},%
		ylabel absolute, every axis y label/.append style={anchor=base, yshift=-1em}
		]
	\nextgroupplot[
		ymax=7, ylabel=pH,
		extra y ticks={4.6}, extra y tick labels={},%
		]
	\addplot+[no marks,black] table[x expr={\thisrowno{0}/3600.+0.05}]{Y189_28800s.pH};
	\node[base left=0] at (axis cs:8,4.6) {isoelectric};

	\nextgroupplot[
		ylabel={$G^\prime$ (\si{\pascal})}
		]
	\addplot+[no marks, black] table[x expr={\thisrowno{0}/3600.+0.05}]{Y235_28800s.prise};

	\nextgroupplot[
		ylabel={$\xi$ (\si{\micro\metre})}, restrict y to domain=0:10,
		]
	\addplot+[no marks, black] table[x expr={\thisrowno{0}/3600.+0.2}]{ech14_pore_size.txt};
	\addplot+[no marks, black] table[x expr={\thisrowno{0}/3600.+0.3}]{ech12_pore_size.txt};
	\end{groupplot}
\begin{scope}[every node/.style={anchor=south east, text height=0.8em, text depth=0.2em}]
\node at (g c1r1.south east) {(a)};
\node at (g c1r2.south east) {(b)};
\node at (g c1r3.south east) {(c)};
\end{scope}
\end{tikzpicture}
%\end{figure}
	\caption{Acid-induced protein gels properties behave non-monotonously with pH. (a) pH decrease in 4\%w sodium caseinate solution acidified by 4\%w GDL. Horizontal line indicates isoelectric pH of casen (b) Evolution of storage modulus measured in a rheometer (full adhesion to rotor and stator). (c) Evolution of pore size measured by confocal microscopy in a half coated microscopy cell (no adhesion to cell ceiling, syn\ae{}resis and swelling allowed).}
	\label{fig:acidification}
\end{figure}

\begin{figure}
	\tikzsetnextfilename{sideview}
	\begin{figure}
\begin{tikzpicture}
	\begin{groupplot}[%
		group style={
			group name=g, group size=1 by 2,
			x descriptions at=edge bottom,
			vertical sep=0.5em,
			},
		xmin=0, xmax=180, xtick={0,30,...,150},
		extra tick style={grid=major},%
		extra x tick labels={},%
		scale only axis,
		width=\columnwidth-4em,
		height=6\baselineskip,
		ylabel absolute, every axis y label/.append style={anchor=base, yshift=-1em, xshift=0.5em}
		]
	
	\nextgroupplot[ylabel={Volume (\%)}, ymin=20, ymax=100, ytick={40,60,80,100}]
	\addplot+[no marks,black] table[x expr={\thisrowno{0}+15}, y expr={\thisrowno{1}*100}]{relative_volume_excess_area_plis.txt}  coordinate[pos=0] (V0) (V0) |- (axis cs:0,100);
	%\addplot+[only marks,Accent2, mark=+] table[y expr={\thisrowno{1}*100}]{volume_rel_half_cas8_toi.txt};
	
	\nextgroupplot[ylabel={Excess area (\%)}, ymin=0, ymax=6, ytick={0,2,4,6}, 
	xlabel={time (min)}, every axis y label/.append style={xshift=-1em}]
	\addplot+[no marks,black] table[x expr={\thisrowno{0}+15}, y expr={\thisrowno{2}*100}]{relative_volume_excess_area_plis.txt};
	%\addplot+[only marks,Accent2, mark=+] table{excess_area_pc_half_cas8_toi.txt};
	\end{groupplot}

	\matrix[matrix of nodes, matrix anchor=south east, inner sep=0, row sep=0.2em, nodes={anchor=west}, column sep=0.1em]  (m) at ($(g c1r1.north east) +(0,1em)$) {
	\SI{10}{\minute} & \includegraphics[width=0.88\columnwidth, height=0.054\columnwidth]{coupe_cloque_t000.png}\\
	\SI{23}{\minute} & \includegraphics[width=0.88\columnwidth, height=0.052\columnwidth]{coupe_plis_t016.png}\\
	%\includegraphics[width=\columnwidth, height=0.052\columnwidth]{coupe_plis_t032.png} & \SI{32}{\minute}\\
	\SI{35}{\minute} & \includegraphics[width=0.88\columnwidth, height=0.046\columnwidth]{coupe_plis_t038.png} \\
	\SI{36}{\minute} & \includegraphics[width=0.88\columnwidth, height=0.046\columnwidth]{coupe_plis_t040.png} \\
	\SI{38}{\minute} & \includegraphics[width=0.88\columnwidth, height=0.046\columnwidth]{coupe_plis_t043.png} & \\
	\SI{44}{\minute} & \includegraphics[width=0.88\columnwidth, height=0.046\columnwidth]{coupe_plis_t055.png}\\
	\SI{53}{\minute} & \includegraphics[width=0.88\columnwidth, height=0.046\columnwidth]{coupe_plis_t070.png} \\
	\SI{1}{\hour}~21 & \includegraphics[width=0.88\columnwidth, height=0.046\columnwidth]{coupe_plis_t123.png} \\
	%\includegraphics[width=\columnwidth, height=0.052\columnwidth]{coupe_plis_t332.png} & \SI{3}{\hour}~15\\
	};
\newdimen\mydima
\newdimen\mydimb
\pgfextractx{\mydima}{\pgfpointanchor{m-1-2}{west}}
\pgfextractx{\mydimb}{\pgfpointanchor{m-1-2}{east}}
	\draw[line width=0.3em, white] ($(m-1-2.south west)+(0.75em,0.75em)$) -- +(0.0786\mydimb-0.0786\mydima,0);

	\begin{axis}[
	name=a,
	anchor=south east,
	at={($(m.north east)+(0,2em)$)},
	width=\columnwidth, height=0.25\columnwidth, scale only axis,
	domain=-0.25*pi:2.25*pi, no markers, ymin=-2, ymax=3,xmin=0,xmax=2*pi,
	axis lines=none, xtick=\empty,
	]
	\fill[lightgray] (axis cs:0.6*pi,-2) rectangle (axis cs:0.75*pi,3);
	%\fill[gray!20] (axis cs:85,5) rectangle (axis cs:95,-4) (axis cs:265,5) rectangle (axis cs:275,-4);
	%\addplot+[Accent1] {sin(deg(x))};
%  	\addplot+[Accent1,variable=\t, name path=top] (
% 		{t-0.75*cos(deg(t))/sqrt(100/pi^2+cos(deg(t))^2)}, 
% 		{sin(deg(t))+0.75/sqrt(1+(pi/10*cos(deg(t)))^2)});
%  	\addplot+[Accent1,variable=\t, name path=bot] (
% 		{t+0.75*cos(deg(t))/sqrt(100/pi^2+cos(deg(t))^2)}, 
% 		{sin(deg(t))-0.75/sqrt(1+(pi/10*cos(deg(t)))^2)});
	%\addplot+[Accent1] ({rad(x)-0.75*cos(x)/sqrt(1+cos(x)^2)}, {sin(x)+0.75/sqrt(1+cos(x)^2)}) coordinate[pos=0.25] (M1);
	%\addplot+[Accent1] ({rad(x)+0.75*cos(x)/sqrt(1+cos(x)^2)}, {sin(x)-0.75/sqrt(1+cos(x)^2)}) coordinate[pos=0.25] (M1);
	%\addplot+[Accent1]({x+0.75*cos(x)/sqrt(cos(x)^2+1)}, {1+sin(x)-0.75/sqrt(cos(x)^2+1)}) coordinate[pos=0.2475] (M2);
	\addplot+[Accent1, line width=2em] {sin(deg(x))};
	\addplot+[draw=none,name path=top] {sin(deg(x))+0.75};
	\addplot+[draw=none,name path=bot] {sin(deg(x))-0.75};
	%\addplot+[Accent1] {sin(x)+0.75};
 	\path[name path=Z] (axis cs:pi-1,-100/pi^2) -- (axis cs:pi+1,100/pi^2);
 	\path[name path=D, name intersections={of=top and Z, by=Z2}] (Z2) -- +(0,-6\baselineskip);
 	\draw[ultra thick, Accent2, name intersections={of=top and Z, by=Z2}, name intersections={of=bot and D, by=D1}] (Z2) --(D1) (axis cs:0.5*pi,0.25) -- (axis cs:0.5*pi,1.75);
	
	%\addplot+[dashed, black]{0};
	\addplot+[dashed, help lines]{0.75};
	\addplot+[dashed, help lines]{-0.75};
	\draw[<->] (axis cs:1.5*pi,0.75) -- (axis cs:1.5*pi,3) node[midway, right] {$H_2$}; 
	\draw[<->] (axis cs:0.1*pi,-0.75) -- (axis cs:0.1*pi,-2) node[midway, right] {$H_1$};
 	%\draw[<->, help lines] (axis cs:0.5*pi,0) -- (axis cs:0.5*pi,1) node[midway, left] {$w$};
	\draw[->] (axis cs:0.5*pi,-0.75) -- (axis cs:0.5*pi,0.25) node[midway, left] {$w(x,t)$};
 	\draw[<->] (axis cs:0.5*pi,0.25) -- (axis cs:0.5*pi,1.75) node[midway, left] {$h$};
  	\draw[<->, name intersections={of=bot and Z, by=Z1}, name intersections={of=top and Z, by=Z2}] (Z1) -- (Z2) node[midway, anchor=south east, inner sep=0] {$h$};
	
	\draw[->, ultra thick] (axis cs:0.6*pi,2.25) + (-0.5em, 0) -- +(0.5em,0) node[pos=0, left] {$Q_2(x,t)$};
	\draw[->, ultra thick] (axis cs:0.75*pi,2.25) + (-0.5em, 0) -- +(0.5em,0) node[pos=1, right] {$Q_2(x+dx,t)$};
	\draw[->, ultra thick] (axis cs:0.6*pi,-1.5) + (-0.5em, 0) -- +(0.5em,0) node[pos=0, left] {$Q_1(x,t)$};
	\draw[->, ultra thick] (axis cs:0.75*pi,-1.5) + (-0.5em, 0) -- +(0.5em,0) node[pos=1, right] {$Q_1(x+dx,t)$};
	\node at (axis cs:0.675*pi,2.25) {$P_2$};
	\node at (axis cs:0.675*pi,-1.5) {$P_1$};
	\draw[->, lightgray, ultra thick] (axis cs:0.675*pi,0.25) -- (axis cs:0.675*pi, 1.5) node[pos=0, right] {$v_{Darcy}$};

	\draw[->] (axis cs:1.8*pi,-1.75) -- +(1.5em,0) node[right]{$x$};
	\draw[->] (axis cs:1.8*pi,-1.75) -- +(0,1.5em) node[right]{$z$};
	
	%\node[below] at (axis cs:90,-1.75) (pb1) {$P_1$};
	%\node[above] at (axis cs:90,1.75) (ph2) {$P_2$};
	%\node[above] at (axis cs:270,1.75) (ph1) {$P_1$};
	%\node[below] at (axis cs:270,-1.75) (pb2) {$P_2$};
	%\draw[line width=0.1em, ->] (ph2) -- (axis cs:90,0.25) node[midway, left] {Darcy} node[midway, right] {$v$};
	%\draw[line width=0.1em, ->] (pb2) -- (pb1) node[midway, above] {Poiseuille} node[midway, below] {$u \sim \frac{\lambda}{H} v$};
\end{axis}
\fill[pattern=north east lines,pattern color=Accent2] (a.south west) rectangle +(\columnwidth,-1em) (a.north west) rectangle +(\columnwidth,1em);

%\draw (g c1r2.outer south east) rectangle +(-\columnwidth,\textheight);

\begin{scope}[every node/.style={anchor=north east, text height=0.8em, text depth=0.2em}]
\node at (a.north east) {(a)};
\node[white] at (m-1-2.north east) {(b)};
\node at (g c1r1.north east) {(c)};
\node at (g c1r2.north east) {(d)};
\end{scope}
\end{tikzpicture}
	
\end{figure}
	\caption{Modelisation  and measurements. (a) Schematic side view of the cell. The constant thickness gel sheet (orange) is surrounded by water. Yellow lines show the path of transmitted light through the gel phase. A region of interest is highlighted in gray. (b) Confocal XZ cuts showing syn\ae{}resis, swelling and (cascade) wrinkling. Scale bar is \SI{100}{\micro\metre} (real size ratio). Evolution of (c) volume of the gel phase relative to cell volume and (d) excess area measured by confocal microscopy.}
	\label{fig:sideview}
\end{figure}

\begin{figure*}
	\tikzsetnextfilename{Darcy_vs_Poiseuille}
	%\begin{figure}
\begin{tikzpicture}
\begin{groupplot}[group style={
			group name=g, 
	group size=3 by 1,
	horizontal sep=1em,
	y descriptions at=edge left,
			},
	scale only axis,
	width=0.333\textwidth-1.66em,
	xmin=0, xmax=3,ymin=0,ymax=3,
	ylabel={$\lambda$ measured (\si{\milli\metre})},
	cycle list name=black white,
	no marks]
\nextgroupplot[xlabel={$\lambda$ Darcy (\si{\milli\metre})},]
%\addplot+[only marks, error bars/.cd,x dir=both,y dir=both, x explicit,y explicit] table[x=D, x error=pmD, y=B, y error=pmB]{plis.txt};
\addplot+[only marks, error bars/.cd,x dir=both,y dir=both, x explicit,y explicit] table[x=D, x error=pmD, y=I, y error=pmI]{plis.txt};


\nextgroupplot[xlabel={$\lambda$ Poiseuille (\si{\milli\metre})},]
%\fill[gray] (axis cs:0.08,0.36) circle (5pt) (axis cs:0.36,0.46) circle (5pt);
%\addplot+[only marks, error bars/.cd,x dir=both,y dir=both, x explicit,y explicit] table[x=P, x error=pmP, y=B, y error=pmB]{plis.txt};
%\addplot+[forget plot] {3.5*x};
\addplot+[only marks, error bars/.cd,x dir=both,y dir=both, x explicit,y explicit] table[x=P, x error=pmP, y=I, y error=pmI]{plis.txt};
\addplot+[black, dashed] {0.81*x};
\addplot+[black] {0.56*x+0.28};
%\addplot+[only marks, error bars/.cd,x dir=both,y dir=both, x explicit,y explicit] coordinates{(0.41, 0.36) +-(0.05,0.04)};
%\node[anchor=north west, red, font=\footnotesize] (I) at (rel axis cs:0,1) {Inter blister};
%\node[anchor=north west, blue, font=\footnotesize] (B) at (I.south west) {Blister size};

\nextgroupplot[xlabel={$\lambda$ mixte (\si{\milli\metre})},]
%\fill[gray] (axis cs:0.08,0.36) circle (5pt) (axis cs:0.36,0.46) circle (5pt);
%\addplot+[only marks, error bars/.cd,x dir=both,y dir=both, x explicit,y explicit] table[x=P, x error=pmP, y=B, y error=pmB]{plis.txt};
%\addplot+[forget plot] {3.5*x};
\addplot+[only marks, error bars/.cd,x dir=both,y dir=both, x explicit,y explicit] table[x=M, x error=pmM, y=I, y error=pmI]{plis.txt};
\addplot+[black, dashed] {0.75*x};
\end{groupplot}
%\draw (g c1r1.outer north west) rectangle +(\textwidth, -\textheight);

\begin{scope}[every node/.style={anchor=north west, text height=0.8em, text depth=0.2em}]
\node at (g c1r1.north west) {(a)};
\node at (g c2r1.north west) {(b)};
\node at (g c3r1.north west) {(c)};
\end{scope}

\end{tikzpicture}
%\caption{Comparing model predictions with measured wavelengths. Continuous line is the perfect match ($\lambda_{th}=\lambda_{xp}$), dashed line is the best linear fit through the origin (prefactor is 0.81 in a, 0.75 in b), dotted line is the best affine fit ($\lambda_{xp}=0.56*\lambda_{th}+\SI{0.28}{\milli\metre}$).}
%\end{figure}
	\caption{Comparing model predictions with measured wavelengths. Continuous line is the perfect match ($\lambda_{th}=\lambda_{xp}$), dashed line is the best linear fit through the origin (prefactor is 0.81 in a, 0.75 in b), dotted line is the best affine fit ($\lambda_{xp}=0.56\lambda_{th}+\SI{0.28}{\milli\metre}$).}
	\label{fig:DarcyPoiseuille}
\end{figure*}

\begin{figure}
	\tikzsetnextfilename{mesh}
	%\begin{figure}
\begin{tikzpicture}
\begin{groupplot}[group style={
			group name=g, group size=1 by 2,
			x descriptions at=edge bottom,
			vertical sep=0.5em,
			},
		scale only axis,
		width=0.5\columnwidth,
		xlabel={q (\si{\per\micro\metre})},
		ylabel={I(q) (a.u.)},
		domain=0.1:8,
		xmode=log,
		ymode=log,
		xmin=0.03, xmax=1e2,
		ymin=0.5, ymax=16, ytick={1, 2, 4,8}, yticklabels={1, 2, 4,8},
		clip mode=individual,
	]
	\nextgroupplot
	\addplot+[only marks, mark options={scale=0.3, Accent1}] table[y expr=\thisrowno{1}/500] {ech14_t008.Sq};
	\addplot+[black, no marks] {9.47/(1+(0.73*x)^2)^0.63};%xi=2.0um, chi = 5.24, d_2D=1.27
	\nextgroupplot%[ymin=6e5, ymax=1.5e7,]
	\addplot+[only marks, mark options={scale=0.3, Accent1}] table[y expr=\thisrowno{1}/500]{ech6.Sq};
	\addplot+[black, no marks, domain=0.5:6.5] {4.47/(1+(0.15*x)^2)^1};%xi=0.31um, chi = 2.23, d_2D=2
\end{groupplot}
\coordinate (topright) at  (g c1r1.outer north east -| g c1r2.outer south east);
\newdimen\mydima
\newdimen\mydimb
\pgfextracty{\mydima}{\pgfpointanchor{g c1r1}{north}}
\pgfextracty{\mydimb}{\pgfpointanchor{g c1r1}{south}}
\pgfmathsetlength{\mydima}{\mydima-\mydimb}
\node[inner sep=0, anchor=north west] at ($(topright|-g c1r1.north)+(-\columnwidth,0)$) (im1) {\includegraphics[height=\mydima]{ech14_t008.jpg}};
%scale bar 10 um
\draw[line width=0.3em] (im1.south west) ++(1em,1em) -- +(0.197\mydima,0);
\node[inner sep=0, anchor=north west] at ($(topright|-g c1r2.north)+(-\columnwidth,0)$) (im2) {\includegraphics[height=\mydima]{ech6_x64_4.jpg}};
%scale bar 10 um
\draw[line width=0.3em] (im2.south west) ++(1em,1em) -- +(0.197\mydima,0);

\node[inner sep=0, above right= 0.1\mydima of g c1r1.south west] (sp1) {\includegraphics[width=0.5\mydima] {ech14_fft_t008.jpg}};
\draw[line width=0.3em] (sp1.south west) ++(1em,1em) --+(0.157\mydima,0); %10um-1
\node[inner sep=0, above right= 0.1\mydima of g c1r2.south west] (sp2) {\includegraphics[width=0.5\mydima] {ech6_fft.jpg}};
\draw[line width=0.3em] (sp2.south west) ++(1em,1em) --+(0.157\mydima,0);%10um-1

\begin{scope}[every node/.style={anchor=north east, text height=0.8em, text depth=0.2em}]
\node[fill=white, draw=white] at (im1.north east) {(a)};
\node at (g c1r1.north east) {(b)};
\node[fill=white, draw=white] at (im2.north east) {(c)};
\node at (g c1r2.north east) {(d)};
\end{scope}

%\draw (topright) rectangle +(-\columnwidth,-\textheight);
\end{tikzpicture}
%\end{figure}
	\caption{Gel microstructure. (a) Detail of a confocal micrograph of a 4\%w casein, 4\%w GDL in water, when only the slide is coated (not the cover slip). Time corresponds to the minimum in mesh size. Scale bar is \SI{10}{\micro\metre}. (b) Corresponding radial averaged Fourier transform. Black line is the best fit by Eq.XXX with $\xi=\SI{4.9}{\micro\metre}$ and $d=2.4$. Inset shows the center of the non radial average spectrum (arbitrary logarithmic color scale). Scale bar is \SI{10}{\per\micro\metre}. (c-d) Idem in a 50\%w glycerol solvent, with $\xi=\SI{1.5}{\micro\metre}$ and $d=2.5$.}
	\label{fig:mesh}
\end{figure}

\end{document}