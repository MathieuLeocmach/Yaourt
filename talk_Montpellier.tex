\documentclass[xcolor=table]{beamer}
\usepackage[utf8]{inputenc}
\usepackage[british]{babel}
\usepackage[super]{nth}
%\usetheme{Boadilla}
%\usecolortheme{rose}
%\usecolortheme{crane}
\usefonttheme{structuresmallcapsserif}
\setbeamertemplate{navigation symbols}{}

\definecolor{Main}{rgb}{0.74, 0.13, 0.19}
\definecolor{Accent1}{rgb}{0.76,0.36,0.13}
\definecolor{Accent2}{rgb}{0.54,0.1,0.4}

\usecolortheme{rose}
%\useinnertheme[shadow]{circles}
\usecolortheme{whale}
%\useoutertheme{infolines}

\usecolortheme[named=Accent1]{structure}

\setbeamercolor{alerted text}{fg=Accent2}
%\setbeamercolor{palette primary}{fg=white}
%\setbeamercolor{palette secondary}{bg=Accent1}
%\setbeamercolor{palette tertiary}{bg=Accent2,fg=white}


\setbeamerfont{page number in head/foot}{size=\large}
\setbeamercolor{page number in head/foot}{fg=Main}
% page/total
%\setbeamertemplate{footline}[frame number]
% pas de total
\setbeamertemplate{footline}{%
    	\hfill%
	\usebeamercolor[fg]{page number in head/foot}%
	\usebeamerfont{page number in head/foot}%
	\insertframenumber\kern1em\vskip2pt%
}

\setbeamersize{text margin left=1em}
\setbeamersize{text margin right=1em}

%font
\usepackage[T1]{fontenc}
\usepackage[oldstylenums]{kpfonts}


%proper math and math symbols
%\usepackage{amsmath}
\usepackage{amssymb}

\usepackage{datenumber,fp}

\usepackage{siunitx}

\usepackage{tabu}
\usepackage{multirow}
\usepackage{booktabs}

% Allow the usage of graphics (.jpg, .png, etc.) in the document
\usepackage{graphicx}
\usepackage{tikz}
\usetikzlibrary{arrows,shapes,backgrounds, calc, positioning, topaths,chains, intersections, decorations.markings, decorations.text, shapes.geometric, matrix,patterns,mindmap}
%\usetikzlibrary{positioning, patterns,topaths,chains,matrix}

\usepackage{pgfplots}
\usepackage{pgfplotstable}
\pgfplotsset{compat=1.9}
\usepgfplotslibrary{groupplots}
\usepgfplotslibrary{external}
\makeatletter
\newcommand*{\overlaynumber}{\number\beamer@slideinframe}
\tikzset{
  beamer externalizing/.style={%
    execute at end picture={%
      \tikzifexternalizing{%
        \ifbeamer@anotherslide
        \pgfexternalstorecommand{\string\global\string\beamer@anotherslidetrue}%
        \fi
      }{}%
    }%
  },
  external/optimize=false
}
\let\orig@tikzsetnextfilename=\tikzsetnextfilename
\renewcommand\tikzsetnextfilename[1]{\orig@tikzsetnextfilename{#1-\overlaynumber}}
\makeatother

\tikzset{every picture/.style={beamer externalizing}}
\tikzexternalize
\tikzsetexternalprefix{fig_presentation/}
%\tikzset{external/optimize=false}
%\tikzset{external/force remake}


%link or play movies
\usepackage{multimedia}



%beamer related package

\usepackage{todonotes}
\presetkeys{todonotes}{inline}{}


%bibliography
\usepackage[style=authoryear-comp, language=british,eprint=false, url=false, doi=false, sortcites=true, sorting=none, isbn=false, firstinits=true,maxcitenames=6]{biblatex}
%minimal citations
\AtEveryCitekey{%
	\clearfield{title}
	\clearfield{pages}
	\clearfield{volume}
	\clearfield{number}
	\clearfield{month}}
\newcommand{\myfullcite}[1]{{\scriptsize\fullcite{#1}}}
\renewbibmacro{in:}{%
  \ifentrytype{article}{}{%
  \printtext{\bibstring{in}\intitlepunct}}}
%\bibliography{biblio}


\newcolumntype{P}[1]{>{\raggedright}p{#1}}

\institute[iLM]{Institut Lumière Matière, Lyon, France}
\title[Yoghurt under stress]{Acid-induced protein gels: from gelation to stress-induced failure}
\author[M. Leocmach]{Mathieu Leocmach}
\date{10 June 2015}
\titlegraphic{
	\begin{tabu}{X[c]X[c]X[c]X[c]}
		\includegraphics[height=3\baselineskip]{Chris}&
		\includegraphics[height=3\baselineskip]{Thomas}&
		\includegraphics[height=3\baselineskip]{Thibaut}&
		\includegraphics[height=3\baselineskip]{Seb}\\
		Christophe Perge & Thomas\linebreak Gibaud & Thibaut\linebreak Divoux & Sebastien Manneville\\
	\end{tabu}
	
	\vfill
	\includegraphics[height=2\baselineskip]{logo_ens-lyon}\quad
	\includegraphics[height=2\baselineskip]{logo_ums_grand}\quad
	\includegraphics[height=2\baselineskip]{CNRSfilaire-Q}\quad
	\includegraphics[height=2\baselineskip]{CRPP}\quad
	\includegraphics[height=2\baselineskip,clip=true, trim=6mm 14mm 6mm 0]{NEW-Logo-ERC-OUTLINE}
	}


\newlength{\mylength}

%\includeonly{creep_beamer}

\begin{document}
\tikzset{every mark/.append style={scale=0.8}}
\pgfplotsset{every axis/.append style={footnotesize}}

\pgfplotscreateplotcyclelist{earthy}{%
{red!40!black,mark=o},
{red!60!black,mark=triangle, every mark/.append style={rotate=180}},
{red!80!black,mark=square},
{red,mark=triangle},
{red!80!yellow, mark=diamond},
red!60!yellow,
red!40!yellow,
}

\AtBeginSection[]{
	\addtocounter{framenumber}{-1}
	\begin{frame}[plain]
		\tableofcontents[currentsection, hideothersubsections]
	\end{frame}
}

\begin{frame}[plain]
	\titlepage
\end{frame}

\setcounter{framenumber}{0}

\begin{frame}{Texture as important as taste}
\tikzsetnextfilename{texture}%
\begin{tikzpicture}[inner sep=0, callout pointer segments=3]
\node(ishichan) {\includegraphics[height=8\baselineskip]{ishichan}};
\node[above=of ishichan] {\includegraphics[height=8\baselineskip]{calamar}};
\node[cloud callout,anchor=west, draw, callout relative pointer={(-30:2\baselineskip)}] at ($(ishichan.north)+(-0.5\textwidth,0)$) {\includegraphics[height=5\baselineskip]{aldente}};
\node[cloud callout, aspect=0.75, anchor=east, draw, callout relative pointer={(-150:2\baselineskip)}] at ($(ishichan.north)+(0.5\textwidth-10pt,0)$) {\includegraphics[height=5\baselineskip]{chewie}};
\end{tikzpicture}
\end{frame}

\begin{frame}{Mouth feel}
\begin{columns}
\column{0.8\textwidth}
\begin{itemize}
\item A very complex, multidimensional sensation
\item Behaviour of the food at large deformation?
\item How does food break into pieces?
\end{itemize}
\column{0.2\textwidth}
\includegraphics[width=\textwidth]{violon}
\end{columns}

\bigskip
\begin{columns}[T]
\column{0.475\textwidth}
\begin{block}{Ductile fracture}
\includegraphics[height=5\baselineskip]{deflated_balloon}

Irreversible deformation
\begin{itemize}
\item rubber
\item lead
\item butter
\end{itemize}
\end{block}

\column{0.475\textwidth}
\begin{block}{Brittle fracture}
\includegraphics[height=6\baselineskip]{spaghetti_breaking}\hfill
\includegraphics[height=6\baselineskip]{Tea_bowl_fixed_in_the_Kintsugi_method}
\begin{itemize}
\item concrete
\item asphalt
\item glass
\end{itemize}

\end{block}
\end{columns}
\end{frame}



%
%
%
%\begin{frame}{Fundamental issues in biogels}
%\begin{itemize}
%\item Linear elasticity is not well understood \textit{\scriptsize Gardel et al. Science 2004}
%\item Strain and stress stiffening \textit{\scriptsize Storm et al. Nature 2005}
%\item Fractures \textit{\scriptsize Bonn et al. Science 1998, Baumberger et al. Nature Material 2006}
%\item Mechanical instabilities and morphogenesis \textit{\scriptsize Shyer et al. Science 2013}
%\end{itemize}
%\begin{tabu}{X[c]X[c]}
%\includegraphics[height=6\baselineskip]{Bonn_fracture} &
%\includegraphics[height=6\baselineskip]{Villi_sq}
%\end{tabu}
%\begin{block}{Our questions}
%\begin{itemize}
%\item What is the behaviour of a biogel under shear deep into the nonlinear regime?
%\item What is the role of the solvent in mechanical instabilities?
%\end{itemize}
%\end{block}
%\end{frame}

%\section{Our system: Yoghurt}


%\tikzset{external/force remake}
\begin{frame}{The simplest yoghurt}
\begin{block}{Acid-set protein gel}
\begin{itemize}
\item Water (\SI{30}{\celsius})
\item Sodium caseinate (milk protein) 4\%
\raisebox{0.6\normalbaselineskip}[0pt][0pt]{$\left.\rule{0pt}{1.1\normalbaselineskip}\right(\}$ stable solution}
\item Glucono-$\delta$-lactone (GDL) 1\% $\Rightarrow$ slow homogeneous acidification
\end{itemize}
\end{block}

\begin{columns}
\column{0.5\textwidth}
\tikzsetnextfilename{gelation_gdl1}
\begin{tikzpicture}
	\begin{groupplot}[%
		group style={
			group name=g, group size=1 by 2,
			xticklabels at=edge bottom,
			vertical sep=0,
			},
		%xmode=log,
		%xmin=1e2,xmax=3e5,
		xmin=0, xmax=20,
		scale only axis,
		width=\textwidth-4em,
		height=0.3\textwidth,
		extra tick style={grid=major},%
		ylabel absolute, every axis y label/.append style={anchor=base, yshift=-0.5em}
		]
	\nextgroupplot[
		ymin=0, ymax=7, ylabel={pH},
		extra y ticks={4.6}, extra y tick labels={},%
		]
	\addplot+[no marks,Accent2] table[x expr={\thisrowno{0}/3600}]{Y190_cas4_gdl1.pH};
	\draw[help lines] (axis cs:0,4.6) -- (axis cs:3.88,4.6) node[pos=1, above right, font=\scriptsize] {isoelectric point $pH\approx 4.6$}  -- (axis cs:3.88,0);
	

	\nextgroupplot[
		xlabel={time (\si{\hour})},
		ymin=0, ylabel={$\textcolor{Accent2}{G^\prime}, \textcolor{Main}{G^{\prime\prime}}$ (\si{\pascal})},
		extra x ticks={17}, extra x tick labels={},
		]
	\addplot+[no marks,Accent2] table[x expr={\thisrowno{0}/3600}]{cas4_GDL1_Y22.prise};
	\addplot+[no marks,Main] table[x expr={\thisrowno{0}/3600}, y index=2]{cas4_GDL1_Y22.prise};
	\begin{scope}[font=\scriptsize]
		%\node[above right] at (rel axis cs:0,0) {casein ``micelles''};
		%\draw[<-] (axis cs:3.88,654) -- +(1em,0) node[right] {gelation};
		\node[anchor=base,rotate=80] at (axis cs:2.5,400) {gelation};
	\end{scope}
	\end{groupplot}
	%\draw[<-] (g c1r2.south west) ++(0.5em,0) -- +(0,-1em) node[below,font=\footnotesize] {casein ``micelles''};
\end{tikzpicture}

\column{0.5\textwidth}
\includegraphics[width=\textwidth, clip=true, trim=0 0 0 10cm]{MEB_cas4_gdl1_22}

\begin{scriptsize}
\textit{Kaláb (1983),\linebreak Roefs \& van Vliet (1990),\linebreak Lucey \& Singh (1998)}
\end{scriptsize}
\end{columns}
\end{frame}
%\tikzset{external/force remake=false}

\begin{frame}{Linear rheology: Power law (visco)elastic solid}
\[\text{stress}\rightarrow \sigma = G \gamma\leftarrow\text{strain}\]
\[\textcolor{Accent1}{\text{storage}\rightarrow} G^\prime + \imath G^{\prime\prime} \textcolor{Accent2}{\leftarrow\text{loss}}\]
\tikzsetnextfilename{powerlaw}
\begin{tikzpicture}
\begin{loglogaxis}[
	height=0.8\textheight,
	width=\textwidth,
	xlabel={Frequency (\si{\hertz})}, ylabel={\textcolor{Accent1}{Storage} and \textcolor{Accent2}{loss} moduli (\si{\pascal})},
	domain={6e-2:70},
	]
	\addplot[Accent1, only marks, mark=*] table{freqsweep_Y265_cas4_GDL1.txt};
	\addplot[Accent2, only marks, mark=o] table[y=LossModulus]{freqsweep_Y265_cas4_GDL1.txt};
	\addplot[Main, no marks]{300*x^0.15}  node[midway, below right] {$G^\prime\sim G^{\prime\prime}\sim f^{0.15}$};
%	\addplot+[only marks, mark=*, forget plot] table{freqsweep_Y265_cas4_GDL1.txt} node[anchor=base east] {1\%};
%	\addplot+[only marks, mark=o, forget plot] table[y=LossModulus]{freqsweep_Y265_cas4_GDL1.txt} node[anchor=base east] {1\%};
%	\addplot{590*x^0.14};
%	\addplot+[only marks, mark=*, forget plot] table{freqsweep_Y277_cas4_GDL1.25.txt} node[anchor=base east] {1.25\%};
%	\addplot+[only marks, mark=o, forget plot] table[y=LossModulus]{freqsweep_Y277_cas4_GDL1.25.txt} node[anchor=base east] {1.25\%};
%	\addplot{350*x^0.12};
%	\addplot+[only marks, mark=*, forget plot] table{freqsweep_Y275_cas4_GDL1.5.txt} node[anchor=base east] {1.5\%};
%	\addplot+[only marks, mark=o, forget plot] table[y=LossModulus]{freqsweep_Y275_cas4_GDL1.5.txt} node[anchor=base east] {1.5\%};
%	\addplot{246*x^0.1};
%	\addplot+[only marks, mark=*, forget plot] table{freqsweep_Y268_cas4_GDL2.txt} node[anchor=base east] {2\%};
%	\addplot+[only marks, mark=o, forget plot] table[y=LossModulus]{freqsweep_Y268_cas4_GDL2.txt} node[anchor=base east] {2\%};
%	\addplot{163*x^0.067};
%	\addplot+[only marks, mark=*, forget plot] table{freqsweep_Y270_cas4_GDL3.txt} node[anchor=base east] {3\%};
%	\addplot+[only marks, mark=o, forget plot] table[y=LossModulus]{freqsweep_Y270_cas4_GDL3.txt} node[anchor=base east] {3\%};
%	\addplot{109*x^0.044};
%	\addplot+[only marks, mark=*, forget plot] table{freqsweep_Y271_cas4_GDL4.txt} node[anchor=base east] {4\%};
%	\addplot+[only marks, mark=o, forget plot] table[y=LossModulus]{freqsweep_Y271_cas4_GDL4.txt} node[anchor=base east] {4\%};
%	\addplot{76*x^0.035};
	
\end{loglogaxis}
\end{tikzpicture}
\end{frame}





\section*{Creep and yielding}

\begin{frame}{Creep and yielding}
Just accepted in PRL, \textit{ArXiv 1401.8234}

\bigskip
\structure{Merciless yoghurt breakers}

\bigskip
\begin{tabu}{X[c]X[c]c}
\includegraphics[height=0.3\textheight]{Chris}&
\includegraphics[height=0.3\textheight]{Thibaut}&
\includegraphics[height=0.3\textheight]{Seb}\\
Christophe Perge & Thibaut Divoux & Sebastien Manneville\\
PhD student & CNRS researcher & Professor\\
E.N.S. Lyon &  CRPP Bordeaux & E.N.S. Lyon\\
\end{tabu}


\bigskip
\begin{description}[Special thanks]
\item[Special thanks] to Alan Parker (Firmenich) for prompting this study and providing casein \& GDL
\end{description} 

\end{frame}

%\tikzset{external/force remake}
\begin{frame}{The yoghurt creep experiment}
\begin{tikzpicture}
\node[anchor=south west] (webcam){\includegraphics[width=0.2\textwidth]{webcam}};
\node[anchor=south] at (0.4\textwidth,0) (spoon) {\includegraphics[width=0.3\textwidth]{spoon}};
\node[anchor=south east] at (\textwidth,0) (ultrasound) {\includegraphics[width=0.4\textwidth]{ultrasound}};

\node[anchor=south west] at ($(ultrasound.north west)+(0,2em)$) (rheometer) {\includegraphics[height=0.4\textheight]{rheometer}};

\begin{scope}[draw]
	\node[below=0of spoon.south] (couette) {A Taylor-Couette cell where the yogurt is made \emph{in situ}};
	\draw[->] (couette) -- ($(spoon.south)+(0,2em)$);
	\node[above=0of ultrasound] {ultrasonic imaging};
	\node[above=0of webcam, text width=0.2\textwidth, align=center] {optical imaging};
	\draw[->] (spoon.north east) +(235:2em) -- (rheometer);
	\node[right=0of rheometer, text width=0.2\textwidth, align=center] {A rheometer imposes a constant stress $\sigma$ and records the strain $\gamma(t)$};
\end{scope}
\end{tikzpicture}
\end{frame}

%\tikzset{external/force remake}
\begin{frame}{The yoghurt creep experiment}
\movie[externalviewer]{\begin{tikzpicture}
\node[inner sep=0] (a) {\includegraphics[height=16\baselineskip]{creep_experiment.jpg}};
\node[below right=\baselineskip of a.north east, inner sep=0] (vmap) {\includegraphics[width=4\baselineskip]{jet_cmap}};

\begin{scope}[font=\footnotesize]
	\node[above=0 of vmap] {velocity map};
	\node[below=0 of vmap.south west] {-50};
	\node[below=0 of vmap.south] (v0) {0};
	\node[below=0 of vmap.south east] {50};
	\node[below=0 of v0, inner sep=0]{\si{\micro\metre\per\second}};
\end{scope}

\begin{scope}[shift={(a.north west)}, shift={(5.8\baselineskip,-1.1\baselineskip)}, ultra thick]
	\draw[Accent1, x radius=3.1\baselineskip, y radius=0.3\baselineskip] ellipse[] ++(3.1\baselineskip,0)  -- ++(0,-7.4\baselineskip)  ++(-3.1\baselineskip,0) ellipse[] ++(-3.1\baselineskip,0)  -- ++(0,7.4\baselineskip);
	\shadedraw[Accent2,ball color=Accent2, x radius=2\baselineskip, y radius=0.1\baselineskip] ellipse[] ++(2\baselineskip,0)  -- ++(0,-7.4\baselineskip)  ++(-2\baselineskip,0) ellipse[] ++(-2\baselineskip,0)  -- ++(0,7.4\baselineskip);
	\fill[Accent2, left color=black,right color=black, middle color=Accent2, ] (-0.3\baselineskip,0) rectangle +(0.6\baselineskip, \baselineskip);
	\begin{scope}[Main]
		\node[draw, minimum height=4.2\baselineskip, minimum width=1.5\baselineskip] at (2.5\baselineskip,-4\baselineskip) (window){};
		\node[draw, minimum height=7.2\baselineskip, minimum width=4\baselineskip] at (8.9\baselineskip,-4.2\baselineskip) (USV){};
		\draw(window.north east) -- (USV.north west) (window.south east) -- (USV.south west);
		\node[right, text width=4\baselineskip] at (USV.east) {ultrasound region of interest};
	\end{scope}

\draw[<->,thin] ++(-2\baselineskip,-\baselineskip) -- +(-1.1\baselineskip,0) node[midway,above] {\SI{2}{\milli\metre}};
\end{scope}
\draw[->] (window.north west) ++(-1em, 0.2em) -- +(1em,0) node[midway,above,inner sep=1pt] {$r$};
\draw[->] (window.north west) ++(-1em, 0.2em) -- +(0,-1em)  node[midway,left] {$z$};

\node[text width=4\baselineskip] at (-3\baselineskip,-2.5\baselineskip) {shear rate response};
\node at (5\baselineskip,-2.5\baselineskip) {strain response};
\end{tikzpicture}%
}{Yaourt141_enhanced.avi}
\end{frame}
%\tikzset{external/force remake=false}

\begin{frame}{Creep rheology: Three regimes}
\begin{tikzpicture}
\begin{loglogaxis}[
   name=g,
	width=\textwidth,
	height=0.8\textheight,
	xmin=2e-2, xmax=4e5, xlabel={time (\si{\second})},
	ymin=0.2,ymax=2,
	ytick={0.2, 0.4,0.8,1.6}, yticklabels={0.2, 0.4,0.8,1.6},
	ylabel={strain $\gamma$},
	extra tick style={grid=major},%
	extra y ticks={1}, extra y tick labels={1.0},
	cycle list name=earthy,
	axis on top,
	]
	\fill[Accent1!20]
		(axis cs:10,0.4) ellipse[rotate=-30, x radius=0.2\textwidth, y radius=0.12\textwidth]
		(axis cs:2e2,1.6) ellipse[x radius=0.35\textwidth, y radius=0.05\textwidth]
		(axis cs:3e3,0.85) ellipse[rotate=-7.5, x radius=0.3\textwidth, y radius=0.07\textwidth]
		;
	\node[above right] at (axis cs:1e1,0.2) {primary};
	\node[below left] at (axis cs:4e5,0.7) {secondary};
	\node[left] at (axis cs:5,1.6) {tertiary};
	\addplot table{Y27_200Pa_gamma_decimated.txt} node (s200){};
	\addplot table{Y38_300Pa_gamma_decimated.txt} node (s300){};
	\addplot table{Y25_400Pa_gamma_decimated.txt}  node (s400){};
	\addplot table{Y32_550Pa_gamma_decimated.txt}  node (s550){};
	\addplot table{Y39_1000Pa_gamma_decimated.txt} node (s1000){};
\end{loglogaxis}
\begin{scope}[anchor=base, every node/.style={yshift=0.2em}]
	\node[red!40!black] at (s200 |- g.outer north) {\SI{200}{\pascal}};
	\node[red!60!black] at (s300 |- g.outer north) {300};
	\node[red!80!black] at (s400 |- g.outer north) {400};
	\node[red] at (s550 |- g.outer north) {550};
	\node[red!80!yellow] at (s1000 |- g.outer north) {1000};
\end{scope}
\end{tikzpicture}

Failure at $\gamma\approx 1$ for a well defined time $\tau_f$
\end{frame}

%\tikzset{external/force remake}
\begin{frame}{Failure time}
\begin{columns}
\column{0.55\textwidth}
\begin{tikzpicture}
\begin{loglogaxis}[
	width=\textwidth,
	height=0.8\textwidth,
	xlabel={$\sigma$ (\si{\pascal})},
	ylabel={$\tau_f$ (\si{\second})},
	xmin=100, ymin=10, ymax=1e6,
	xtick={100, 200,..., 1000}, xticklabels={100, 200,,, 500,,,,, 1000},
	]
	\addplot[only marks, Accent1] table[y index=3]{creep_cas4_gdl1_MCR.txt};
	\addplot[no marks, black, domain={100:1000}] {4.2e17*x^(-5.45)} node [midway, above right] {$\tau_f\sim \sigma^{-5.5}$};
\end{loglogaxis}
\end{tikzpicture}

\begin{block}{Basquin law}
\begin{itemize}
\item fatigue (oscillatory stress)
\item heterogeneous solids
\item no yield stress
\end{itemize}
\end{block}

\column{0.45\textwidth}
\begin{tikzpicture}
\begin{groupplot}[
	group style={
		group name=g, group size=1 by 2,
		},
	width=\textwidth,
	height=0.7\textwidth,
	ymode=log,
	ylabel={$\tau_f$ (\si{\second})},
	ymin=10, ymax=1e6,
	]
	\nextgroupplot[
		xlabel={$\sigma$ (\si{\pascal})},
		xtick={200,600,1000},
		]
		\addplot[only marks, Accent1] table[y index=3]{creep_cas4_gdl1_MCR.txt};
		\addplot[no marks, black, domain={100:1000}] {6.5e5*exp(-x/84)} node [pos=0.25, above right] {$\tau_f\nsim e^{-\frac{\sigma}{\sigma_0}}$};
		
	\nextgroupplot[
		xlabel={$1/\sigma$ (\si{\per\pascal})},
		]
		\addplot[only marks, Accent1] table[x expr=1/\thisrowno{0}, y index=3]{creep_cas4_gdl1_MCR.txt};
		\addplot[no marks, black, domain={1e-3:7e-3}] {20.3*exp(1720*x)} node [pos=0.25, below right] {$\tau_f\nsim e^\frac{\sigma_0}{\sigma}$};
\end{groupplot}
\end{tikzpicture}

Activated processes unlikely (Eyring, Griffith, Taylor, \ldots)
\end{columns}
\end{frame}
\tikzset{external/force remake=false}

%\tikzset{external/force remake}
\begin{frame}{Primary creep}
\begin{columns}
\column{0.55\textwidth}
\begin{tikzpicture}
\begin{loglogaxis}[
	name=g,
	width=\textwidth,
	height=0.8\textwidth,
	xmin=2e-2, xmax=4e5, xlabel={time (\si{\second})},
	ylabel={Strain rate $\dot{\gamma}$ (\si{\per\second})},
	cycle list name=earthy,
	no marks,
	]
	\addplot table[y index=2]{Y27_200Pa_gdot_decimated.txt} node (s200){};
	\addplot table[y index=2]{Y38_300Pa_gdot_decimated.txt} node (s300){};
	\addplot table[y index=2]{Y25_400Pa_gdot_decimated.txt}  node (s400){};
	\addplot table[y index=2]{Y32_550Pa_gdot_decimated.txt}  node (s550){};
	\addplot table[y index=2]{Y39_1000Pa_gdot_decimated.txt} node (s1000){};
	\addplot[Accent2, ultra thick, domain={0.1:1e3}] {0.01*x^(-0.85)} node[pos=0.75, below left] {$\dot{\gamma}(t)\sim t^{-0.85}$};
\end{loglogaxis}
\begin{scope}[anchor=base east, every node/.style={ rotate=90}]
	\node[red!40!black] at (s200 |- g.outer north) {\SI{200}{\pascal}};
	\node[red!60!black] at (s300 |- g.outer north) {300};
	\node[red!80!black] at (s400 |- g.outer north) {400};
	\node[red] at (s550 |- g.outer north) {550};
	\node[red!80!yellow] at (s1000 |- g.outer north) {1000};
\end{scope}
\end{tikzpicture}

\begin{block}{Primary creep $\Rightarrow$ ``Andrade'' creep}
Power-law creep known in \alert{solids} for over a century

Classically explained by disinclination dynamics in crystals
\end{block}

\column{0.45\textwidth}
\includegraphics[width=\textwidth]{Andrade_1910}

\bigskip
\includegraphics[height=7\baselineskip]{Andrade_1910_fig1}
\begin{tikzpicture}
\node[inner sep=0] (a){\includegraphics[height=7\baselineskip]{Andrade_1910_fig2}};
\node[below right=0 and 1em of a.north west, font=\footnotesize,fill=white] {lead \& copper};
\node[above left=1em of a.south east, Accent2] {$\gamma-\gamma_0 \sim t^{1/3}$};
\end{tikzpicture}
\end{columns}
\end{frame}
\tikzset{external/force remake=false}

%\tikzset{external/force remake}
\begin{frame}{Primary creep}
\begin{block}{Local insight from \movie[externalviewer]{ultrasounds}{Yaourt110_primary.avi}}
\begin{itemize}
	\item homogeneous \movie[externalviewer]{strain field}{Yaourt110_profils.avi} (no wall slip, no shear band)
	\item if present, plastic events are below our resolution (a few \si{\micro\metre})
\end{itemize}
\end{block}

\begin{block}{Partial creep and recovery}
\begin{tikzpicture}
\begin{axis}[
	width=\textwidth,
	height=8\baselineskip,
	xlabel={time (\si{\second})},
	ylabel={strain $\gamma$},
	xmin=-1e3, xmax=2.8e4,
	axis background/.style={fill=white},
	]
	\addplot[black, no marks] table {Y261_oscil_creep_recovery_100Pa_decimated.txt};
	\begin{scope}[<-,Main, inner sep=0]
	\draw (axis cs:0,0) -- +(1em,0) node[anchor=west]{$\sigma\leftarrow \SI{100}{\pascal}$};
	\draw (axis cs:600,0.2727007) -- +(1em,0) node[anchor=west]{$\sigma\leftarrow 0$};
	\draw (axis cs:8600,0.0159) -- +(0,1em) node[anchor=base east]{$\gamma\leftarrow 0$};
	\end{scope}
\end{axis}
\end{tikzpicture}
\end{block}

$\Rightarrow$ Primary creep is ``kinetically'' reversible (no detectable structure damage, but viscous dissipation)
\end{frame}
\tikzset{external/force remake=false}




%\tikzset{external/force remake}
\begin{frame}{Dimensionless creep response}
\begin{columns}
\column{0.55\textwidth}
\begin{tikzpicture}
\begin{loglogaxis}[
	width=\textwidth,
	height=0.8\textwidth,
	xmin=1e-5, xmax=2, xlabel={$t/\tau_f$},
	ylabel={$\dot{\gamma}/\dot{\gamma}_\text{min}$},
	cycle list name=earthy,
	clip mode=individual,
	]
	\addplot table[x index=1, y index=3]{Y27_200Pa_gdot_decimated.txt} node (s200){};
	\addplot table[x index=1, y index=3]{Y38_300Pa_gdot_decimated.txt} node (s300){};
	\addplot table[x index=1, y index=3]{Y25_400Pa_gdot_decimated.txt}  node (s400){};
	\addplot table[x index=1, y index=3]{Y32_550Pa_gdot_decimated.txt}  node (s550){};
	\addplot table[x index=1, y index=3]{Y39_1000Pa_gdot_decimated.txt} node (s1000){};
	\addplot[yellow, ultra thick, domain={1e-4:0.1}] {0.378*x^(-0.85)} node[Accent2,pos=0.75, below left] {$\frac{\dot{\gamma}(t)}{\dot{\gamma}_\text{min}}\sim \left(\frac{t}{\tau_f}\right)^{-0.85}$};
\end{loglogaxis}
\end{tikzpicture}

Linear viscoelasticity accounts for the Andrade exponent

\column{0.45\textwidth}
\begin{tikzpicture}
\begin{loglogaxis}[
	height=0.8\textwidth,
	width=\textwidth,
	xlabel={Frequency (\si{\hertz})}, ylabel={\textcolor{Accent1}{$G^\prime$}, \textcolor{Accent2}{$G^{\prime\prime}$}(\si{\pascal})},
	domain={6e-2:70},
	ymin=10,
	]
	\addplot[Accent1, only marks, mark=*] table{freqsweep_Y265_cas4_GDL1.txt};
	\addplot[Accent2, only marks, mark=o] table[y=LossModulus]{freqsweep_Y265_cas4_GDL1.txt};
	\addplot[Main, no marks]{100*x^0.15}  node[midway, below=1em] {$G^\prime\sim G^{\prime\prime}\sim f^{0.15}$};
	
\end{loglogaxis}
\end{tikzpicture}
\textcolor{Main}{
\begin{align*}
\Rightarrow & J(t) \equiv& \frac{\gamma(t)}{\sigma}\sim& t^{0.15}\\
\Rightarrow & \dot{\gamma}(t)\sim& t^{0.15-1} =& t^{-0.85}
\end{align*}
}
\end{columns}
\end{frame}
\tikzset{external/force remake=false}

%\tikzset{external/force remake}
\begin{frame}{Tertiary creep}
\begin{columns}
\column{0.55\textwidth}
Time to failure

\begin{tikzpicture}
\begin{loglogaxis}[
	width=\textwidth,
	height=0.8\textwidth,
	xmin=1e-5, xmax=2, xlabel={$(\tau_f-t)/\tau_f$},
	x dir=reverse,
	ylabel={$\dot{\gamma}/\dot{\gamma}_\text{min}$},
	cycle list name=earthy,
	clip mode=individual,
	]
	\addplot table[x expr=1-\thisrowno{1}, y index=3]{Y27_200Pa_gdot_decimated.txt} node (s200){};
	\addplot table[x expr=1-\thisrowno{1}, y index=3]{Y38_300Pa_gdot_decimated.txt} node (s300){};
	\addplot table[x expr=1-\thisrowno{1}, y index=3]{Y25_400Pa_gdot_decimated.txt}  node (s400){};
	\addplot table[x expr=1-\thisrowno{1}, y index=3]{Y32_550Pa_gdot_decimated.txt}  node (s550){};
	\addplot table[x expr=1-\thisrowno{1}, y index=3]{Y39_1000Pa_gdot_decimated.txt} node (s1000){};
	\addplot[yellow, ultra thick, domain={1e-4:0.1}] {0.187/x} node[Accent2, pos=0.6, below right, inner sep=0] {$\frac{\dot{\gamma}(t)}{\dot{\gamma}_\text{min}}\sim \left(\frac{\tau_f}{\tau_f-t}\right)$};
\end{loglogaxis}
\end{tikzpicture}

Close to failure: 
\begin{itemize}
\item finite time singularity
\item logarithmic divergence
\end{itemize}
$\Rightarrow$ dominated by fracture growth

\column{0.45\textwidth}
\begin{block}{Fracture growth}
\includegraphics[width=0.33\textwidth]{Y110_2013-03-01_03-15-00.jpg}\hfill
\includegraphics[width=0.33\textwidth]{Y110_2013-03-01_03-20-00.jpg}\hfill
\includegraphics[width=0.33\textwidth]{Y110_2013-03-01_03-21-15.jpg}

\begin{tikzpicture}
\begin{axis}[
	width=\textwidth,
	height=0.8\textwidth,
	xmin=1e-3, xmax=1, xlabel={$(\tau_f-t)/\tau_f$},
	xmode=log, x dir=reverse,
	ylabel={$\ell/H$},
	ymin=0, ymax=0.5,
	clip mode=individual,
	axis background/.style={fill=white},
	]
	\addplot[only marks, Accent1, error bars/.cd, y dir=both, y explicit] table[y error index=2]{Y110_fracturelength.txt};
	\addplot[Accent2, ultra thick, no marks, domain={1e-3:1}] {- 0.06 * ln(x)} node [pos=0.25,left=1em] {$\frac{\ell}{H}\sim \ln\frac{\tau_f-t}{\tau_f}$};
\end{axis}
\end{tikzpicture}
\end{block}
\end{columns}
\end{frame}
\tikzset{external/force remake=false}

%\tikzset{external/force remake}
\begin{frame}{Secondary creep: a crossover}
\begin{columns}
\column{0.55\textwidth}
Creep responses in linear scales

\begin{tikzpicture}
\begin{axis}[
	width=\textwidth,
	height=0.8\textwidth,
	xmin=0, xmax=1, xlabel={$t/\tau_f$},
	ylabel={$\dot{\gamma}/\dot{\gamma}_\text{min}$},
	ymin=0.5, ymax=4,
	cycle list name=earthy,
	restrict y to domain=0.5:10,
	clip mode=individual,
	]
	\addplot table[x index=1, y index=3]{Y27_200Pa_gdot_decimated.txt} node (s200){};
	\addplot table[x index=1, y index=3]{Y38_300Pa_gdot_decimated.txt} node (s300){};
	\addplot table[x index=1, y index=3]{Y25_400Pa_gdot_decimated.txt}  node (s400){};
	\addplot table[x index=1, y index=3]{Y32_550Pa_gdot_decimated.txt}  node (s550){};
	\addplot table[x index=1, y index=3]{Y39_1000Pa_gdot_decimated.txt} node (s1000){};
	\addplot[yellow, ultra thick, domain={0.01:0.99}] {0.378*x^(-0.85) + 0.187/(1-x)};
	\draw[<-] (axis cs:0.56,1) -- +(0,1em) node[above] {$\tau_\text{min}$};
\end{axis}
\end{tikzpicture}
\begin{block}{Master curve}
\[\frac{\dot{\gamma}(t)}{\dot{\gamma}_\text{min}} = \underbrace{\lambda \left(\frac{t}{\tau_f}\right)^{-0.85}}_\text{Andrade} + \underbrace{\frac{\mu}{1 - t/\tau_f}}_\text{fractures}\]
\end{block}


\column{0.45\textwidth}
\structure{Monkman-Grant relation}
\begin{tikzpicture}
\begin{loglogaxis}[
	width=\textwidth,
	height=0.8\textwidth,
	xlabel={$\tau_f$ (\si{\second})},
	ylabel={$\tau_\text{min}$ (\si{\second})},
	xmin=10, xmax=1e6, ymin=10, ymax=1e6,
	]
	\addplot[only marks, Accent1] table[x index=3, y index=4]{creep_cas4_gdl1_MCR.txt};
	\addplot[no marks, black, domain={10:1e6}] {0.56*x)};
\end{loglogaxis}
\end{tikzpicture}
\[\tau_\text{min} \approx 0.6 \tau_f\]
``Early time'' response allows to predict failure time
\end{columns}
\end{frame}
\tikzset{external/force remake=false}

\begin{frame}{What about theories and models?}
\begin{columns}
\column{0.35\textwidth}
\begin{block}{Discrete dislocation dynamics (dense)}
\textit{\footnotesize Miguel et al., PRL 2002}

\hspace{1em}\includegraphics[width=\textwidth-2em]{Miguel_2002_img}

\medskip
\hspace{1em}\includegraphics[width=\textwidth-2em]{Miguel_2002_graph}

predicts Andrade creep \alert{but} no fracture
\end{block}
\column{0.65\textwidth}
\begin{columns}
\column{0.5\textwidth+1em}
\structure{Fibre bundle model~\#1}

\textit{\footnotesize Jagla et al., PRE 2011}
\begin{itemize}
\item elastic fibres
\item local yield strain
\item local coupling
\end{itemize}

\hspace{1em}\includegraphics[width=\textwidth-1em]{Jagla_2011_gamma}
\[\tau_f\sim\left(\sigma-\alert{\sigma_c}\right)^{-1.25}\]

\column{0.5\textwidth-1em}
\includegraphics[width=\textwidth]{Jagla_2011_3regimes}
\end{columns}
predicts Andrade creep and failure \alert{but} critical stress
\end{columns}
\end{frame}


\begin{frame}{What about theories and models?}
\structure{Fibre bundle model~\#2}

elastic fibres + local yield strain + damage accumulation
\begin{block}{\textit{\footnotesize Kun et al. J. Stat Mech 2006, PRL 2008}}
\begin{columns}
\column{0.7\textwidth}
\begin{tikzpicture}[inner sep=0]
\node(a){\includegraphics[height=5\baselineskip]{Kun_2006_basquin.png}};
\node[fill=white,yshift=-0.5em] at (a) {$\tau_f \sim \sigma^{-\beta}$};
\end{tikzpicture}\,
\includegraphics[height=5\baselineskip]{Kun_2008_Basquin}\,
\includegraphics[height=5\baselineskip]{asphalt}
\column{0.3\textwidth}
$\beta$ exponent of damage accumulation
\[\dot{\gamma}\sim \left(\tau_f-t\right)^{-1/(1+\beta)}\]
\end{columns}
\end{block}

\begin{columns}
\column{0.65\textwidth}
\begin{block}{\textit{\footnotesize Halász et al., PRE 2012}}
\tikzsetnextfilename{Halasz}
\begin{tikzpicture}[inner sep=0, ultra thick, font=\footnotesize]
\node (g) {\includegraphics[height=7\baselineskip]{Halasz_2012_Basquin}};
\node[draw=green!50!black, below right=0 and \baselineskip of g.north east] (c) {\includegraphics[height=4\baselineskip]{Halasz_2012_crack}};
\node[draw=red, above right=0 and 6\baselineskip of g.south east] (d) {\includegraphics[height=4\baselineskip]{Halasz_2012_disordered}};
\draw[->,green!50!black] (c.west) -- ++(-2\baselineskip,0) node[above] {$\beta=5$} node[below right=0 of c.north east, text width=5\baselineskip]{single crack propagation};
\draw[->, red] (d.south west) -- +(-5\baselineskip,0) --($(g.south east) +(0, \baselineskip)$) node[below=1em] {$\beta=1$} node[above left=0 of d.south west] {diffusive damage};
\end{tikzpicture}
\end{block}
\column{0.35\textwidth}
Predicts Basquin law \& fractures \alert{but} no Andrade creep \& too slow divergence
\end{columns}
\end{frame}

\begin{frame}{Creep and yielding summary}
\begin{itemize}
\item failure involves a single time scale $\tau_f \sim \sigma^{-5.5}$ Basquin law
\item reversible, homogeneous creep $\rightarrow$ irreversible fracture growth
\item a single expression captures all the global rheological response
\end{itemize}
\begin{tikzpicture}
	\begin{groupplot}[%
		group style={
			group name=g, group size=3 by 1,
			horizontal sep=3em,
			},
		width=0.35\textwidth,
		cycle list name=earthy,
		clip mode=individual,
		]
	\nextgroupplot[
		xmin=1e-5, xmax=2, xlabel={$t/\tau_f$},
		ylabel={$\dot{\gamma}/\dot{\gamma}_\text{min}$},
		xmode=log,ymode=log,
		xtick={1e-4, 1e-2, 1},
		]
		\addplot table[x index=1, y index=3]{Y27_200Pa_gdot_decimated.txt};
		\addplot table[x index=1, y index=3]{Y38_300Pa_gdot_decimated.txt};
		\addplot table[x index=1, y index=3]{Y25_400Pa_gdot_decimated.txt};
		\addplot table[x index=1, y index=3]{Y32_550Pa_gdot_decimated.txt};
		\addplot table[x index=1, y index=3]{Y39_1000Pa_gdot_decimated.txt};
		\addplot[yellow, ultra thick, samples at={1e-5,1e-4, 1e-3,1e-2,0.1,0.2,0.3,0.4,0.5,0.6,0.7,0.8,0.9, 0.99, 0.999, 0.9999, 0.99999}] {0.378*x^(-0.85) + 0.187/(1-x)};
		
	\nextgroupplot[
		xmin=0, xmax=1, xlabel={$t/\tau_f$},
		ymin=0.5, ymax=4,
		restrict y to domain=0.5:10,
		]
		\addplot table[x index=1, y index=3]{Y27_200Pa_gdot_decimated.txt};
		\addplot table[x index=1, y index=3]{Y38_300Pa_gdot_decimated.txt};
		\addplot table[x index=1, y index=3]{Y25_400Pa_gdot_decimated.txt};
		\addplot table[x index=1, y index=3]{Y32_550Pa_gdot_decimated.txt};
		\addplot table[x index=1, y index=3]{Y39_1000Pa_gdot_decimated.txt};
		\addplot[yellow, ultra thick, domain={0.01:0.99}] {0.378*x^(-0.85) + 0.187/(1-x)};
		
	\nextgroupplot[
		xmin=1e-5, xmax=2, xlabel={$(\tau_f-t)/\tau_f$},
		x dir=reverse,
		xmode=log,ymode=log,
		xtick={1e-4, 1e-2, 1},
	]
		\addplot table[x expr=1-\thisrowno{1}, y index=3]{Y27_200Pa_gdot_decimated.txt};
		\addplot table[x expr=1-\thisrowno{1}, y index=3]{Y38_300Pa_gdot_decimated.txt};
		\addplot table[x expr=1-\thisrowno{1}, y index=3]{Y25_400Pa_gdot_decimated.txt};
		\addplot table[x expr=1-\thisrowno{1}, y index=3]{Y32_550Pa_gdot_decimated.txt};
		\addplot table[x expr=1-\thisrowno{1}, y index=3]{Y39_1000Pa_gdot_decimated.txt};
		\addplot[yellow, ultra thick, samples at={1e-5,1e-4, 1e-3,1e-2,0.1,0.2,0.3,0.4,0.5,0.6,0.7,0.8,0.9, 0.99, 0.999, 0.9999, 0.99999}] {0.378*(1-x)^(-0.85) + 0.187/x};
	\end{groupplot}
\end{tikzpicture}
\begin{itemize}
\item a model soft solid well captured by fibre-bundle models
\end{itemize}
\end{frame}



\begin{frame}{Can we generalise to biogels? \textup{\normalsize(physical gels
$E_{bond}\gg k_B T$)}}
\begin{tabu}{cX[c]c}
\includegraphics[height=0.3\textheight]{cell_mech} &
\includegraphics[height=0.3\textheight]{electrophoresis} &
\includegraphics[height=0.3\textheight]{cosmetics} \\
Cell mechanics & Electrophoresis & Cosmetics\\
\end{tabu}
\begin{tabu}{X[c]X[c]}
\includegraphics[height=0.3\textheight]{bacterial_culture} &
\includegraphics[height=0.3\textheight]{food} \\
Bacterial culture & Food\\
\end{tabu}

\begin{footnotesize}
\begin{block}{Sources}
\begin{tabu}{lXl}
Wikimedia Commons & www.madaboutscience.com & www.keautystore.com\\
\end{tabu}
\end{block}
\end{footnotesize}
\end{frame}

\section*{Beyond critical gels}

\begin{frame}{Over-acidified protein gels: beyond isoelectric pH}
	\tikzsetnextfilename{prise_cas4}
	\begin{tikzpicture}
\pgfplotsset{every axis/.append style={xlabel absolute, every axis x label/.append style={anchor=base, yshift=-1em}, ylabel absolute, every axis y label/.append style={anchor=base, yshift=0em}}}

\begin{groupplot}[
	group style={
			group name=g, group size=1 by 2,
			vertical sep=0.5em,
			x descriptions at=edge bottom,
		},
	scale only axis,
	width=0.8\textwidth,
	height=4\baselineskip,
	xlabel={time (h)}, 
	xmin=0, xmax=4.5,
	extra tick style={grid=major},%
	]

 %%%% pH %%%%
\nextgroupplot[
	ymin=0, ymax=7, ylabel={pH},
	extra y ticks={4.6}, extra y tick labels={},%
	cycle list name=earthy,
	no marks,
	xmin=0, xmax=20,xtick={0,2,...,20},
	]
	\begin{scope}[every node/.style={anchor=base west, inner xsep=0, font=\small}]
	\addplot table[x expr={\thisrowno{0}/3600.+0.05}]{Y190_cas4_gdl1.pH}  (axis cs:20,4) node[below left] {1\%};
	\pgfplotsset{cycle list shift=4};
	\addplot table[x expr={\thisrowno{0}/3600.+0.05}]{Y189_28800s.pH}  node {4\%};;
	\end{scope}
	\node[base left=0] at (axis cs:17,4.6) {isoelectric};



 %%%% prise unscaled %%%%
\nextgroupplot[
	height=6\baselineskip,
	ylabel={$G^\prime$ (\si{\pascal})},
	cycle list name=earthy,
	no marks,
	xmin=0, xmax=20,ymin=0,ymax=950,xtick={0,2,...,20},
	]
	\begin{scope}[every node/.style={anchor=base west, inner xsep=0, font=\small}]
	\addplot table[x expr={\thisrowno{0}/3600}]{cas4_GDL1_Y265.prise} node {1\%};
	\addplot table[x expr={\thisrowno{0}/3600}]{cas4_GDL1.25_Y277.prise} node {1.25\%};
	\addplot table[x expr={\thisrowno{0}/3600}]{cas4_GDL1.5_Y275.prise} node {1.5\%};
	\addplot table[x expr={\thisrowno{0}/3600}]{cas4_GDL2_Y268.prise} node[yshift=0.1em] {2\%};
	\addplot table[x expr={\thisrowno{0}/3600}]{cas4_GDL3_Y270.prise} node[yshift=-0.1em] {3\%};
	\addplot table[x expr={\thisrowno{0}/3600}]{cas4_GDL4_Y271.prise} node[yshift=-0.5em] {4\%};
	\end{scope}

\end{groupplot}

\end{tikzpicture}
	\begin{itemize}
	\item[$\Rightarrow$] still solid
	\item From critical to super-critical gels?
	\end{itemize}
\end{frame}

\begin{frame}{Linear rheology, from power-law to soft glassy}
	\tikzsetnextfilename{sweep_cas4}
	\begin{figure}
\begin{tikzpicture}
\pgfplotstableread{freqsweep_Y265_cas4_GDL1.txt}\GDLa
\pgfplotstableread{freqsweep_Y277_cas4_GDL1.25.txt}\GDLb
\pgfplotstableread{freqsweep_Y275_cas4_GDL1.5.txt}\GDLc
\pgfplotstableread{freqsweep_Y268_cas4_GDL2.txt}\GDLd
\pgfplotstableread{freqsweep_Y270_cas4_GDL3.txt}\GDLe
\pgfplotstableread{freqsweep_Y271_cas4_GDL4.txt}\GDLf

\begin{groupplot}[
	group style={
			group name=g, group size=2 by 2,
			vertical sep=2.5em,
			horizontal sep=4em,
			%xticklabels at=edge bottom,
		},
	scale only axis,
	width=0.6\columnwidth-4em,
	height=4\baselineskip,
	domain={6e-2:70},
	cycle list name=earthy,
	xmode=log, ymode=log,
	xmin=5e-2, xmax=1e3,
	xlabel={$f$ (\si{\hertz})},
	xlabel absolute, every axis x label/.append style={anchor=base, yshift=-0.5em}
	]

%%%%Elastic%%%%
\nextgroupplot[ylabel={$G^\prime$ (\si{\pascal})}, ymin=18, ymax=3e3]
\begin{scope}[
	every axis plot post/.append style={only marks, mark=*, mark options={scale=0.1}},
	every node/.style={anchor=west, ,font=\scriptsize, pos=0}
]
	\addplot table{\GDLa} node[anchor=base west]{1\%};
	\addplot table{\GDLb} node[anchor=base west]{1.25\%};
	\addplot table{\GDLc} node[yshift=0.1em]{1.5\%};
	\addplot table{\GDLd} node[yshift=0.1em]{2\%};
	\addplot table{\GDLe} node{3\%};
	\addplot table{\GDLf} node[yshift=-0.2em]{4\%};
\end{scope}
\pgfplotsset{cycle list shift=-6}
\addplot {590.277*x^0.139256};
\addplot {349.762*x^0.121815};
\addplot {246.573*x^0.102707};
\addplot {162.851*x^0.0676911};
\addplot {109.351*x^0.0437222};
\addplot {76.7364*x^0.035779};
\draw[->, thick] (axis cs: 7,850) -- (axis cs:12,60) node[pos=0, anchor=south east]{\textsc{gdl}$\searrow$};

%%%Exponent%%%
\nextgroupplot[
	width=0.4\columnwidth-4em-5pt,
	xlabel={GDL (\%)},
	xmode=linear,
	ymode=linear,
	xmin=0, xmax=4.5,
	ylabel={exponent}, ymin=0,ymax=0.15,
	ytick={0,0.05,0.1}, yticklabels={0,0.05,0.10}
]
\addplot[mark=*] file{sweep_exponents.txt};


%%%Loss%%%
\nextgroupplot[ylabel={$G^{\prime\prime}$ (\si{\pascal})}, ymin=3, ymax=5e2]
\begin{scope}[
	every axis plot post/.append style={only marks, mark=*, mark options={scale=0.1}},
	every node/.style={anchor=west,font=\scriptsize, pos=0}
]
	\addplot table[y index=2]{\GDLa} node{1\%};
	\addplot table[y index=2]{\GDLb} node{1.25\%};
	\addplot table[y index=2]{\GDLc} node[yshift=-0.1em]{1.5\%};
	\addplot table[y index=2]{\GDLd} node{2\%};
	\addplot table[y index=2]{\GDLe} node{3\%};
	\addplot table[y index=2]{\GDLf} node[yshift=-0.2em]{4\%};
\end{scope}
\pgfplotsset{cycle list shift=-6}
\addplot {140.022*x^0.139256};
\addplot {72.4462*x^0.121815};
\addplot {43.0813*x^0.102707};
\addplot {18.9117*x^0.0676911};
\addplot {8.39838*x^0.0437222};
\addplot {4.99978*x^0.035779};
\draw[->, thick] (axis cs: 7,200) -- (axis cs:12,4) node[pos=0, above left=-0.5em and 0]{\textsc{gdl}$\searrow$};

\nextgroupplot[
	width=0.4\columnwidth-4em-5pt, 
	ylabel={$G^{\prime\prime}/G^{\prime\prime}_{\SI{1}{\hertz}}$}, 
	ymode=linear, ymin=0.5,
	xmin=0.05,xmax=100, xtick={0.1,1,10},
	%xmin=0, xmax=60,
]
\begin{scope}[
	%every axis plot post/.append style={mark=*, mark options={scale=0.1}},
	every node/.style={anchor=west,font=\scriptsize, pos=0}
]
	\addplot table[y expr={\thisrowno{2}/140.022}]{\GDLa} node[pos=0.25,left]{1\%};
	\addplot table[y expr={\thisrowno{2}/72.4462}]{\GDLb};
	\addplot table[y expr={\thisrowno{2}/43.0813}]{\GDLc};
	\addplot table[y expr={\thisrowno{2}/18.9117}]{\GDLd};
	\addplot table[y expr={\thisrowno{2}/8.39838}]{\GDLe};
	\addplot table[y expr={\thisrowno{2}/4.99978}]{\GDLf}node[pos=0.25,right]{4\%};
\end{scope}
\draw[->, thick] (axis cs: 20,1.8) -- (axis cs:60,1.2) node[pos=0, above left=-0.5em and 0]{\textsc{gdl}$\searrow$};
\end{groupplot}

\end{tikzpicture}
\end{figure}
	\begin{itemize}
	\item Near isoelectric, power law rheology $G^\prime, G^{\prime\prime}$ typical of critical gels 
	\item Over-acidification lowers exponent and creates a minimum in $G^{\prime\prime}$
	\item Transition to soft-glassy behaviour \\\textit{\footnotesize Sollich, PRE (1998); Mason, Curr. Opin. Colloid Interface Sci. (1999)}
	\end{itemize}
\end{frame}

\begin{frame}{Questions}
\begin{description}
\item[What] is the change in microstructure?
\item[How] does it affects the failure mechanism?
\end{description}
\end{frame}

\begin{frame}{Microscructure: fluorescent confocal microscopy}
	\tikzsetnextfilename{structure_confocal}
	\begin{tikzpicture}[every axis/.style={xlabel absolute, every axis x label/.append style={anchor=base, yshift=-1em}}]
\begin{groupplot}[
	group style={
			group name=g, group size=1 by 3,
			vertical sep=0.5em,
			xticklabels at=edge bottom,
		},
	scale only axis,
	height=3.5\baselineskip,
	width=0.5\textwidth-3em,
	xmin=0, xmax=8, ymin=0,
	cycle list name=earthy,
	no marks,
	]
\nextgroupplot[ylabel={$\xi$ (\si{\micro\metre})}, height=3\baselineskip, ymax=10, ytick={0,3,6,9}]
\addplot table[skip coords between index={0}{64}, x expr={\thisrowno{0}/3600.}, y expr={6.28*\thisrowno{1}}]{ech17_pore_size.txt};
\pgfplotsset{cycle list shift=4};
\addplot table[skip coords between index={0}{2}, x expr={\thisrowno{0}/3600}, y expr={6.28*\thisrowno{1}}]{ech26_pore_size.txt};

\nextgroupplot[ylabel={$\chi$ (a.u.)}]
\begin{scope}%[every node/.style={anchor=base west}]
\addplot table[skip coords between index={0}{64}, x expr={\thisrowno{0}/3600}, y expr={\thisrowno{2}/1e3}]{ech17_pore_size.txt} node[anchor=north east, pos=0.65] (l1) {\textsc{gdl} 1\%};
\pgfplotsset{cycle list shift=4};
\addplot table[x expr={\thisrowno{0}/3600}, y expr={\thisrowno{2}/2e3}]{ech26_pore_size.txt} node[anchor=base west, pos=0.3] {\textsc{gdl} 4\%};
\end{scope}

\nextgroupplot[xlabel={time (\si{\hour})}, ylabel={$d$}]
\addplot table[skip coords between index={0}{64}, x expr={\thisrowno{0}/3600}, y index=3]{ech17_pore_size.txt};
\pgfplotsset{cycle list shift=4};
\addplot table[x expr={\thisrowno{0}/3600}, y index=3]{ech26_pore_size.txt};
\node[anchor=south east, font=\scriptsize] at (rel axis cs:1,0) {$\displaystyle I(q) = \frac{2\chi\Gamma(d)}{\left(1+\left((d+1)\xi q\right)^2\right)^{d/2}}$};
\end{groupplot}


\matrix[matrix of nodes, inner sep=0, row sep=0.2em, column sep=0.2em, matrix anchor=north west] 
at ($(g c1r1.right of north east)+(-\textwidth,0)$)
(m) {
\includegraphics[width=2.5\baselineskip]{ech26_t00.jpg}&
\includegraphics[width=2.5\baselineskip]{ech26_t05.jpg}&
\includegraphics[width=2.5\baselineskip]{ech26_t30.jpg}&
\includegraphics[width=2.5\baselineskip]{ech26_t70.jpg}\\
\SI{5}{\minute} & \SI{27}{\minute} & \SI{49}{\minute} & \SI{8}{\hour}\\
};
\draw[ultra thick, white] (m-1-1.south east) ++(-0.25\baselineskip,1em) -- +(-2\baselineskip,0) node[midway, above] {\SI{20}{\micro\metre}};

\newdimen\mydima
\newdimen\mydimb
\pgfextracty{\mydima}{\pgfpointanchor{g c1r1}{north}}
\pgfextracty{\mydimb}{\pgfpointanchor{g c1r3}{south}}
\begin{loglogaxis}[
	name=Sq,
	scale only axis,
	width=0.3\textwidth,
	height=4.5\baselineskip,
	anchor=left of south west,
	at={($(g c1r3.right of south east)+(-\textwidth,0)$)},
	cycle list name=earthy,
	xlabel={$q$ (\si{\per\micro\metre})},
	ylabel={$I(q)$ (a.u.)},
	domain=0.3:10,
	xmin=0.08, xmax=4e3,
	ymin=1e2, ymax=2e4,
	clip mode=individual,
	]
\begin{scope}[
	every axis plot post/.append style={only marks, mark options={scale=0.3}},
]
\addplot table{ech26_t00.Sq};
\addplot table{ech26_t02_4.Sq};
\addplot table{ech26_t02_5.Sq};
\addplot table{ech26_t02_6.Sq};
\addplot table{ech26_t02_7.Sq};
\addplot table{ech26_t03.Sq};
%\addplot table{ech26_t70.Sq};

\pgfplotsset{cycle list shift=-7}
\addplot table[x expr={100*\thisrowno{0}}]{ech26_t04.Sq};
\addplot table[x expr={100*\thisrowno{0}}]{ech26_t14.Sq};
\addplot table[x expr={100*\thisrowno{0}}]{ech26_t30.Sq};
\addplot table[x expr={100*\thisrowno{0}}]{ech26_t49.Sq};
\pgfplotsset{cycle list shift=4}
\addplot table[x expr={100*\thisrowno{0}}]{ech26_t70.Sq};
\end{scope}
\pgfplotsset{cycle list shift=-3}
\addplot+[domain=0.3:5] {433/(1+(1.69 * x)^2)^(0.24)} coordinate[pos=0] (t0);
\addplot {659/(1+(0.82 * x)^2)^(0.33)};
\addplot {1094/(1+(0.32 * x)^2)^(0.72)};
\addplot {2610/(1+(0.37 * x)^2)^(0.91)};
\addplot {4217/(1+(0.45 * x)^2)^(0.95)};
\addplot {7686/(1+(0.59 * x)^2)^(0.99)} coordinate[pos=0] (t1);
\pgfplotsset{cycle list shift=-2}
\addplot+[domain=30:1000] {9658/(1+(0.61e-2 * x)^2)^(0.98)} coordinate[pos=0] (t2);
\addplot+[domain=30:1000] {9652/(1+(0.60e-2 * x)^2)^(1.00)};
\addplot+[domain=30:1000] {6749/(1+(0.49e-2 * x)^2)^(1.07)};
\addplot+[domain=30:1000] {3797/(1+(0.43e-2 * x)^2)^(1.10)};
\pgfplotsset{cycle list shift=1}
\addplot+[domain=30:1000] {3111/(1+(0.44e-2 * x)^2)^(1.04)} coordinate[pos=0] (t3);

\draw[->, line width=0.2em] (t0) -- (t1) node[midway,rotate=90, above] {time};
\draw[->, line width=0.2em] (t2) -- (t3) node[midway,rotate=90, above] {time};

\end{loglogaxis}
\node[anchor=south west] at (Sq.outer south west) (gdl) {\textcolor{red!60!yellow}{\textsc{gdl} 4\%}};

\end{tikzpicture}

	\vspace{-1em}
	\begin{itemize}
	\item Maximum $G^\prime$ corresponds to maximum contrast between phases
	\item Pore size $\xi$ decrease during over-acidification
	\item<2> Casein aggregates swell $\sim$ over compressed gel $\Rightarrow$ soft glassy
	\end{itemize}
\end{frame}

\begin{frame}{Three creep regimes + softness\hfill\textsc{gdl} 4\%}
\vfill
	\tikzsetnextfilename{creep_cas4_gdl4}
	\begin{tikzpicture}
	\begin{groupplot}[%
		group style={
			group name=g, group size=3 by 1,
			horizontal sep=3em,
			},
		width=0.35\textwidth,
		cycle list name=earthy,
		clip mode=individual,
		]
	\nextgroupplot[
		xmin=1e-5, xmax=2, xlabel={$t/\tau_f$},
		ylabel={$\dot{\gamma}/\dot{\gamma}_\text{min}$},
		xmode=log,ymode=log,
		xtick={1e-4, 1e-2, 1},
		]
		\addplot table {Y242_20Pa_gdot_decimated.txt};
        \addplot table {Y239_40Pa_gdot_decimated.txt};
        \addplot table {Y238_50Pa_gdot_decimated.txt};
        \addplot table {Y246_60Pa_gdot_decimated.txt};
        \addplot table {Y236_100Pa_gdot_decimated.txt};
        \addplot+[restrict x to domain=-2.75:1] table {Y249_120Pa_gdot_decimated.txt};
	    \addplot[yellow, ultra thick, samples at={1e-5,1e-4, 1e-3,1e-2,0.1,0.2,0.3,0.4,0.5,0.6,0.7,0.8,0.9, 0.99, 0.999, 0.9999, 0.99999}] {0.378*x^(-0.96) + 0.187/(1-x)};
		\addplot[domain=1e-3:1e-1] {x^(-0.96)} node[pos=0, right] {$\alpha-1$};
		
	\nextgroupplot[
		xmin=0, xmax=1, xlabel={$t/\tau_f$},
		ymin=0.5,ymax=10, 
	xtick={0, 0.2, ..., 1}, ytick={2,4,6,8}
		]
		\begin{scope}[/pgfplots/restrict y to domain=0:11,]
		\addplot table {Y242_20Pa_gdot_decimated.txt};
        \addplot table {Y239_40Pa_gdot_decimated.txt};
        \addplot table {Y238_50Pa_gdot_decimated.txt};
        \addplot table {Y246_60Pa_gdot_decimated.txt};
        \addplot table {Y236_100Pa_gdot_decimated.txt};
        \addplot table {Y249_120Pa_gdot_decimated.txt};
		\end{scope}
		\addplot[yellow, ultra thick, domain={0.001:0.99}] {0.378*x^(-0.96) + 0.187/(1-x)};
		
	\nextgroupplot[
		xmin=1e-5, xmax=2, xlabel={$(\tau_f-t)/\tau_f$},
		x dir=reverse,
		xmode=log,ymode=log,
		xtick={1e-4, 1e-2, 1},
	]
        \addplot table[x expr={1-\thisrowno{0}}] {Y242_20Pa_gdot_decimated.txt};
        \addplot table[x expr={1-\thisrowno{0}}] {Y239_40Pa_gdot_decimated.txt};
        \addplot table[x expr={1-\thisrowno{0}}] {Y238_50Pa_gdot_decimated.txt};
        \addplot table[x expr={1-\thisrowno{0}}] {Y246_60Pa_gdot_decimated.txt};
        \addplot table[x expr={1-\thisrowno{0}}] {Y236_100Pa_gdot_decimated.txt};
        \addplot table[x expr={1-\thisrowno{0}}] {Y249_120Pa_gdot_decimated.txt};
		\addplot[yellow, ultra thick, samples at={1e-5,1e-4, 1e-3,1e-2,0.1,0.2,0.3,0.4,0.5,0.6,0.7,0.8,0.9, 0.99, 0.999, 0.9999, 0.99999}] {0.378*(1-x)^(-0.96) + 0.187/x};
		\addplot[domain=1e-4:1e-2] {0.1*x^(-1)} node[midway, below] {-1};
	\end{groupplot}
\end{tikzpicture}

\begin{itemize}
\item reversible, homogeneous creep $\rightarrow$ irreversible fracture growth
\item nucleation regime \alert{NOT universal}
\begin{itemize}
	\item possible to reform bond?
	\item friction between swollen strands?
	\item glassy dissipation?
\end{itemize}
\end{itemize}

\end{frame}

\begin{frame}{Over-acidification summary}
\begin{itemize}
\item The critical network is still there but over-compressed
\item Transition to soft glassy rheology
\item Power-law creep and fracture growth still present but do not rule everything
\item[$\Rightarrow$] Rupture scenario of critical gel only partially valid for other gels.
\end{itemize}
\end{frame}


%\begin{frame}{Merciless yoghurt breakers}
%\textit{Leocmach et al. PRL 113, 038303 (2014), \footnotesize ArXiv 1401.8234}
%
%\bigskip
%\begin{tabu}{X[c]X[c]c}
%\includegraphics[height=0.3\textheight]{Chris}&
%\includegraphics[height=0.3\textheight]{Thibaut}&
%\includegraphics[height=0.3\textheight]{Seb}\\
%Christophe Perge & Thibaut Divoux & Sebastien Manneville\\
%PhD student & CNRS researcher & Professor\\
%E.N.S. Lyon &  CRPP Bordeaux & E.N.S. Lyon\\
%\end{tabu}
%
%
%\bigskip
%\begin{description}[Special thanks]
%\item[Special thanks] to Alan Parker (Firmenich) for prompting this study and providing casein \& GDL
%\end{description} 
%
%\end{frame}

\appendix
\newcounter{finalframe}
\setcounter{finalframe}{\value{framenumber}}

\begin{frame}[plain]
\end{frame}

\begin{frame}{Creep rheology: Three regimes}
\tikzsetnextfilename{three_regimes}
\begin{tikzpicture}
\begin{loglogaxis}[
   name=g,
	width=\textwidth,
	height=0.8\textheight,
	xmin=2e-2, xmax=4e5, xlabel={time (\si{\second})},
	ymin=0.2,ymax=2,
	ytick={0.2, 0.4,0.8,1.6}, yticklabels={0.2, 0.4,0.8,1.6},
	ylabel={strain $\gamma$},
	extra tick style={grid=major},%
	extra y ticks={1}, extra y tick labels={1.0},
	cycle list name=earthy,
	axis on top,
	]
	\fill[Accent1!20]
		(axis cs:10,0.4) ellipse[rotate=-30, x radius=0.2\textwidth, y radius=0.12\textwidth]
		(axis cs:2e2,1.6) ellipse[x radius=0.35\textwidth, y radius=0.05\textwidth]
		(axis cs:3e3,0.85) ellipse[rotate=-7.5, x radius=0.3\textwidth, y radius=0.07\textwidth]
		;
	\node[above right] at (axis cs:1e1,0.2) {primary};
	\node[below left] at (axis cs:4e5,0.7) {secondary};
	\node[left] at (axis cs:5,1.6) {tertiary};
	\addplot table{Y27_200Pa_gamma_decimated.txt} node (s200){};
	\addplot table{Y38_300Pa_gamma_decimated.txt} node (s300){};
	\addplot table{Y25_400Pa_gamma_decimated.txt}  node (s400){};
	\addplot table{Y32_550Pa_gamma_decimated.txt}  node (s550){};
	\addplot table{Y39_1000Pa_gamma_decimated.txt} node (s1000){};
\end{loglogaxis}
\begin{scope}[anchor=base, every node/.style={yshift=0.2em}]
	\node[red!40!black] at (s200 |- g.outer north) {\SI{200}{\pascal}};
	\node[red!60!black] at (s300 |- g.outer north) {300};
	\node[red!80!black] at (s400 |- g.outer north) {400};
	\node[red] at (s550 |- g.outer north) {550};
	\node[red!80!yellow] at (s1000 |- g.outer north) {1000};
\end{scope}
\end{tikzpicture}

Failure at $\gamma\approx 1$ for a well defined time $\tau_f$
\end{frame}

\setcounter{framenumber}{\value{finalframe}}
\end{document}

