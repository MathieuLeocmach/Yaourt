\documentclass[a4paper, parskip=true, firsthead=false, fromemail=true, foldmarks=false]{scrlttr2}
\usepackage[utf8]{inputenc}
\setkomavar{fromemail}{thomas.gibaud@ens-lyon.fr}
\setkomavar{signature}{Mathieu Leocmach,\\ Mathieu Nespoulous,\\ Sébastien Manneville,\\ Thomas Gibaud\\
{\small Ecole Normale Supérieure de Lyon/CNRS}}


\usepackage{amsmath}
\usepackage{amsfonts}
\usepackage{amssymb}
\usepackage{graphicx}
\usepackage[british]{babel}

\usepackage{kpfonts}
\usepackage{tabu}
\usepackage{wrapfig}

\begin{document}
\begin{letter}{
\vspace{-10\baselineskip}
From:\\
Thomas Gibaud,\\
Laboratoire de physique,\\
Ecole Normale Supérieure de Lyon/CNRS,\\
France\\	
\texttt{thomas.gibaud@ens-lyon.fr}
}
\opening{Dear Dr.~Pàmies,}

\vspace{2\baselineskip}
First we would like to thank you again for your e-mail concerning our paper \emph{Hierarchical wrinkling in a confined permeable biogel} (NM15020540).

We are glad that you noticed that wrinkling is a hot topic. We still believe that our paper is not just another paper on wrinkling but that it presents major breakthroughs in view of the current state of knowledge. Therefore we would be grateful if you could reconsider your decision based on the three following points:
\begin{enumerate}

\item We focus on a porous biogel immersed in a buoyancy-matched viscous medium while previous experiments have focused on non-buoyant interfaces. This original gravity-free case, including selectively engineered confinement boundaries that are non adhesive at the molecular scale, does not only raise true experimental challenges but is also crucial to biological situations.


\item We go much further than associating wrinkling to the interplay of gelation dynamics and confinement constraints. We clearly relate the cause of the excess area that drives the wrinkling instability to the microstructural dynamics of the biogel network. This is unprecedented as other groups have up to now focused on wrinkling instabilities without caring about their interplay with the evolution of the material microstructure during the wrinkling process.

\item The model that we propose and confront to our experimental data is thoroughly new since it does not only rely on elasticity but also on viscosity and porosity, both ingredients that have been overlooked so far and that should be key to many practical situations including the morphogenesis of thin biological tissues.

\end{enumerate}

We sincerely hope that you would find that our paper has an impact large enough to warrant publication in Nature Materials.

\closing{\bf Sincerely yours,} 

\end{letter} 
\end{document}