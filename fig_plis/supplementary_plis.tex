\documentclass[twocolumn,superscriptaddress,showpacs,preprintnumbers,
amsmath,amssymb,prl]{revtex4-1}
\usepackage[utf8]{inputenc}

\usepackage{graphicx} 
\usepackage{xcolor}
\usepackage{hyperref}
\usepackage[range-units = single]{siunitx}
%\usepackage{multirow}
\usepackage{bm}
%\usepackage{dcolumn,ulem}
\usepackage{amsfonts}

\definecolor{Main}{rgb}{1, 0.57, 0}
\definecolor{Accent1}{rgb}{1,0.28,0}
\definecolor{Accent2}{rgb}{1,0.74,0}

\usepackage{xr,xc}
\externaldocument[M-]{article_plis}
\externalcitedocument{article_plis}

% Hack for making figures Say \figurename S\thefigure, e.g. Figure S1:
\makeatletter
\makeatletter \renewcommand{\fnum@figure}
{\textbf{Fig.~S\thefigure}}
\makeatother

\renewcommand{\thetable}{\arabic{table}}
\makeatletter
\makeatletter \renewcommand{\fnum@table}
{\textbf{Tab.~S\thetable}}
\makeatother

\begin{document}
\title{Supplementary informations to Hierarchical wrinkling in a confined permeable biogel}
\author{Mathieu Leocmach}
\affiliation{Universit\'e de Lyon, Laboratoire de Physique, \'Ecole Normale Sup\'erieure de Lyon, CNRS UMR 5672, 46 All\'ee d'Italie, 69364 Lyon cedex 07, France}
\author{Mathieu Nespoulous}
\affiliation{Aix-Marseille Universit\'e, CNRS, MADIREL UMR 7246, Marseille, France}
\author{S\'ebastien Manneville}
\affiliation{Universit\'e de Lyon, Laboratoire de Physique, \'Ecole Normale Sup\'erieure de Lyon, CNRS UMR 5672, 46 All\'ee d'Italie, 69364 Lyon cedex 07, France}
\author{Thomas Gibaud}
\email{thomas.gibaud@ens-lyon.fr}
\affiliation{Universit\'e de Lyon, Laboratoire de Physique, \'Ecole Normale Sup\'erieure de Lyon, CNRS UMR 5672, 46 All\'ee d'Italie, 69364 Lyon cedex 07, France}

\maketitle



\subsection*{Wavelength measurements}

\begin{figure}
	\includegraphics{generations}%
	\caption{Fourier spectra of binary altitudes of Main Figure~\ref{M-fig:dynamics}a corresponding to primary, secondary and tertiary patterns. Inset: reconstructed binary altitudes at time \SI{2}{\hour}. The scale bar is \SI{1}{\milli\metre}.}%
	\label{fig:generations}%
\end{figure}

\begin{figure}
\includegraphics{ripples.pdf}
\caption{Evidence for the ripples generated by the junction to flat patches, see~\cite{LeGoff2014}. (a) Position of the bottom surface of the gel obtained by confocal images. (b) one dimensional cut of the region of interest highlighted in (a) using an orange box. Only the first ripple is visible. (c) Schematic view of the interference between ripples.}
\label{fig:ripples}
\end{figure}

\begin{figure*}
	\includegraphics{patterns}
	\caption{Patterns corresponding to the samples of Supplementary Table~\ref{tab:data}. (a-d) increasing cell thickness. (e-f) larger solid content. (g-i) increasing glycerol content. All pictures are stitching of fluorescent microscopy images except (c,e) which are details of reflected light macroscope images. The scale is common to all panels (scale bar \SI{1}{\milli\metre}). Arrows show the measured primary wavelength.}
	\label{fig:patterns}
\end{figure*}

To measure the wrinkle wavelength $\lambda$, we use large field pictures of the final state. These picture are obtained either by stitching fluorescent microscopy images obtained at 10x magnification, Supplementary Figure~\ref{fig:patterns} except c and e, or by transmitted or reflected light macroscope at magnifications ranging from 1x to 6x, Supplementary Figure~\ref{fig:patterns}c and e. The contrast in such pictures is a measure of the magnitude of the altitude gradient of the gel film. White lines in fluorescence and reflection (respectively black lines in transmission) denote a sharp transition between two extreme altitudes (on top wall or on bottom wall). It is thus possible to reconstruct a binary picture of the pattern, inset of Supplementary Figure~\ref{fig:generations}. Furthermore since the shape and size of a given generation of pattern do not change after formation, it is possible to revert the final binary picture to the configuration of the primary pattern.

Figure~\ref{fig:generations} shows Fourier spectra of binarized altitudes at different times. When only the first generation pattern is present a peak clearly indicates the dominant mode. When the secondary pattern has appeared, the dominant wavelength is halved and the peak much weaker. This allows us to use the secondary pattern is our analysis by setting their wavelength to half of the primary wavelength. Further generations do not seem to appear simultaneously and the corresponding peaks are weak and broad.

\subsection*{Confocal images processing}

At 10x magnification and depending on cell thickness, we acquire stacks of 20 to 50 images spaced by \SI{6.69}{\micro\metre} (Nyquist sampling). Each confocal slice is $(\SI{256}{px})^2$ with pixels of \SI{4.97}{\micro\metre}. At such low resolution details of the gel microstructure are undistinguishable as $\xi$ is smaller than the pixel size for all compositions and the gel appears as a continuous medium.

We first de-noise each plane by a short-range Gaussian blur ($\sigma=\SI{2}{px}\ll\lambda$). Then, for each $(x,y)$, we focus on the z-dependent intensity profile $I(z)$. We define the position of the top surface of the gel $z_\text{top}$ as the point where $I(z)$ first crosses half of its maximum value over the given vertical line. Since this criterion is local, we do not need to correct for background intensity variation. In this way we obtain $z_\text{top}(x,y)$ with subpixel resolution ($\pm\SI{0.1}{px}=\SI{0.67}{\micro\metre}$), and $z_\text{bottom}(x,y)$ in a symmetrical way.

From these data, we obtain for each stack the volume and thickness of the gel phase, area and peak-to-peak amplitude of the mid surface. Velocity is obtained by time differentiation. Once the instability has saturated on the top and bottom walls of the cell, peaks in the histograms of $z_\text{top}$ and $z_\text{bottom}$ yield the measure of the positions of both walls, thus of the cell thickness $e$ and retrospectively of $H_1$ and $H_2$.

\subsection*{Characteristic time measurements}

In most of our samples it is possible to measure directly the peak-to-peak amplitude $A(t)$ from our confocal measurements. However when the wavelength is too large compared to the field of view, this approach breaks down since we cannot observe both the highest and lowest points of the surface. Even when valid, such a local measurement is rather noisy and we prefer to focus on the measure of the vertical velocity $v$ of the gel that can be integrated over the whole field of view:
\begin{equation}
v(t) = \frac{\int \left|\frac{dz}{dt}\right| dxdy}{\int dxdy},
\end{equation}
where $z(x,y,t)$ is the altitude of the mid surface of the gel.

\subsection*{Fraction of free casein as a function of pH}

The fraction of free casein proteins of a \SI{100}{\milli\litre}, 4\%w sodium caseinate dispersion is obtained either by titration by 1 molar HCl, see Supplementary Video~\ref{vid:titration}, or during the action of 4\%w GDL. In both cases, \SI{0.4}{\milli\litre} of the \SI{100}{\milli\litre} dispersion is sampled regularly as a function of pH, and centrifuged (centrifuge 5424, Eppendorf) in Eppendorf tubes at 14500 rpm for \SI{3}{\minute}. The supernatant is diluted and analyzed with a UV absorption spectrometer (USB4000, Ocean Optics). According to~\cite{Roefs1986}, we deduce the fraction of free casein proteins from the adsorption spectrum as:
%
\begin{align}
x_\text{free} &= \frac{A(\text{pH})}{A(\text{pH}=7)},\\
\text{with }A(\text{pH}) &= A_{280}-1.706 A_{320},
\end{align}
%
where $A_{280}$ is the absorption peak at \SI{280}{\nano\metre} due to aromatic amino acids in the casein molecules plus a turbidity background whereas the absorption peak at \SI{320}{\nano\metre} is only due to turbidity. 

Above the isoelectric point, the homogeneous acidification by GDL continuously creates very small protein aggregates that are difficult to centrifuge. Such suspensions are too turbid and too rapidly evolving to reliably measure $x_\text{free}$.

\subsection*{Microstructure measurements}

\begin{figure}
	\includegraphics{nonmonotonic}%
	\caption{Microstructure evolution of a 4\%w sodium caseinate solution acidified with 4\%w GDL in water. (a) Details of confocal images in fully adhesive conditions (top row) and with no adhesion on top (bottom row). The scale bar is \SI{10}{\micro\metre}. Arrows indicates the pore size. (b) Corresponding Fourier spectra when no adhesion on top, same times as (a). Lines are fits to equation~(\ref{eq:fractalS}). The inset shows a schematic association between real space structures and wave numbers. (c-e) Evolution of cut-off length $\xi$, susceptibility $\chi$ and fractal dimension $d$ in full adhesive (black line) and no adhesion on top (orange dashed line) situations. Symbols indicate the times of the images in (a).}%
	\label{fig:nonmonotonic}%
\end{figure}

To follow the evolution of the microstructure of the gel, we acquire slices of $(\SI{2048}{px})^2$ with pixels of \SI{98}{\nano\metre}, Supplementary Figure~\ref{fig:nonmonotonic}a and Supplementary Videos~\ref{vid:stick63} and \ref{vid:nostick63}. After Hann windowing we perform Fourier transform and radially average the resulting spectrum to obtain $I(q)$, see Supplementary Figure~\ref{fig:nonmonotonic}b that is fitted by the following fractal form~\cite{VanDijk1986}:
\begin{equation}
I(q) = \frac{2\chi\Gamma(d)}{\left(1+\left((d+1)\xi q\right)^2\right)^{d/2}}
\label{eq:fractalS}
\end{equation}
where the susceptibility $\chi$ is related to the fluorescence contrast between the network and the solvent, the cut-off length $\xi$ corresponds to the largest pore size and $d$ is the fractal dimension. $\Gamma$ is the Gamma function. Equation~(\ref{eq:fractalS}) is a good approximation to the 2D Fourier transform of a fractal pair correlation function $g(r) \sim 1+ e^{-r/\xi} r^{d-2}$.
%En 3D on peut citer \cite{Martin1987,Chen1986}

Figure~\ref{M-fig:acidification}d of the main text and Supplementary Figure~\ref{fig:nonmonotonic}c-e show the evolution of the fitting parameters in two different sets of boundary conditions: (i) complete adhesion of the gel on all walls, preventing syneresis and swelling; (ii) complete adhesion except on the top wall, allowing syneresis and swelling. The latter situation is a proxy to the situation where we observe wrinkling, since it is necessary to keep the gel into the thin focal plane of the 63x objective lens of the confocal microscope. To do so, we coat only the microscope slide with acrylamide brush. The cover slip is cleaned but not coated, thus keeping strong adhesion on the side closer to the objective lens. Neither buckling nor wrinkling is detected in such a configuration.

In situation (i), the gel keeps its topology throughout the gelation and over-acidification. Left of Supplementary Figure~\ref{fig:nonmonotonic}a shows that the arms of the gel first shrink laterally while keeping their length, and then swell again. Syneresis and swelling are thus local processes. By contrast in situation (ii), the gel as a whole compresses while forming. Since our focal plane is at a fixed depth, we cannot follow the topology of the same subset of the network, however the pore size is clearly smaller than in situation (ii). The isoelectric gel, at minimum gel volume, is probably isotropic or only marginally anisotropic in $z$. This explains that subsequent swelling is not occurring only along $z$ but also in plane, causing buckling.

$\xi$ is also systematically measured in the same way at the end of each wrinkling experiment and reported in Supplementary Table~\ref{tab:data}. %Final fractal dimension was always close to 2 within error bars. Comparing $\chi$ across different setting of the microscope is not relevant.


\begin{figure*}%
	\includegraphics{permeability}%
	\caption{Permeability measurements. (a) Schematic representation of the experimental set up. (b-c) Evolution of the height of the interface in tube 1 relative to the final height in tube 2. The black line is the best exponential fit $Ae^{-t/\tau}$. (b) Gel is 4\%w casein, 4\%w GDL in water, $H=\SI{2.3}{\milli\metre}$ and $\tau=\SI{57}{\minute}$. (c) Same as (b) for a 50\%w glycerol--water mixture, $H=\SI{4}{\milli\metre}$ and $\tau=\SI{100}{\hour}$.}%
	\label{fig:permeability}%
\end{figure*}
\subsection*{Permeability measurements}

To measure the permeability of our gels, we adapt a method from \cite{VanDijk1986}. First, the gel of interest is formed at the lower extremity of a thin glass tube. The tube is held vertical in a water saturated atmosphere to prevent evaporation until the pH has reached the isoelectric point ($\approx\SI{1}{\hour}$ for our most common composition). Then, we dip the tube in a bath of acetic acid/acetate buffer at pH=4. The buffer has the same glycerol content as the gel solvent, so that the viscosity is constant throughout the experiment. Alongside the first tube, see Supplementary Figure~\ref{fig:permeability} a, we dip a second identical tube with no gel in it. A webcam (Logitech Webcam Pro 9000) captures the rise of the liquid in each tube. Since the liquid height in tube 2 reaches its final value quickly, we measure directly $z_\infty-z(t)$.

Darcy law through the gel of permeability $\alpha$ and height $H$ reads 
\begin{equation}
\frac{dz}{dt} = \frac{\alpha}{\eta}\frac{\Delta P}{H}
\end{equation}
where $\Delta P=\rho g(z_\infty-z(t))$ is the hydrostatic pressure drop through the gel. We use tabulated values of water-glycerol mixtures for the density $\rho$ and viscosity $\eta$ of the buffer. We can thus write
\begin{equation}
\frac{d(z-z_\infty)}{dt} = -\frac{z-z_\infty}{\tau}\text{, with\,} \tau = \frac{\eta H}{\alpha \rho g}.
\end{equation}

By fitting $z(t)-z_\infty$ to an exponential, see Supplementary Figure~\ref{fig:permeability} b-c, we obtain $\tau$ and thus $\alpha$. Note that even with $H$ of the order of a few millimeters high glycerol contents yield $\tau\approx \SI{100}{\hour}$ due to both high viscosity and low permeability. Obtaining $z_\infty$ without tube 2 would be time prohibitive.









\subsection*{Mixed Darcy--Poiseuille model}

\subsubsection*{General framework for wrinkling}

The mechanical equilibrium of a plate of thickness $h$, Young modulus $E$ and Poisson ratio $\nu$ submitted to a compression load $\sigma_\parallel$ along $x$ and a transverse load $\sigma_\perp$ along $z$ writes
\begin{equation}
\frac{E h^3}{12(1-\nu^2)}\frac{\partial^4 w}{\partial x^4} + \sigma_\parallel h \frac{\partial^2 w}{\partial x^2} = \sigma_\perp,
\label{eq:mecheq}
\end{equation}
where $w(x,t)$ is the deflection of the plate along $z$~\cite{Biot1957}. Since we deal with slow rates of deformation, inertia is neglected.

We decompose the deflection in sinusoidal modes $w(x,t) =  A \sin (qx)$ of wavevector $q$. Equation~(\ref{eq:mecheq}) then reads
\begin{equation}
\left(\frac{1}{12} B h^3 q^4 - \sigma_\parallel h q^2\right)w = \sigma_\perp,
\label{eq:mecheqop}
\end{equation}
where $B=E/(1-\nu^2)$ is the modulus of the plate.

Following \cite{Biot1957}, we consider the case where the compression load $\sigma_\parallel$ is increased quasistatically. The observed mode is the one that arises at the lowest possible $\sigma_\parallel$, i.e. the mode minimising $\sigma_\parallel$.

\subsubsection*{Elastic film in an infinite elastic medium}

In this framework, we recall here the case of an elastic film lying over an infinite elastic medium~\cite{Gough1940,Bijlaard1946,Biot1957,Cerda2003}. The deformation of the substrate of modulus $B_s$ gives the transverse load as $\sigma_\perp = -q B_s w$, and thus the compression load writes
\begin{equation}
\sigma_\parallel = B \frac{h^2}{12} q^2 + \frac{B_s}{hq}.
\label{eq:sigma0}
\end{equation}

Minimising $\sigma_\parallel$ as a function of $q$ yields $hq = \left(6B_s/B\right)^{1/3}$ \cite{Biot1957,Cerda2003} so that the dominant wavelength is 
\begin{equation}
\lambda = 2\pi h \left(\frac{B}{6B_s}\right)^{1/3}.
\label{eq:lambdaElEl}
\end{equation}

\subsubsection*{Elastic film on a viscous layer}

\begin{figure}
\includegraphics{notations}%
\caption{Schematic side view of the slit. A single wavelength is represented but the longitudinal dimension $L$ of the system is much larger than $\lambda$. The constant thickness gel film (orange) is surrounded by solvent. The region of interest for mass conservation is highlighted in gray. Arrows are oriented along $x$ or $z$, defining positive sign.}%
\label{fig:notations}%
\end{figure}

Let us now turn to the case of an elastic, impermeable film on a single viscous layer. As in \cite{Huang2002} we neglect gravity, although it would be difficult to do so in practice at a free interface. Assuming the layer thickness $H$ to be much smaller than $\lambda_d$, we can use the lubrication approximation. Further considering only small deflections, we may neglect the displacement of the plate along $x$. The flux along $x$ is thus of the Poiseuille form
\begin{equation}
Q = -\frac{H^3}{12\eta}\frac{\partial p}{\partial x}.
\label{eq:PoiseuilleFlux}
\end{equation}
where $p(x)$ is the pressure in the layer of viscosity $\eta$.

Mass conservation (on the lower gray area of Supplementary Figure~\ref{fig:notations}) writes
\begin{equation}
\frac{\partial Q}{\partial x} + \frac{\partial w}{\partial t} = 0.
\label{eq:conservation}
\end{equation}

Assuming an exponentially growing amplitude $A(t) = A_0 \exp(t/\tau)$ and keeping only linear order terms, one gets
\begin{equation}
\frac{12\eta}{\tau} \frac{w}{H^3q^2} = -\sigma_\perp.
\label{eq:PoiseuilleLoad}
\end{equation}

Combining Equations~(\ref{eq:mecheqop}) and (\ref{eq:PoiseuilleLoad}), the compression load writes
\begin{equation}
\sigma_\parallel = \frac{1}{12}B h^2 q^2 + 12\frac{\eta}{\tau}\frac{1}{H^3 h q^4}
\label{eq:sigma0P}
\end{equation}
and is minimised at $q_P$ such that
\begin{equation}
q_P^6 = 2\times 12^2 \frac{1}{h^3H^3\Upsilon},
\label{eq:qP}
\end{equation}
with $\Upsilon = B\tau/\eta$ the viscoelastic factor. The dominant wavelength for the Poiseuille model is thus
\begin{equation}
\lambda_P = \pi\sqrt{hH}\left(\frac{2}{9}\Upsilon\right)^{1/6}.
\end{equation}
%
We note that the mode minimising $\sigma_\parallel$ is also maximising the growth rate
\begin{equation}
\frac{1}{\tau} = \frac{H^3q^2}{12^2\eta}\left(12\sigma_\parallel - Bh^3q^4\right).
\end{equation}

When the elastic film is sandwiched between two viscous layers of respective thickness $H_1$ and $H_2$, the linearity of equation~(\ref{eq:mecheq}) allows us to use an effective substrate of thickness 
\begin{equation}
\frac{1}{H^3} = \frac{1}{H_1^3}+\frac{1}{H_2^3}.
\end{equation}
This corresponds to the experimental case where the gel film is far from both of the cell walls. Since we have an almost buoyant gel and no more free surface, gravity can indeed be neglected.

\subsubsection*{Porous elastic film with no possible Poiseuille flow}
Another limiting case is that of a porous elastic film immersed in a viscous medium where longitudinal resistance to flow is infinite. It is the case in our experiments when the film is at contact with one of the walls. The only way to deform the film is to flow some liquid through the film of permeability $\alpha$. The transverse load derives from the Darcy law:
\begin{equation}
\sigma_\perp = p_1-p_2 = \frac{h\eta}{\alpha}\frac{\partial w}{\partial t}.
\label{eq:DarcyLoad}
\end{equation}

Combining equations~(\ref{eq:mecheqop}) and (\ref{eq:DarcyLoad}), the compression load writes
\begin{equation}
\sigma_\parallel = \frac{1}{12}B h^2 q^2 + \frac{\eta}{\tau}\frac{1}{\alpha q^2}
\label{eq:sigma0D}
\end{equation}
and is minimised at $q_D$ such that
\begin{equation}
q_D^4 = \frac{12}{h^2\alpha\Upsilon}.
\label{eq:qD}
\end{equation}
%
The dominant wavelength for Darcy model is thus
\begin{equation}
\lambda_D = 2\pi h^{1/2}\alpha^{1/4}\left(\frac{\Upsilon}{12}\right)^{1/4}.
\end{equation}
Here also the dominant mode maximizes the growth rate.


\subsubsection*{Porous elastic film between two viscous layers}
Finally the most realistic scenario mixes both Poiseuille and Darcy mechanisms, i.e. considers a porous elastic film between two viscous layers of arbitrary thickness. In this case, see Supplementary Figure~\ref{fig:notations}, the mass conservation $Q_1 + Q_2 = 0$ over the whole height of the cell yields
\begin{equation}
H_1^3 p_1 + H_2^3 p_2 = 0.
\label{eq:pressures}
\end{equation}

The mass conservation in the lower viscous layer is the same as equation~(\ref{eq:conservation}) with an added leak $v$ due to the porosity
\begin{equation}
\frac{\partial Q_1}{\partial x} + \frac{\partial w}{\partial t} + v = 0.
\label{eq:conservationDarcy}
\end{equation}
%
$v$ can be expressed by the Darcy law
\begin{equation}
v = \frac{\alpha}{\eta} \frac{p_1-p_2}{h}.
\label{eq:Darcy}
\end{equation}

Combining equations~(\ref{eq:pressures}), (\ref{eq:conservationDarcy}) and (\ref{eq:Darcy}) we obtain the transverse load
\begin{equation}
\sigma_\perp = p_1-p_2 = - \frac{\eta}{\tau} \frac{w}{\frac{H^3 q^2}{12} + \frac{\alpha}{h}},
\label{eq:MixtLoad}
\end{equation}
%
Note that we recover equation~(\ref{eq:PoiseuilleLoad}) when $\alpha \rightarrow 0$ and equation~(\ref{eq:DarcyLoad}) when $H \rightarrow 0$. We can recast this intuition in terms of the dimensionless number $H/H^*$ where $H^*$ is obtained by equating $q_P$ and $q_D$:
\begin{equation}
H^* = 2^{2/3} 3^{1/6} \alpha^{1/2} \Upsilon^{1/6}.
\end{equation}

Analytic minimisation of $\sigma_\parallel$ knowing $\sigma_\perp$ is possible but cumbersome. Instead we rewrite equation~(\ref{eq:mecheqop}) in terms of previously calculated $\lambda_P$ and $\lambda_D$:
\begin{align}
\frac{12\sigma_\parallel}{q_D^2 h^2 B} &= \left(\frac{q}{q_D}\right)^2 + \frac{1}{2\frac{q_D^2}{q_P^6}q^4  + \left(\frac{q}{q_D}\right)^2}\\
&= \left(\frac{\lambda_D}{\lambda}\right)^2 + \frac{1}{2\frac{\lambda_P^6}{\lambda_D^2\lambda^4}  + \left(\frac{\lambda_D}{\lambda}\right)^2}
\end{align}
and we minimise this expression numerically.

\subsection*{Samples characteristics}
To test the theoretical predictions we have fully characterised 9 samples. Their characteristics are gathered in Supplementary Table \ref{tab:data}.


\begin{table*}
\begin{tabular}{cSSSScSrrScSSSScSSSSSS}
%\toprule
&\multicolumn{4}{c}{Preparation} & &  \multicolumn{4}{c}{Properties} & & \multicolumn{4}{c}{Measurements} & & \multicolumn{6}{c}{Results} \\ 
\cline{2-5} \cline{7-10} \cline{12-15} \cline{17-22}\\[-2ex]
\# &{cas} & {GDL} & {gly} & {$e$} && {$\eta$} & \multicolumn{1}{c}{$G^\prime$} & \multicolumn{1}{c}{$\alpha$} & {$\xi$} && {$h$} & {$H_1$} & {$H_2$} & {$v_\text{max}$} && {$\Upsilon$} & {$H/H^*$} & {$\lambda_\text{exp}$} & {$\lambda_{D}$} & {$\lambda_{P}$} & {$\lambda_{D+P}$} \\ 
&{\%w} & {\%w} & {\%w} & \si{\micro\metre} && \si{\milli\pascal\second} & \multicolumn{1}{c}{\si{\pascal}} & \si{\square\nano\metre} & \si{\micro\metre} && \si{\micro\metre} & \si{\micro\metre} & \si{\micro\metre} & \si{\micro\metre/\second} && {$10^8$} &  & \si{\milli\metre} & \si{\milli\metre} & \si{\milli\metre} & \si{\milli\metre} \\ 
\cline{2-5} \cline{7-10} \cline{12-15} \cline{17-22}\\[-2ex]
%ech2
a & 4 & 4 & 0 & 74.5 && 1.0 & 487 & 67000 & 4.9 && 38.3 & 6.3 & 29.3 & 0.11 && 1.62 & 0.54 & 0.89 & 1.20 & 0.88 & 1.22 \\ 
& & & & && & & & && 40.2 & 1.7 & 32.9 & 0.09 && 0.70 & 0.09 & 0.45 & 1.00 & 0.39 & 1.00\\
b & 4 & 4 & 0 & 92.4 && 1.0 & 487 & 67000 & 4.9 && 27.8 & 13.5 & 50.8 & 0.20 && 1.57 & 1.17 & 0.85 & 1.01 & 1.10 & 1.20 \\ 
c & 4 & 4 & 0 & 106.1 && 1.0 & 487 & 67000 & 4.9 && 32.7 & 30.7 & 42.9 & 0.14 && 2.54 & 2.23 & 1.52 & 1.24 & 1.85 & 1.90 \\ 
d & 4 & 4 & 0 & 138.0 && 1.0 & 487 & 67000 & 4.9 && 48.2 & 32.0 & 63.9 & 0.11 && 4.05 & 2.28 & 1.62 & 1.69 & 2.56 & 2.62 \\
& & & & && & & & && 57.4 & 2.1 & 80.9 & 0.05 && 1.25 & 0.09 & 0.81 & 1.37 & 0.6 & 1.37 \\
\cline{2-5} \cline{7-10} \cline{12-15} \cline{17-22}\\[-2ex]
e & 8 & 8 & 0 & 85.2 && 1.0 & 1817 & 2150 & 3.1 && 49.8 & 10.0 & 26.8 & 0.10 && 6.50 & 3.79 & 1.00 & 0.82 & 1.60 & 1.61 \\ 
& & & & && & & & && 60.5 & 1.1 & 24.0 & 0.11 && 1.94 & 0.42 & 0.50 & 0.77 & 0.55 & 0.82\\
f & 8 & 8 & 0 & 91.0 && 1.0 & 1817 & 2150 & 3.1 && 23.4 & 30.2 & 47.7 & 0.10 && 12.05 & 9.70 & 1.50 & 0.66 & 2.04 & 2.04 \\ 
& & & & && & & & && 30.7 & 1.6 & 55.0 & 0.02 && 15.60 & 0.32 & 0.75 & 0.82 & 0.59 & 0.83 \\
\cline{2-5} \cline{7-10} \cline{12-15} \cline{17-22}\\[-2ex]
g & 4 & 4 & 20 & 63.7 && 1.6 & 1142 & 24600 & 2.7 && 39.1 & 1.2 & 10.9 & 0.21 && 0.86 & 0.20 & 0.49 & 0.80 & 0.36 & 0.80 \\ 
h & 4 & 4 & 35 & 82.3 && 2.7 & 1858 & 25300 & 1.8 && 58.3 & 0.9 & 12.6 & 0.02 && 10.98 & 0.13 & 0.61 & 1.06 & 0.56 & 1.06 \\ 
i & 4 & 4 & 50 & 95.3 && 5.4 & 4660 & 5700 & 1.5 && 65.1 & 7.9 & 14.4 & 0.02 && 16.30 & 1.52 & 1.23 & 1.50 & 1.85 & 1.96 \\ 
\end{tabular}
\caption{Characteristics of the samples used for Figure~\ref{M-fig:DarcyPoiseuille}. Lines where preparation and properties are left blank correspond to the average of the secondary blisters of the previous line. `cas', `GDL' and `gly' indicate the weight fraction of sodium caseinate, GDL and glycerol respectively. $\xi$ is the pore size at the end of each experiment. $h$ is the thickness of the gel film just before buckling.}
\label{tab:data}
\end{table*}


\subsection*{Supplementary Videos}

\begin{enumerate}
\renewcommand{\labelenumi}{\textbf{Video \arabic{enumi}:}}
\itemindent 8mm
\item\label{vid:transmitted} Transmitted light microscopy images of the pattern formation in a 4\%w casein and 4\%w GDL aqueous dispersion. Top and bottom slides of the optical cell are coated with acrylamide brushes. The total duration of the Video is \SI{2}{\hour}~30. The scale bar is \SI{1}{\milli\metre}.

\item\label{vid:reconstructed} 3D reconstruction of the pattern formation using confocal microscopy in a 4\%w casein and 4\%w GDL aqueous dispersion, Supplementary Table~\ref{tab:data}c. 10\% of the proteins are fluorescently labelled. Top and bottom slides of the optical cell are coated with acrylamide brushes. The total duration of the Video is \SI{2}{\hour}~30. The scale bar is \SI{1}{\milli\metre}.


%\item[Video 3] Titration of a dispersion with 1\%w casein by 1 molar HCl. The duration of the Video is \SI{13}{\minute}. The scale bar is XXX $\mu$m


\item\label{vid:titration} Titration of a dispersion with 4\%w casein by 1 molar HCl. The total duration of the Video is \SI{14}{\minute}. The Becher diameter is \SI{55}{\milli\metre}.


\item\label{vid:stick63} Confocal microscopy $(x,y)$ cut of the evolution of the local structure of a 4\%w casein and 4\%w GDL aqueous dispersion when sticky boundary are on. 10\% of the proteins are fluorescently labelled. The total duration of the Video is \SI{132}{\minute}. The scale bar is \SI{10}{\micro\metre}.

\item\label{vid:nostick63} Confocal microscopy $(x,y)$ cut of the evolution of the local structure of a 4\%w casein and 4\%w GDL aqueous dispersion when the top wall is coated with acrylamide brushes and the bottom wall remains sticky. 10\% of the proteins are fluorescently labelled. The total duration of the Video is \SI{147}{\minute}. The scale bar is \SI{10}{\micro\metre}.

\item\label{vid:cutDarcy} Confocal microscopy $(x,z)$ cut of the evolution of the gel film deflection of a 4\%w casein and 4\%w GDL aqueous dispersion when both top and bottom walls are coated with acrylamide brushes, Supplementary Table~\ref{tab:data}a. 10\% of the proteins are fluorescently labelled. The scale bar is \SI{100}{\micro\metre} (with a pixel size ratio of 1.3). Here the gel initially lies close to the bottom wall, $H<H^*$. Darcy mechanism dominates the formation of the primary pattern. The total duration of the video is \SI{145}{\minute}.

\item\label{vid:cutPoiseuille} Confocal microscopy $(x,z)$ cut of the evolution of the gel film deflection of a 4\%w casein and 4\%w GDL aqueous dispersion when both top and bottom walls are coated with acrylamide brushes, Supplementary Table~\ref{tab:data}c. 10\% of the proteins are fluorescently labelled. The scale bar is \SI{100}{\micro\metre} (with a pixel size ratio of 1.3). Here the gel initially lies far from either top or bottom wall, $H>H^*$. Poiseuille mechanism dominates the formation of the primary pattern. The total duration of the video is \SI{181}{\minute}.

\end{enumerate}


\end{document}