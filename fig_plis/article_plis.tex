\documentclass[twocolumn,superscriptaddress,showpacs,preprintnumbers,
amsmath,amssymb,prl]{revtex4-1}
\usepackage[utf8]{inputenc}

\usepackage{graphicx}
\usepackage{xcolor}
\usepackage{hyperref}
\usepackage{siunitx}
%\usepackage{multirow}
\usepackage{bm}
\usepackage{dcolumn,ulem}
\usepackage{amsfonts}

\definecolor{Main}{rgb}{1, 0.57, 0}
\definecolor{Accent1}{rgb}{1,0.28,0}
\definecolor{Accent2}{rgb}{1,0.74,0}

\newcommand{\seb}[1]{\textbf{\color{blue}#1}} % remarques Seb
\newcommand{\sseb}[1]{\sout{\textbf{\color{blue}#1}}} % remarques Seb 

\begin{document}
\title{Hierarchical wrinkling in a confined permeable biogel}
\author{Mathieu Leocmach}
\affiliation{Universit\'e de Lyon, Laboratoire de Physique, \'Ecole Normale Sup\'erieure de Lyon, CNRS UMR 5672, 46 All\'ee d'Italie, 69364 Lyon cedex 07, France}
\author{Mathieu Nespoulous}
\affiliation{Aix-Marseille Université, CNRS, MADIREL UMR 7246, Marseille, France}
\author{Sébastien Manneville}
\affiliation{Universit\'e de Lyon, Laboratoire de Physique, \'Ecole Normale Sup\'erieure de Lyon, CNRS UMR 5672, 46 All\'ee d'Italie, 69364 Lyon cedex 07, France}
\author{Thomas Gibaud}
\email{thomas.gibaud@ens-lyon.fr}
\affiliation{Universit\'e de Lyon, Laboratoire de Physique, \'Ecole Normale Sup\'erieure de Lyon, CNRS UMR 5672, 46 All\'ee d'Italie, 69364 Lyon cedex 07, France}

\begin{abstract}
\seb{\it Ajouter l'abstract.}
\end{abstract}

\maketitle

\section*{Introduction}


There are many ways and reasons for a film to wrinkle. An elastic film buckles due to excess area with respect to its boundaries. Wrinkling is a buckling hindered by a substrate: the mismatch of elastic properties between the film and the substrate selects the preferred wrinkling wavelength~\cite{Gough1940, Bijlaard1946}. On the engineering side the last two decades have seen a bloom of methods to obtain well-controlled patterns via linear~\cite{Bowden1998, Genzer2006, Hu1998, Kim2010, Vandeparre2011, Li2013} or nonlinear wrinkling~\cite{Efimenko2005, Guvendiren2010, Kim2011a, Brau2011}. Such patterns can be triggered by temperature dilation, swelling or the removal of pre-strain. On the biology side wrinkling-controlled morphogenesis is ubiquitous: ageing and the loss of elastic fibres makes our skin wrinkles~\cite{Bissett1987, Genzer2006}, difference in growth rates between the gut tube and its dorsal anchoring is responsible for the vilification of guts~\cite{Savin2011, Ciarletta2014, Shyer2013}, localized cell death in biofilms focuses mechanical forces and initiates 3D labyrinth pattern~\cite{Trejo2013, Asally2012}. On the physics side analogues of wrinkling have been found in situations where an effective substrate comes from the resistance to uniaxial stretching~\cite{Cerda2003}, from gravity~\cite{Kolinski2009, Vella2009, Pineirua2013, Lucantonio2013}, from boundaries~\cite{Vandeparre2011a, Li2013} or from adhesion~\cite{Vella2009a}. Dealing only with (semi)infinite substrates, Biot generalised the elasticity of both the film and the substrate to viscoelasticity~\cite{Biot1957} or poroelasticity~\cite{Biot1964}, leading to time-dependent kinetic wavelength selection. Stability analysis of a film lying on a thin viscous substrate was studied only recently~\cite{Huang2002}. Since in practice gravity sets the wavelength in such situation~\cite{Smoluchowski1910}, the wrinkling on a confined viscous substrate has not been studied experimentally.

Yet, in a biological context, thin tissues, e.g. epithelimum or endothelium, are naturally confined and immersed in a nearly buoyant medium, e.g. lymph, blood or mucus, a practical situation were a thin viscous substrate would set the wrinkling wavelength. Moreover, most biological films are porous, and little is known on the interplay between permeability and wrinkling~\cite{Ma2004,Longley2013}. Therefore benchmark experiments that explore the possibility of wrinkling in confined porous soft materials immersed in a buoyancy-matched viscous medium are in line to get fundamental insights into such a biologically relevant situation.

Here, we introduce model permeable biogels that produce hierarchical wrinkling as a result of their interplay between their gelation dynamics and the confinement conditions. In slit geometry, we slowly acidify a dispersion of casein, one of the milk proteins.  As a result, not only does the casein form a gel as expected~\cite{Bremer1989} --this is a well-known effect used in cheese or yogurt making-- but the gel starts to produce wrinkling patterns in cascade when casein adhesion to the slit walls is turned off. Using a combination of rheology, light microscopy and confocal microscopy, we demonstrate that upon acidification the casein network spontaneously shrinks then swells at the micron level and wrinkles at the millimetre scale. We systematically investigate such swelling-induced wrinkling of a porous gel film sandwiched between two viscous layers, the whole system being confined between two glass plates. As shown in Fig.~\ref{fig:dynamics} and in Supplementary Movies 1 and 2, transverse confinement leads to saturation of the amplitude on both the top and bottom walls followed by a cascade of secondary and ternary buckling. Such cascade buckling has already been observed by~\cite{Roman1999} in a one-dimensional macroscopic system, a sheet of steel forced to buckle between two solid plates. In addition one has to note that the primary wavelength is much smaller than the size of the system, implying that a simple buckling scenario cannot explain the observations in Fig.~\ref{fig:dynamics} and that only a wrinkling instability can be invoked. 

\begin{figure}[ht!]
	\includegraphics{dynamics}%
	\caption{Dynamics of pattern formation in a confined film of casein gel (caseinate 4\%w, GDL 4\%w). (a) Light transmission microscopy. Successive generations of patterns are highlighted in colour to stress the absence of growth after formation. (b) 3D reconstruction from fluorescent confocal microscopy. Scale bars are \SI{1}{\milli\metre}.}%
	\label{fig:dynamics}%
\end{figure}

The main goal of the present study is thus to identify the selection mechanism of the wavelength \seb{and to discuss its implication for practical situations (?)}. In the following we first describe the physical chemistry that allows us to form such patterns. Then we  model the situation and identify two possible selection mechanisms, namely porous Darcy flow or viscous Poiseuille flow. Comparing our experimental data to these two models we finally show that a combination of both Darcy and Poiseuille mechanisms accurately describes our data. This combined model pins down the dynamical origin of the constraints exerted on the gel and nicely predicts the wrinkling wavelength.
%Poiseuille~\cite{Poiseuille1842} flows of the solvent above and below the gel film for long wavelengths and a new mechanism, Darcy~\cite{Darcy1856} flows of the solvent through the gel film for short wavelengths.

\section*{Pattern formation}

\begin{figure}[b]
	\includegraphics{acidification}%
	\caption{Acid-induced protein gel properties behave nonmonotonically with pH. (a) pH decrease over time in a 4\%w sodium caseinate solution acidified by 4\%w GDL. The horizontal line indicates isoelectric pH of casein. (b) Evolution of the storage modulus $G'$ measured in a rheometer with full adhesion to rotor and stator. (c) Evolution of pore size measured by confocal microscopy in a slit geometry with either adhesion to all walls (black line) or no adhesion to top wall, i.e. allowed syn\ae{}resis and swelling (orange dashed line).}%
	\label{fig:acidification}%
\end{figure}


The slow acidification of a protein (sodium caseinate) solution down to the isoelectric pH induces gelation~\cite{Bremer1989} (Fig.~\ref{fig:acidification}a). Since caseins strongly adsorb to interfaces, the resulting gel sticks to the container walls, allowing rheological measurements (Fig.~\ref{fig:acidification}b). Further decrease of the pH beyond the isoelectric point makes the casein aggregates swell (Supplementary Fig.~\ref{fig:nonmonotonic}), thus weakening the gel~\cite{Braga2006}. In a microscopy cell of slit geometry coated so that the casein cannot adsorb on the top wall, this microscopic nonmonotonic behaviour translates into little variation of the pore size $\xi$ but to solvent expulsion out of the gel phase (syn\ae{}resis) followed by global swelling (Fig.~\ref{fig:acidification}c and Supplementary Fig.~\ref{fig:nonmonotonic}).



Figure~\ref{fig:sideview}a shows confocal measurements in the case where both top and bottom walls have been coated. Starting from a homogeneous solution (first panel Fig.~\ref{fig:sideview}a where black margins correspond to the glass walls), the gel forms and simultaneously expels solvent on both sides. Since two interfaces clearly separate a protein-rich phase and a protein-poor phase, we model the resulting situation as a porous elastic film of thickness $h$ sandwiched between two viscous layers of respective thicknesses $H_1$ and $H_2$ in a cell of thickness $e=H_1+h+H_2$ (Fig.~\ref{fig:sideview}b). Wavelength being much larger than thickness, we can assume small strain in the film and lubrication in the layers. Relative increase in protein concentration during syn\ae{}resis can be deduced from the brighter fluorescence in the second panel of Fig.~\ref{fig:sideview}a. Extracting the top and bottom surface of the gel phase allows for quantitative measurements of the relative volume (Fig.~\ref{fig:sideview}c) and excess surface area of the gel film (Fig.~\ref{fig:sideview}d) (see Supplementary Methods). This confirms that the instability is due to swelling.

Note that adhesion to side walls is necessary to prevent total syn\ae{}resis and thus allow subsequent planar stresses to develop while swelling: the only situation were wrinkling was observed is when both top and bottom walls are coated. We also confirmed that at any given point in time and space the thickness $h$ of the film of gel is homogeneous. The contrast in transmission microscopy (Fig.~\ref{fig:dynamics}a) is therefore not due to thickness inhomogeneities but to altitude gradients. Indeed the length of the optical path along the vertical direction $z$ through the gel phase is larger where the gel makes a slope compared to a flat situation, as indicated by the yellow vertical lines in Fig.~\ref{fig:sideview}b.

\begin{figure}
	\includegraphics{sideview}
	\caption{Measurements and modeling. (a) Confocal ($x$,$z$) cuts showing syn\ae{}resis, swelling, wrinkling and cascade buckling. Scale bar is \SI{100}{\micro\metre} (real size ratio). Evolution of (b) volume of the gel phase relative to cell volume and (c) excess area measured by confocal microscopy. Crosses correspond to times in (a). (d) Schematic side view of the cell. The constant thickness gel film (orange) is surrounded by water. Yellow lines show the path of transmitted light through the gel phase. A region of interest is highlighted in gray.}
	\label{fig:sideview}
\end{figure}

\section*{Models and scaling laws}

Longitudinally constrained swelling of the gel film is the origin and driving force of the buckling instability. The gel area becomes in excess compared to the available area, thus the gel must fold. However, this does not explains how a wavelength is selected.

The characteristic length $\lambda^*$ of a wrinkling pattern is set by the competition between the flexural modulus of the film $B=E/(1-\nu^2)$ ($E$ Young modulus, $\nu$ Poisson ratio) and a transverse load $\sigma_\perp$~\cite{Vella2009,Kolinski2009}: 
\begin{equation}
\lambda^* \sim h \left(\frac{B}{\sigma_\perp}\right)^{1/3}.
\label{eq:lstar}
\end{equation}
For small deflections the actual wavelength $\lambda$ is related to $\lambda^*$ and to the wrinkle amplitude $A$ as $\lambda \sim A^{1/4} \lambda^{*3/4}$.

% Pour Mathieu : ajouter quelque part des applications numériques pour les trois longueurs d'ondes gravité, Darcy, Poiseuille
We identify three possible candidates for the transverse load $\sigma_\perp$: the weight or buoyancy of the gel, the porous flow through the gel or the viscous flows in the surrounding water layers. We confirmed that the former plays no significant role by successfully repeating our experiment in a cell held vertically, i.e. with the weight acting longitudinally rather than transversely.
%Since the gel is almost buoyant with the surrounding medium, we expect that the weight plays no significant role, with an elasto-gravitational length at least centimetric. We confirmed this insight by successfully repeating our experiment in a cell held vertically, i.e. with the weight acting longitudinally rather than transversely.

\begin{figure*}
	\includegraphics{Darcy_vs_Poiseuille}
	\caption{Comparing model predictions with measured wavelengths. Dots come from primary pattern, squares from secondary blisters. Dashed lines are the best linear fit through the origin taking only into account the points that should be (a) in Darcy mode $H<H^*$, (b) in Poiseuille mode $H>H^*$ (c) all points. Prefactors are 0.63, 0.69 and 0.67 respectively. Continuous line is the best affine fit ($\lambda_{\rm exp}=0.52\lambda_{P}+\SI{0.33}{\milli\metre}$) to all the points in (b).}
	\label{fig:DarcyPoiseuille}
\end{figure*}

In a first limit where the film of gel would be sitting without sticking to the bottom wall of the cell ($H_1\rightarrow0$), creating a 2D blister requires flowing solvent of viscosity $\eta$ through the gel of permeability $\alpha$. Darcy law~\cite{Darcy1856} relates the pressure gradient, estimated as $\sigma_\perp/h$, to the volume flux per unit area $v = \frac{\alpha}{\eta}\frac{\sigma_\perp}{h} = \frac{dA}{dt}$. Injecting this expression in Eq.~(\ref{eq:lstar}) yields the wavelength of the Darcy mode as
\begin{equation}
\lambda_D \sim h^{1/2} \alpha^{1/4} \Upsilon^{1/4}
\end{equation}
where $\Upsilon = B\tau/\eta$ is a dimensionless viscoelastic factor quantifying the relative stiffness of the film and the solvent at a characteristic time $\tau$. This expression emphasizes the kinetic nature of the wavelength selection through Darcy flow. However we note that the wavelength of a generation of the pattern does not evolve with time (see Fig.~\ref{fig:dynamics}a and Sup. Fig.~\ref{fig:growth}) once the gel is touching both walls. This arrest may be due to bending rigidity~\cite{LeGoff2014}. The relevant $\tau$ corresponds to the times it takes for the amplitude to saturate (see Supplementary Method for the precise definition).

The opposite limiting case is when the gel film lies in the middle of the cell ($H_1=H_2$) and impermeable ($\alpha\rightarrow 0$). There destabilisation over a wavelength $\lambda$ creates a lubrication ($H\ll\lambda$) flow in the viscous layers. By symmetry the transverse load across the gel $\sigma_\perp$ is also the pressure difference over the wavelength. Using a Poiseuille profile for the flow, $\frac{dA}{dt} \sim \sigma_\perp \frac{H^3}{\eta\lambda^2}$~\cite{Poiseuille1842}. Injecting this expression in Eq.~(\ref{eq:lstar}) yields the wavelength of the Poiseuille mode as
\begin{equation}
\lambda_P \sim (hH)^{1/2} \Upsilon^{1/6}
\end{equation}
where we recognize the scaling for the wrinkles of an elastic film on a thin ($H\ll\lambda$) elastic substrate~\cite{Cerda2003} where $\Upsilon$ would be the ratio between the flexural modulus of the film and that of the substrate. Indeed within the lubrication approximation or at low Reynolds numbers one can consider a viscous film as elastic with effective Young modulus $3\eta/\tau$~\cite{Biot1957,Boudaoud2001}.

A more complete derivation of the models can be found in the Supplementary Method and yields the following prefactors:
\begin{align}
\lambda_D &= 2\pi h^{1/2}\alpha^{1/4}\left(\frac{\Upsilon}{12}\right)^{1/4}\\
\lambda_P &= \pi (hH)^{1/2}\left(\frac{2}{9}\Upsilon\right)^{1/6}
\end{align}
where $1/H^3 = 1/H_1^3 + 1/H_2^3$. The crossover between the two models where $\lambda_D=\lambda_P$ is found for $H = H^* \equiv 2^{2/3} 3^{1/6} \alpha^{1/2} \Upsilon^{1/6}$.

To sum up, we expect two regimes of viscosity-dominated wrinkling: for $H\ll H^*$ the main dissipation mechanism would be the flow through the porous film, whereas for $H\gg H^*$ the flow in the viscous layers would dominate at set the wavelength.

\section*{Experimental tests}

\begin{table*}
\begin{tabular}{SSSScSrrScSSSScSSSSSS}
%\toprule
\multicolumn{4}{c}{Preparation} & &  \multicolumn{4}{c}{Properties} & & \multicolumn{4}{c}{Measurements} & & \multicolumn{6}{c}{Results} \\ 
\cline{1-4} \cline{6-9} \cline{11-14} \cline{16-21}\\[-2ex]
{cas} & {GDL} & {gly} & {$e$} && {$\eta$} & \multicolumn{1}{c}{$G^\prime$} & \multicolumn{1}{c}{$\alpha$} & {$\xi$} && {$h$} & {$H_{min}$} & {$H_{max}$} & {$v$} && {$\Upsilon$} & {$H/H^*$} & {$\lambda_{xp}$} & {$\lambda_{D}$} & {$\lambda_{P}$} & {$\lambda_{D+P}$} \\ 
{\%w} & {\%w} & {\%w} & \si{\micro\metre} && \si{\milli\pascal\second} & \multicolumn{1}{c}{\si{\pascal}} & \si{\square\nano\metre} & \si{\micro\metre} && \si{\micro\metre} & \si{\micro\metre} & \si{\micro\metre} & \si{\micro\metre/\second} && {$10^8$} &  & \si{\milli\metre} & \si{\milli\metre} & \si{\milli\metre} & \si{\milli\metre} \\ 
\cline{1-4} \cline{6-9} \cline{11-14} \cline{16-21}\\[-2ex]
%ech2
4 & 4 & 0 & 74.5 && 1.0 & 487 & 67000 & 4.9 && 38.3 & 6.3 & 29.3 & 0.11 && 1.62 & 0.54 & 0.89 & 1.20 & 0.88 & 1.22 \\ 
& & & && & & & && 40.2 & 1.7 & 32.9 & 0.09 && 0.70 & 0.09 & 0.45 & 1.00 & 0.39 & 1.00\\
4 & 4 & 0 & 92.4 && 1.0 & 487 & 67000 & 4.9 && 27.8 & 13.5 & 50.8 & 0.20 && 1.57 & 1.17 & 0.85 & 1.01 & 1.10 & 1.20 \\ 
4 & 4 & 0 & 106.1 && 1.0 & 487 & 67000 & 4.9 && 32.7 & 30.7 & 42.9 & 0.14 && 2.54 & 2.23 & 1.52 & 1.24 & 1.85 & 1.90 \\ 
4 & 4 & 0 & 138.0 && 1.0 & 487 & 67000 & 4.9 && 48.2 & 32.0 & 63.9 & 0.11 && 4.05 & 2.28 & 1.62 & 1.69 & 2.56 & 2.62 \\
& & & && & & & && 57.4 & 2.1 & 80.9 & 0.05 && 1.25 & 0.09 & 0.81 & 1.37 & 0.6 & 1.37 \\
\cline{1-4} \cline{6-9} \cline{11-14} \cline{16-21}\\[-2ex]
8 & 8 & 0 & 85.2 && 1.0 & 1817 & 2150 & 3.1 && 49.8 & 10.0 & 26.8 & 0.10 && 6.50 & 3.79 & 1.00 & 0.82 & 1.60 & 1.61 \\ 
& & & && & & & && 60.5 & 1.1 & 24.0 & 0.11 && 1.94 & 0.42 & 0.50 & 0.77 & 0.55 & 0.82\\
8 & 8 & 0 & 91.0 && 1.0 & 1817 & 2150 & 3.1 && 23.4 & 30.2 & 47.7 & 0.10 && 12.05 & 9.70 & 1.50 & 0.66 & 2.04 & 2.04 \\ 
& & & && & & & && 30.7 & 1.6 & 55.0 & 0.02 && 15.60 & 0.32 & 0.75 & 0.82 & 0.59 & 0.83 \\
\cline{1-4} \cline{6-9} \cline{11-14} \cline{16-21}\\[-2ex]
4 & 4 & 20 & 63.7 && 1.6 & 1142 & 24600 & 2.7 && 39.1 & 1.2 & 10.9 & 0.21 && 0.86 & 0.20 & 0.49 & 0.80 & 0.36 & 0.80 \\ 
4 & 4 & 35 & 82.3 && 2.7 & 1858 & 25300 & 1.8 && 58.3 & 0.9 & 12.6 & 0.02 && 10.98 & 0.13 & 0.61 & 1.06 & 0.56 & 1.06 \\ 
4 & 4 & 50 & 95.3 && 5.4 & 4660 & 5700 & 1.5 && 65.1 & 7.9 & 14.4 & 0.02 && 16.30 & 1.52 & 1.23 & 1.50 & 1.85 & 1.96 \\ 
\end{tabular}
\caption{Characteristics of the samples discussed in the main text. Lines where preparation and properties are left blank correspond to the average of the secondary blisters of the previous line.}
\label{tab:data}
\end{table*}

The two above models yield wavelengths that are of the same order in most of the cases under study. Thus only systematic measurements of the experimental $\lambda$ together with estimates for all the parameters involved in either model shall allow one to discriminate between them (see Supplementary Method for details on such measurements). As summarized in Table~\ref{tab:data} we systematically varied (i) the thickness of the cell, impacting $h$ and the maximal amplitude, (ii) the gel composition and thus both its stiffness and its permeability, (iii) the viscosity of the solvent by adding glycerol, which also impacted the gel stiffness, permeability and $\tau$. Finally, the only parameter that we had no control upon was the initial altitude $H$ of the gel before wrinkling that varied randomly. However we were able to obtain more data at small $H/H^*$ ratios by considering not only the primary patterns but also the secondary ones. Since the saturation of the primary pattern flattens the gel on the top and bottom walls, the secondary blisters always appear in a Darcy situation ($H\ll H^*$). Here the wavelength associated to the secondary blister is taken as half of that of the primary pattern as shown in Supplementary Fig.~\ref{fig:generations}.

Our results are displayed in Fig.~\ref{fig:DarcyPoiseuille}. Although they provide the correct order of magnitude for $\lambda$, neither model alone is able to quantitatively reproduce the full data set. However considering only the points where $H<H^*$ (respectively $H>H^*$), the Darcy (resp. Poiseuille) model fits the data well albeit with a prefactor. Note that it would be easy to miss out on the failure of the Poiseuille model at small wavelengths by adding an unexplained offset (dashed line in Fig.~\ref{fig:DarcyPoiseuille}b). Nonetheless, a complete model (derived in Supplementary Methods), summing up the influence of both Poiseuille and Darcy dissipation shows a better agreement for all values of $H/H^*$.

\seb{\it Ajouter une conclusion.}
%optical elements ?
%wallpaper ?
%role of gravity

\section*{Methods summary}
To prevent adsorption of caseins to surfaces, microscope slides and cover slips were cleaned and subsequently coated with a silane agent (3-(trimethoxysilyl) propyl methacrylate, Sigma Aldrich) to which a polyacrylamide brush was polymerized from methacrylate groups.
Sodium caseinate (Firmenich) was labelled with Dylight~550 NHS ESTER (Thermo Scientific). Excess dye was removed by centrifugation. Unlabelled sodium caseinate powder was dissolved in deionised water (or water--glycerol mixture). To induce gelation glucono-$\delta$-lactone (GDL) in powder (Firmenich) was added to the solution. \SI{45}{\micro\litre} of this solution was immediately mixed with \SI{5}{\micro\litre} of labelled caseinate solution, stirred and injected in the microscopy cell, immediately sealed using ultraviolet-cured glue (Norland Optical) and placed under the microscope before gelation could take place. 
3D data were collected on a Zeiss LSM510 confocal microscope, using \SI{532}{\nano\meter} laser excitation. We used either a 10x (air) lens to measure the geometry of wrinkling (see Supplementary Material) or a 63x (oil) lens to observe the microstructure of the gel. Larger scope pictures, obtained either by stitching fluorescent microscopy images (Nikon Eclipse Ti) or by transmitted/reflected light macroscope (Nikon SMZ745T/Leica DMS1000) were used to measure the wavelength of the primary pattern (Supplementary Figures~\ref{fig:patterns} and \ref{fig:generations}). Secondary pattern wavelength is set to half of the primary wavelength, according to Supplementary Fig.~\ref{fig:generations}.
Data shown in Fig.~\ref{fig:acidification} a-b were obtained with a SevenCompact pH-meter (Mettler Toledo) and an MCR 301 rheometer (Anton Paar).
Permeability is measured using a protocol found in \cite{VanDijk1986} (see Materials and Supplementary Fig.~\ref{fig:permeability}).

\bibliographystyle{naturemag3}
\bibliography{Yaourt}

\paragraph*{Acknowledgements}
Confocal experiments were conducted at SFR BioSciences Gerland - Lyon Sud (US8 / UMS3444).
The authors thank José Bico, Arezeki Boudaoud, Cyprien Gay, L. Mahadevan and Olivier Pierre-Louis for theoretical insights at various stages of the research. 
Precious experimental ideas came from Thibaut Divoux, Catherine Barentin, Denis Bartolo, Elisabeth Bouchaud and Maxime Lefranc.
Alan Parker at Firmenich provided the casein and GDL; 
Special thanks to Madame Biot who made M.A. Biot work accessible through the \href{http://www.olemiss.edu/sciencenet/poronet/}{poronet} website. 
ML thanks the Region Rhône Alpes and the Programme d'Avenir Lyon - Saint Etienne (PALSE NoGELPo) for postoctoral grant. 
TG and MN acknowledge funding from ANR XXXX. 
ML and SM acknowledge funding from the European Research Council under the European Union's Seventh Framework Program (FP7/2007-2013) / ERC grant agreement No. 258803.


\paragraph*{Author Contributions}
ML and MN conducted preliminary experiments. ML, SM and TG designed the experiments and interpreted the results. ML and TG performed quantitative experiments. ML performed data analysis and modelling. ML, SM and TG wrote the manuscript.


\paragraph*{Author Information} 
The authors declare that they have no competing financial interests. 
Correspondence and requests for materials should be addressed to TG (\href{mailto:thomas.gibaud@ens-lyon.fr}{thomas.gibaud@ens-lyon.fr}).






\clearpage
\newpage
\setcounter{figure}{0}

\section*{Supplementary material}

\subsection*{Microstructure measurement}

\begin{figure}
	\includegraphics{nonmonotonic}%
	\caption{Microstructure evolution. (a) Details of confocal images in fully adhesive conditions (top row) and with no adhesion on top (bottom row). Scale bar is \SI{10}{\micro\metre}. Arrows indicates the pore size. (b) Corresponding Fourier spectra in second situation and same times. Lines are fits by Eq.~\ref{eq:fractalS}. (c-d) Evolution of susceptibility and fractal dimension in both situations (black line and orange dashed line respectively). Symbols indicates the times of the images in (a).}%
	\label{fig:nonmonotonic}%
\end{figure}

To follow the evolution of the microstructure of the gel, we acquired slices of $(\SI{2048}{px})^2$ with pixels of \SI{98}{\nano\metre} (Fig.~\ref{fig:nonmonotonic}a). After Hann windowing we performed Fourier transform and radially averaged the resulting spectrum to obtain $I(q)$ (see Fig.~\ref{fig:nonmonotonic}b) that was fitted by the following fractal form:
\begin{equation}
I(q) = \frac{2\chi\Gamma(d)}{\left(1+\left((d+1)\xi q\right)^2\right)^{d/2}}
\label{eq:fractalS}
\end{equation}
where the susceptibility $\chi$ is related to the fluorescence contrast between the network and the solvent, the cut off length $\xi$ corresponds to the largest pore size and $d$ is the fractal dimension. $\Gamma$ is the gamma function. 

Figure~\ref{fig:acidification}c of the main text and Supplementary Fig.~\ref{fig:nonmonotonic}c and d show the evolution of the fitting parameters in two different sets of boundary conditions: (i) complete adhesion of the gel on all walls, preventing syn\ae{}resis and swelling; (ii) complete adhesion except on the top wall, allowing syn\ae{}resis and swelling. The later situation is a proxy to the situation where we observe wrinkling, since it was necessary to keep the gel into the thin focal plane of the 63x objective lens of the confocal microscope. To do so, we coated only the microscope slide with acrylamide brush. The cover slip was cleaned but not coated, thus keeping strong adhesion on the side closer to the objective lens. Neither buckling nor wrinkling was detected in such configuration.

In situation (i), the gel keeps its topology throughout the gelation and over acidification. Left of Fig.~\ref{fig:nonmonotonic}a shows that the arms of the gel first shrink laterally while keeping their length, and then swell again. Syn\ae{}resis and swelling are thus local processes. By contrast in situation (ii), the gel as a whole compresses while forming. Since our focal plane is a fixed depth, we cannot follow the topology of the same subset of the network, however the pore size is clearly smaller than in situation (ii). The isoelectric gel, at minimum gel volume, is probably isotropic or only marginally anisotropic in $z$. This explains that subsequent swelling is not occurring only along $z$ but also in plane, causing buckling.

$\xi$ was also systematically measured in the same way at the end of each wrinkling experiment and reported in table~\ref{tab:data}. %Final fractal dimension was always close to 2 within error bars. Comparing $\chi$ across different setting of the microscope is not relevant.



\subsection*{Confocal images processing}

At 10x magnification and depending on cell thickness, we acquired stacks 20 to 50 images spaced by \SI{6.69}{\micro\metre} (Nyquist sampling). Each confocal slice was $(\SI{256}{px})^2$ with pixels of \SI{4.97}{\micro\metre}. At such low resolution details of the gel microstructure are undistinguishable as $\xi$ is smaller than the pixel size for all compositions and the gel appears as a continuous medium.

We first denoise each plane by a short-range Gaussian blur ($\sigma=\SI{2}{px}\ll\lambda$). Then, for each $(x,y)$, we focus on the z-dependent intensity profile $I(z)$. We define the position of the top surface of the gel $z_\text{top}$ as the point where $I(z)$ first crosses half of its maximum value over the given vertical line. Since this criterion is local, we do not need to correct for background intensity variation. In this way we obtain $z_\text{top}(x,y)$ with subpixel resolution ($\pm\SI{0.1}{px}=\SI{0.67}{\micro\metre}$), and $z_\text{bottom}(x,y)$ in a symmetrical way.

From these data, we obtain for each stack the volume and thickness of the gel phase, area of the midsurface and peak-to-peak amplitude. Velocity is obtained by time differentiation. Once the instability has saturated on the top and bottom walls of the cell, peaks in the histograms of $z_\text{top}$ and $z_\text{bottom}$ yield the measure of the positions of both walls, thus of the cell thickness $e$ and retrospectively of $H_1$ and $H_2$.



\subsection*{Mixed Darcy--Poiseuille model}

\subsubsection*{General framework for wrinkling}

The mechanical equilibrium of a plate of thickness $h$, Young modulus $E$ and Poisson ratio $\nu$ submitted to a compression load $\sigma_\parallel$ along $x$ and a transverse load $\sigma_\perp$ along $z$ writes
\begin{equation}
\frac{E h^3}{12(1-\nu^2)}\frac{\partial^4 w}{\partial x^4} + \sigma_\parallel h \frac{\partial^2 w}{\partial x^2} = \sigma_\perp,
\label{eq:mecheq}
\end{equation}
where $w(x,t)$ is the deflection of the plate along $z$~\cite{Biot1957}. Since we deal with slow rates of deformation, inertia is neglected.

We assume that the deflection can be decomposed in exponentially growing sinusoidal modes and we consider one of them $w(x,t) =  A_0 e^{t/\tau} \sin kx$ of characteristic time $\tau$ and wavevector $k$. Eq.~(\ref{eq:mecheq}) then reads
\begin{equation}
\left(\frac{1}{12} B h^3 k^4 - \sigma_\parallel h k^2\right)w = \sigma_\perp,
\label{eq:mecheqop}
\end{equation}
where $B=E/(1-\nu^2)$ is the modulus of the plate.

Following \cite{Biot1957}, we consider the case where the compression load $\sigma_\parallel$ is increased quasistatically. The observed mode is the one that arises at the lowest possible $\sigma_\parallel$, i.e. the mode minimising $\sigma_\parallel$.

\subsubsection*{Elastic film in an infinite elastic medium}

In this framework, we recall here the well known case of an elastic film lying over an infinite elastic medium. The deformation of the substrate of modulus $B_s$ gives the transverse load as $\sigma_\perp = -k B_s w$, and thus the compression load writes
\begin{equation}
\sigma_\parallel = B \frac{h^2}{12} k^2 + \frac{B_s}{hk}.
\label{eq:sigma0}
\end{equation}

Minimizing $\sigma_\parallel$ as a function of $k$ yields the well-known result $hk = \left(6B_s/B\right)^{1/3}$ \cite{Biot1957,Cerda2003} so that the dominant wavelength is 
\begin{equation}
\lambda = 2\pi h \left(\frac{B}{6B_s}\right)^{1/3}.
\label{eq:lambdaElEl}
\end{equation}

\subsubsection*{Elastic film on a viscous layer}
Let us now turn to the case of an elastic, impermeable film on a single viscous layer. As \cite{Huang2002} we neglect gravity, although it is difficult to do so in practice at a free interface. Assuming the layer thickness $H$ to be much smaller than $\lambda_d$, we can use the lubrication approximation. Further considering only small deflections, we may neglect the displacement of the plate along $x$. The flux along $x$ is thus of the Poiseuille form
\begin{equation}
Q = -\frac{H^3}{12\eta}\frac{\partial p}{\partial x}.
\label{eq:PoiseuilleFlux}
\end{equation}
where $p(x)$ is the pressure in the layer of viscosity $\eta$.

Mass conservation writes
\begin{equation}
\frac{\partial Q}{\partial x} + \frac{\partial w}{\partial t} = 0.
\label{eq:conservation}
\end{equation}
Keeping only linear order terms, one gets
\begin{equation}
\frac{12\eta}{\tau} \frac{w}{H^3k^2} = -\sigma_\perp.
\label{eq:PoiseuilleLoad}
\end{equation}

Combining Eqs.~(\ref{eq:mecheqop}) and (\ref{eq:PoiseuilleLoad}), the compression load writes
\begin{equation}
\sigma_\parallel = \frac{1}{12}B h^2 k^2 + 12\frac{\eta}{\tau}\frac{1}{H^3 h k^4}
\label{eq:sigma0P}
\end{equation}
and is minimized at $k_P$ such that
\begin{equation}
k_P^6 = 2\times 12^2 \frac{1}{h^3H^3\Upsilon},
\label{eq:kP}
\end{equation}
with $\Upsilon = B\tau/\eta$ the viscoelastic factor. The dominant wavelength for the Poiseuille model is thus
\begin{equation}
\lambda_P = \pi\sqrt{hH}\left(\frac{2}{9}\Upsilon\right)^{1/6}
\end{equation}

We note that the mode minimising $\sigma_\parallel$ is also maximising the growth rate
\begin{equation}
\frac{1}{\tau} = \frac{H^3k^2}{12^2\eta}\left(12\sigma_\parallel - Bh^3k^4\right).
\end{equation}

When the elastic film is sandwiched between two viscous layers of respective thickness $H_1$ and $H_2$, the linearity of Eq.~(\ref{eq:mecheq}) allows us to use an effective substrate of thickness 
\begin{equation}
\frac{1}{H^3} = \frac{1}{H_1^3}+\frac{1}{H_2^3}.
\end{equation}
This corresponds to the experimental case where the gel film is far from both of the cell walls. Since we have an almost buoyant gel and no more free surface the gravity can indeed be neglected.

\subsubsection*{Porous elastic film with no possible Poiseuille flow}
Another limiting case is that of a porous elastic film immersed in a viscous medium where longitudinal resistance to flow is infinite. It is the case in our experiments when the film is at contact with one of the walls. The only way to deform the film is to flow some liquid through the film of permeability $\alpha$. The transverse load derives from the Darcy law:
\begin{equation}
\sigma_\perp = p_1-p_2 = \frac{h\eta}{\alpha}\frac{\partial w}{\partial t}.
\label{eq:DarcyLoad}
\end{equation}

Combining Eqs.~(\ref{eq:mecheqop}) and (\ref{eq:DarcyLoad}), the compression load writes
\begin{equation}
\sigma_\parallel = \frac{1}{12}B h^2 k^2 + \frac{\eta}{\tau}\frac{1}{\alpha k^2}
\label{eq:sigma0D}
\end{equation}
and is minimized at $k_D$ such that
\begin{equation}
k_D^4 = \frac{12}{h^2\alpha\Upsilon}.
\label{eq:kD}
\end{equation}

The dominant wavelength for Darcy model is thus
\begin{equation}
\lambda_D = 2\pi h^{1/2}\alpha^{1/4}\left(\frac{\Upsilon}{12}\right)^{1/4}.
\end{equation}
Here also the dominant mode maximises the growth rate.


\subsubsection*{Porous elastic film between two viscous layers}
Finally the most realistic scenario mixes both Poiseuille and Darcy mechanisms, i.e. considers a porous elastic film between two viscous layers of arbitrary thickness. In this case, the mass conservation $Q_1 + Q_2 = 0$ over the whole height of the cell yields
\begin{equation}
H_1^3 p_1 + H_2^3 p_2 = 0.
\label{eq:pressures}
\end{equation}

The mass conservation in the lower viscous layer is the same as Eq.~(\ref{eq:conservation}) with an added leak $v$ due to the porosity
\begin{equation}
\frac{\partial Q_1}{\partial x} + \frac{\partial w}{\partial t} + v = 0.
\label{eq:conservationDarcy}
\end{equation}

$v$ can be expressed by the Darcy law
\begin{equation}
v = \frac{\alpha}{\eta} \frac{p_1-p_2}{h}.
\label{eq:Darcy}
\end{equation}

Combining Eqs.~(\ref{eq:pressures}, \ref{eq:conservationDarcy}, \ref{eq:Darcy}) we obtain the transverse load
\begin{equation}
\sigma_\perp = p_1-p_2 = - \frac{\eta}{\tau} \frac{w}{\frac{H^3 k^2}{12} + \frac{\alpha}{h}},
\label{eq:MixtLoad}
\end{equation}

Note that we recover Eq~(\ref{eq:PoiseuilleLoad}) when $\alpha \rightarrow 0$ and Eq.~(\ref{eq:DarcyLoad}) when $H \rightarrow 0$. We can recast this intuition in terms of the dimensionless number $H/H^*$ where $H^*$ is obtained by equating $k_P$ and $k_D$:
\begin{equation}
H^* = 2^{2/3} 3^{1/6} \alpha^{1/2} \Upsilon^{1/6}.
\end{equation}

Analytic minimization of $\sigma_\parallel$ knowing $\sigma_\perp$ is possible but cumbersome. Instead we rewrite Eq.~(\ref{eq:mecheqop}) in terms of previously calculated $\lambda_P$ and $\lambda_D$:
\begin{align}
\frac{12\sigma_\parallel}{k_D^2 h^2 B} &= \left(\frac{k}{k_D}\right)^2 + \frac{1}{2\frac{k_D^2}{k_P^6}k^4  + \left(\frac{k}{k_D}\right)^2}\\
&= \left(\frac{\lambda_D}{\lambda}\right)^2 + \frac{1}{2\frac{\lambda_P^6}{\lambda_D^2\lambda^4}  + \left(\frac{\lambda_D}{\lambda}\right)^2}
\end{align}
and we minimize this expression numerically.

\subsection*{Wavelength measurement}

\begin{figure}
	\includegraphics{generations}%
	\caption{Fourier spectra of binary altitudes of Main Fig.~\ref{fig:dynamics}a corresponding to primary, secondary and tertiary patterns. Inset: reconstructed binary altitudes at time \SI{2}{\hour}. Scale bars is \SI{1}{\milli\metre}.}%
	\label{fig:generations}%
\end{figure}

\begin{figure*}
	\includegraphics{patterns}
	\caption{Patterns corresponding to the samples of Table~\ref{tab:data} of the main text. (a-d) increasing cell thickness. (e-f) larger solid content. (g-i) increasing glycerol content. All pictures are stitching of fluorescent microscopy images except (c,e) which are details of reflected light macroscope images. Scale is common to all panels (scale bar \SI{1}{\milli\metre}). Length of arrows correspond to the measured primary wavelength.}
	\label{fig:patterns}
\end{figure*}

To measure wrinkle wavelength, we used large field pictures of the final state. These picture were obtained either by stitching fluorescent microscopy images obtained at 10x magnification (Fig.~\ref{fig:patterns} except c and e), or by transmitted or reflected light macroscope at magnifications ranging from 1x to 6x (Fig.~\ref{fig:patterns}c and e). The contrast in such pictures is a measure of the magnitude of the altitude gradient of the gel film. White lines in fluorescence and reflection (respectively black lines in transmission) denote a sharp transition between two extreme altitudes (on top wall or on bottom wall). It is thus possible to reconstruct a binary picture of the pattern (inset of Fig.~\ref{fig:generations}). Furthermore since the shape and size of a given generation of pattern do not change after formation, it is possible to revert the final picture to pure first generation.

Figure~\ref{fig:generations} shows Fourier spectra of binarised altitudes at different times. When only the first generation pattern is present a peak clearly indicates the dominant mode. When second generation has appeared, the dominant wavelength is halved and the peak much weaker. This allows us to use secondary pattern is our analysis by setting their wavelength at half of the primary wavelength. Further generations do not seem to appear simultaneously and the peak is weak and broad. This could be a signature of a transition to chaos of a non-linear oscillator~\cite{Brau2011}.

\subsection*{Characteristic time measurement}
\begin{figure}
	\includegraphics{growth}
	\caption{Growth of the instability. (a) Evolution of the gel velocity $v$ averaged over the whole surface. Inset: Evolution of the peak-to-peak amplitude $A(t)$ in semi logarithmic scale. Black line is the best exponential fit with $\tau=\SI{90}{\second}$. (b) Evolution of the bottom profile of the gel from onset of instability (yellow) to contact to both walls (brown). Note that the longitudinal size of the blister evolves in this regime. Thick black curve is the profile at much longer times. Note that the blister foot, highlighted by the orange vertical line, has not moved since contact. \seb{\it j'ai plusieurs petites questions par rapport a cette figure.}}
	\label{fig:growth}
\end{figure}

In most of our samples it was possible to measure directly the peak-to-peak amplitude $A(t)$ from our confocal measurements. However when the wavelength is too large compared to the field of view, this approach breaks down since we cannot observe both the highest and lowest points of the surface. Even when valid, such a local measurement is rather noisy and we prefer to focus on the measure of the vertical velocity $v$ of the gel that can be integrated over the whole field of view.

We note that when both measurements were available (as in Fig.~\ref{fig:growth}a), the characteristic time $\tau$ obtained from the fit of the exponential growth of $A(t)$ was about $4A_\text{max}/v_\text{max}$, where $A_\text{max} = e-h$ is the amplitude when the gel comes in contact with both walls and $v_\text{max}$ is the maximum velocity of the gel. We used this approximation of $\tau$ for all our samples. This is consistent with the fact that the blister size ($\propto\lambda$) does not evolve after contact (see Fig.~\ref{fig:growth}b), thus the kinetic wavelength is set when $A(t) = A_\text{max}$.

\subsection*{Permeability measurements}
\begin{figure*}
	\includegraphics{permeability}
	\caption{Permeability measurements. (a) Schematic representation of the experiment. (b-c) Evolution of the height of the interface in tube 1 relative to the final height in tube 2. Black line is the best exponential fit $Ae^{-t/\tau}$. (b) Gel is 4\%w casein, 4\%w GDL in water, $H=\SI{2.3}{\milli\metre}$ and $\tau=\SI{57}{\minute}$. (c) Same as (b) for a 50\%w glycerol--water mixture, $H=\SI{4}{\milli\metre}$ and $\tau=\SI{100}{\hour}$.}
	\label{fig:permeability}
\end{figure*}

To measure the permeability of our gels, we adapt a method from \cite{VanDijk1986}. First, the gel of interest is formed at the lower extremity of a thin glass tube. The tube is held vertical in a water saturated atmosphere to prevent evaporation until the pH has reached the isoelectric point ($\approx\SI{1}{\hour}$ for our most common composition). Then, we dip the tube in a bath of acetic acid/acetate buffer at pH=4. The buffer has the same glycerol content as the gel solvent, so that the viscosity is constant throughout the experiment. Alongside the first tube (see Fig.~\ref{fig:permeability} a), we dip a second identical tube with no gel in it. A webcam (Logitech Webcam Pro 9000) captures the rise of the liquid in each tube. Since the liquid height in tube 2 reaches its final value quickly, we measure directly $z_\infty-z(t)$.

Darcy law through the gel of permeability $\alpha$ and height $H$ reads 
\begin{equation}
\frac{dz}{dt} = \frac{\alpha}{\eta}\frac{\Delta P}{H}
\end{equation}
where $\Delta P=\rho g(z_\infty-z(t))$ is the hydrostatic pressure drop through the gel. We use tabulated values of water-glycerol mixtures for the density $\rho$ and viscosity $\eta$ of the buffer. We can thus write
\begin{equation}
\frac{d(z-z_\infty)}{dt} = -\frac{z-z_\infty}{\tau}\text{, with\,} \tau = \frac{\eta H}{\alpha \rho g}.
\end{equation}

By fitting $z(t)-z_\infty$ to an exponential (see Fig.~\ref{fig:permeability} b-c), we obtain $\tau$ and thus $\alpha$. Note that even with $H$ of the order of a few millimeters high glycerol contents yield $\tau\approx \SI{100}{\hour}$ due both to high viscosity and low permeability. Obtaining $z_\infty$ without tube 2 would be time prohibitive.

\end{document}