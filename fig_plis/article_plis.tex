\documentclass[twocolumn,superscriptaddress,showpacs,preprintnumbers,amsmath,amssymb,prl]{revtex4-1}
\usepackage[utf8]{inputenc}

\usepackage{graphicx}
\usepackage{xcolor}
\usepackage{hyperref}
\usepackage{siunitx}
%\usepackage{multirow}



\definecolor{Main}{rgb}{1, 0.57, 0}
\definecolor{Accent1}{rgb}{1,0.28,0}
\definecolor{Accent2}{rgb}{1,0.74,0}

\begin{document}
\title{Wrinkling of a confined porous layer}
\author{Mathieu Leocmach}
\affiliation{Universit\'e de Lyon, Laboratoire de Physique, \'Ecole Normale Sup\'erieure de Lyon, CNRS UMR 5672, 46 All\'ee d'Italie, 69364 Lyon cedex 07, France}
\author{Mathieu Nespoulous}
\affiliation{Aix-Marseille Université, CNRS, MADIREL UMR 7246, Marseille, France}
\author{Sébastien Manneville}
\affiliation{Universit\'e de Lyon, Laboratoire de Physique, \'Ecole Normale Sup\'erieure de Lyon, CNRS UMR 5672, 46 All\'ee d'Italie, 69364 Lyon cedex 07, France}
\author{Thomas Gibaud}
\email{thomas.gibaud@ens-lyon.fr}
\affiliation{Universit\'e de Lyon, Laboratoire de Physique, \'Ecole Normale Sup\'erieure de Lyon, CNRS UMR 5672, 46 All\'ee d'Italie, 69364 Lyon cedex 07, France}

\begin{abstract}

\end{abstract}

\maketitle

\section*{Methods summary}
To prevent adsorption of caseins to surfaces, microscope slides and cover-slips were cleaned and subsequently coated with a silane agent (3-(trimethoxysilyl) propyl methacrylate, Sigma Aldrich) to which a polyacrylamide brush was polymerized from methacrylate groups.
Sodium caseinate (Firmenich) was labelled with Dylight~550 NHS ESTER (Thermo Scientific) and then purified by centrifugation. Unlabelled sodium caseinate powder was dissolved in deionised water (or water+glycerol mixture). To induce gelation glucono-$\delta$-lactone (GDL) in powder (Firmenich) was added to the solution. \SI{45}{\micro\litre} of this solution was immediately mixed with \SI{5}{\micro\litre} of labelled caseinate solution and injected in the microscopy cell, immediately sealed using ultraviolet-cured glue (Norland Optical) and placed under the microscope before gelation could take place. 
3D data were collected on a Zeiss LSM510 confocal microscope, using \SI{532}{\nano\meter} laser excitation. We used either 10x (air) lens to observe wrinkling or 63x (oil) lens to observe the microstructure of the gel. Larger scope pictures were obtained either by stitching fluorescent microscope data or by transmitted/reflected light macroscope (Nikkon SMZ745T/Leica DMS1000).
Data shown in Fig.~\ref{fig:acidification} a-b were obtained with a SevenCompact pH-meter (Mettler Toledo) and an MCR 301 rheometer (Anton Paar).

\bibliographystyle{naturemag3}

\paragraph*{Acknowledgements}
Confocal experiments were conducted at SFR BioSciences Gerland - Lyon Sud (US8 / UMS3444).
The authors thank José Bico, Arezeki Boudaoud, Cyprien Gay and L. Mahadevan for theoretical insights at various sage of the research and Alan Parker at Firmenich for providing the casein and GDL. 
Precious experimental ideas came from Denis Bartolo, Elisabeth Bouchaud and Maxime Lefranc.
Catherine Barentin provided the (late) glass Couette cell of SM XXXX.
Special thanks to Madame Biot who made M.A. Biot work accessible through \href{http://www.olemiss.edu/sciencenet/poronet/}{poronet} website. 
ML thank the Region Rhône Alpes and the Programme d'avenir Lyon - Saint Etienne (PALSE NoGELPo) for postoctoral grant. 
TG and MN acknowledge funding from ANR XXXX. 
ML and SM acknowledge funding from the European Research Council under the European Union's Seventh Framework Program (FP7/2007-2013) / ERC grant agreement No. 258803.


\paragraph*{Author Contributions}
ML and MN conducted preliminary experiments. ML, SM and TG designed the experiments and interpreted the results. ML and TG performed quantitative experiments. ML performed data analysis and modelling. ML, SM and TG wrote the manuscript.


\paragraph*{Author Information} 
The authors declare that they have no competing financial interests. 
Correspondence and requests for materials should be addressed to TG (\href{mailto:thomas.gibaud@ens-lyon.fr}{thomas.gibaud@ens-lyon.fr}).



\begin{figure}
	\includegraphics{dynamics}
	\caption{Dynamics of pattern formation. Left: light transmission microscopy. The same area is highlighted in yellow on all panels to show that the initial pattern is not evolving. Right: 3D reconstruction from fluorescent confocal microscopy. Scale bars are \SI{1}{\milli\metre}.}
	\label{fig:dynamics}
\end{figure}

\begin{figure}
	\includegraphics{acidification}
	\caption{Acid-induced protein gels properties behave non-monotonously with pH. (a) pH decrease in 4\%w sodium caseinate solution acidified by 4\%w GDL. Horizontal line indicates isoelectric pH of casen (b) Evolution of storage modulus measured in a rheometer (full adhesion to rotor and stator). (c) Evolution of pore size measured by confocal microscopy in a half coated microscopy cell (no adhesion to cell ceiling, syn\ae{}resis and swelling allowed).}
	\label{fig:acidification}
\end{figure}

\begin{figure}
	\includegraphics{sideview}
	\caption{Modelisation  and measurements. (a) Schematic side view of the cell. The constant thickness gel sheet (orange) is surrounded by water. Yellow lines show the path of transmitted light through the gel phase. A region of interest is highlighted in gray. (b) Confocal XZ cuts showing syn\ae{}resis, swelling and (cascade) wrinkling. Scale bar is \SI{100}{\micro\metre} (real size ratio). Evolution of (c) volume of the gel phase relative to cell volume and (d) excess area measured by confocal microscopy.}
	\label{fig:sideview}
\end{figure}

\begin{figure*}
	\includegraphics{Darcy_vs_Poiseuille}
	\caption{Comparing model predictions with measured wavelengths. Dashed line is the best linear fit through the origin (prefactor is 0.81 in a, 0.65 in b), continuous line is the best affine fit ($\lambda_{xp}=0.56\lambda_{th}+\SI{0.28}{\milli\metre}$).}
	\label{fig:DarcyPoiseuille}
\end{figure*}

\clearpage
\newpage
\setcounter{figure}{0}

\section*{Supplemental material}

\begin{table*}
\begin{tabular}{SSSScSrrScSSSScSSSSSS}
%\toprule
\multicolumn{4}{c}{Preparation} & &  \multicolumn{4}{c}{Properties} & & \multicolumn{4}{c}{Measurements} & & \multicolumn{6}{c}{Results} \\ 
\cline{1-4} \cline{6-9} \cline{11-14} \cline{16-21}\\[-2ex]
{cas} & {GDL} & {gly} & {$e$} && {$\eta$} & \multicolumn{1}{c}{$G^\prime$} & \multicolumn{1}{c}{$\alpha$} & {$\xi$} && {$h$} & {$H_{min}$} & {$H_{max}$} & {$v$} && {$\Upsilon$} & {$H^*/H$} & {$\lambda_{xp}$} & {$\lambda_{D}$} & {$\lambda_{P}$} & {$\lambda_{D+P}$} \\ 
{\%w} & {\%w} & {\%w} & \si{\micro\metre} && \si{\milli\pascal\second} & \multicolumn{1}{c}{\si{\pascal}} & \si{\square\nano\metre} & \si{\micro\metre} && \si{\micro\metre} & \si{\micro\metre} & \si{\micro\metre} & \si{\micro\metre/\second} && {$10^8$} &  & \si{\milli\metre} & \si{\milli\metre} & \si{\milli\metre} & \si{\milli\metre} \\ 
\cline{1-4} \cline{6-9} \cline{11-14} \cline{16-21}\\[-2ex]
4 & 4 & 0 & 74.5 && 1.0 & 487 & 67000 & 4.9 && 38.3 & 6.3 & 29.3 & 0.11 && 1.62 & 0.97 & 0.89 & 1.20 & 0.88 & 1.22 \\ 
4 & 4 & 0 & 92.4 && 1.0 & 487 & 67000 & 4.9 && 27.8 & 13.5 & 50.8 & 0.20 && 1.57 & 0.45 & 0.85 & 1.01 & 1.10 & 1.20 \\ 
4 & 4 & 0 & 106.1 && 1.0 & 487 & 67000 & 4.9 && 32.7 & 30.7 & 42.9 & 0.14 && 2.54 & 0.24 & 1.52 & 1.24 & 1.85 & 1.90 \\ 
4 & 4 & 0 & 138.0 && 1.0 & 487 & 67000 & 4.9 && 48.2 & 32.0 & 63.9 & 0.11 && 4.05 & 0.23 & 2.46 & 1.69 & 2.56 & 2.62 \\ 
\cline{1-4} \cline{6-9} \cline{11-14} \cline{16-21}\\[-2ex]
8 & 8 & 0 & 85.2 && 1.0 & 1817 & 2150 & 3.1 && 49.8 & 10.0 & 26.8 & 0.10 && 6.50 & 0.14 & 1.00 & 0.82 & 1.60 & 1.61 \\ 
8 & 8 & 0 & 91.0 && 1.0 & 1817 & 2150 & 3.1 && 23.4 & 30.2 & 47.7 & 0.10 && 12.05 & 0.05 & 1.50 & 0.66 & 2.04 & 2.04 \\ 
\cline{1-4} \cline{6-9} \cline{11-14} \cline{16-21}\\[-2ex]
4 & 4 & 20 & 63.7 && 1.6 & 1142 & 24600 & 2.7 && 39.1 & 1.2 & 10.9 & 0.21 && 0.86 & 2.66 & 0.49 & 0.80 & 0.36 & 0.80 \\ 
4 & 4 & 35 & 82.3 && 2.7 & 1858 & 13000 & 1.8 && 58.3 & 0.9 & 12.6 & 0.02 && 10.98 & 4.03 & 0.61 & 1.27 & 0.56 & 1.27 \\ 
4 & 4 & 50 & 95.3 && 5.4 & 4660 & 5700 & 1.5 && 65.1 & 7.9 & 14.4 & 0.02 && 16.30 & 0.34 & 1.23 & 1.50 & 1.85 & 1.96 \\ 
\end{tabular}
\caption{Characteristics of the samples discussed in the main text}
\label{tab:data}
\end{table*}

\begin{figure}
	\includegraphics{mesh}
	\caption{Gel microstructure. (a) Detail of a confocal micrograph of a 4\%w casein, 4\%w GDL in water, when only the slide is coated (not the cover slip). Time corresponds to the minimum in mesh size. Scale bar is \SI{10}{\micro\metre}. (b) Corresponding radial averaged Fourier transform. Black line is the best fit by Eq.XXX with $\xi=\SI{4.9}{\micro\metre}$ and $d=2.4$. Inset shows the center of the non radial average spectrum (arbitrary logarithmic color scale). Scale bar is \SI{10}{\per\micro\metre}. (c-d) Idem in a 50\%w glycerol solvent, with $\xi=\SI{1.5}{\micro\metre}$ and $d=2.5$.}
	\label{fig:mesh}
\end{figure}

\subsection*{Permeability measurements}
\begin{figure*}
	\includegraphics{permeability}
	\caption{Permeability measurements. (a) Schematic representation of the experiment. (b-c) Evolution of the height of the interface in tube 1 relative to the final height in tube 2. Black line is the best exponential fit $Ae^{-t/\tau}$. (b) Gel is 4\%w casein, 4\%w GDL in water, $H=\SI{2.3}{\milli\metre}$ and $\tau=\SI{57}{\minute}$. (c) Idem in 50\%w glycerol, $H=\SI{4}{\milli\metre}$ and $\tau=\SI{100}{\hour}$.}
	\label{fig:permeability}
\end{figure*}

To measure the permeability of our gels, we adapt a method from \cite{VanDijk1986}. First, the gel of interest is formed at the lower extremity of a thin glass tube. The tube is held vertical in a water saturated atmosphere to prevent evaporation until the pH has reached the isoelectric point ($\approx\SI{1}{\hour}$ for our most common composition). Then, we dip the tube in a bath of acetic acid/acetate buffer at pH=4. The buffer has the same glycerol content as the gel solvent, so that the viscosity is constant throughout the experiment. Alongside the first tube (see Fig.~\ref{fig:permeability} a), we dip a second identical tube with no gel in it. A webcam (Logitech Webcam Pro 9000) records the rise of the liquid in each tube. Since the liquid height in tube 2 reaches its final value quickly, we measure directly $z_\infty-z(t)$.

Darcy law through the gel of permeability $\alpha$ and height $H$ writes
\begin{equation}
\frac{dz}{dt} = \frac{\alpha}{\eta}\frac{\Delta P}{H}
\end{equation}
where $\Delta P$ is the the hydrostatic pressure relative to $z_\infty$. We can thus write
\begin{equation}
\frac{d(z_\infty-z)}{dt} = -\frac{z_\infty-z}{\tau}\text{, with\,} \tau = \frac{\eta H}{\alpha \rho g}.
\end{equation}

By fitting $z_\infty-z(t)$ by an exponential (see Fig.~\ref{fig:permeability} b-c), we obtain $\tau$ and thus $\alpha$. Note that even with $H$ of the order of a few millimetres high glycerol contents yield $\tau\approx \SI{100}{\hour}$ due both to high viscosities and low permeability. Obtaining $z_\infty$ without tube 2 would be time prohibitive.

\end{document}