\documentclass[12pt,a4paper]{revtex4}
\usepackage[utf8]{inputenc}
\usepackage[english]{babel}
\usepackage[T1]{fontenc}
\usepackage{amsmath}
\usepackage{amsfonts}
\usepackage{amssymb}
\usepackage{graphicx}
\usepackage{kpfonts}

\usepackage{siunitx}
\usepackage{hyperref}


\begin{document}
\author{Mathieu Leocmach}
\title{Wrinkling à la Biot}
\maketitle

Mechanical equilibrium of a plate of thickness $h$, young modulus $E$ and Poisson ratio $\nu$ submitted to a compression load $\sigma_0$ along $x$ and a lateral load $q$ along $z$:
\begin{equation}
\frac{E h^3}{12(1-\nu^2)}\frac{\partial^4 w}{\partial x^4} + \sigma_0 h \frac{\partial^2 w}{\partial x^2} = q,
\end{equation}
where $w(x,t)$ is the deflection of the plate along $z$. It can be generalised as
\begin{equation}
B\frac{h^3}{12}\frac{\partial^4 w}{\partial x^4} + \sigma_0 h \frac{\partial^2 w}{\partial x^2} = -B_s \frac{\partial w}{\partial x},
\label{eq:mecheq}
\end{equation}
where $B=\frac{E}{1-\nu^2}$ and $B_s$ are the respective moduli (in \si{\pascal}) of the plate and the (effective) substrate. In general $B$ and $B_s$ are operators.

Let us suppose a sinusoidal deflection with a time dependent amplitude $w(x,t) = A(t) \sin kx$. Equation~\ref{eq:mecheq} becomes
\begin{equation}
B\frac{h^3}{12}k^4 - \sigma_0 h k^2 + B_s k =0,
\end{equation}
and the compression load can be expressed as
\begin{equation}
\sigma_0 = B \frac{h^2}{12} k^2 + \frac{B_s}{hk}.
\label{eq:sigma0}
\end{equation}

The dominant wavelength (wavevector) is the one minimizing the compression load.

\section{Elastic film in an infinite elastic medium}

$B$ and $B_s$ are constants, not $k$-dependant operators. We minimise $\sigma_0$:
\begin{equation}
0 = \frac{\partial \sigma_0}{\partial (kh)} = \frac{B}{6} hk -  \frac{B_s}{(hk)^2},
\end{equation}
which yields $hk = \left(\frac{6B_s}{B}\right)^{1/3}$ and thus the dominant wavelength is
\begin{equation}
\lambda_d = 2\pi h \left(\frac{B}{6B_s}\right)^{1/3}.
\label{eq:lambdaElEl}
\end{equation}

\section{Elastic film in an infinite viscous medium}

\citet{Biot1957} gives the modulus of the infinite viscous medium of viscosity $\eta$ as the operator
\begin{equation}
B_s = 4\eta\frac{\partial}{\partial t} = 4\frac{\eta}{\tau},
\end{equation}
if we look for an time dependent amplitude $A(t)= A_0 e^{t/\tau}$.

Still neither $B$ nor $B_s$ are $k$-dependant operators, so we can use directly Eq.~(\ref{eq:lambdaElEl})
\begin{equation}
\lambda_d = \pi h \left(\frac{B\tau}{3\eta}\right)^{1/3}.
\end{equation}

Since Eq.~(\ref{eq:mecheq}) is linear, the case of an elastic film on an semi-infinite viscous medium is dealt with by simply taking half of $B_s$.

\section{Elastic film on a viscous layer}
The layer thickness $H$ is supposed much smaller than $\lambda_d$, we are thus in the lubrication approximation.
\begin{equation}
\left \{
\begin{array}{c @{=} c}
	\frac{\partial p}{\partial z} & 0\\
	\frac{\partial^2 u}{\partial z^2} & \frac{1}{\eta}\frac{\partial p}{\partial x}
\end{array}
\right.,
\label{eq:lubrication}
\end{equation}
with $p$ the pressure in the fluid and $u$ the fluid velocity along $x$.

Since we work at small deflections, we can neglect the displacement of the plate along $x$ and set the boundary conditions to $u(z=0) = u(z=H) = 0$, and a double integration along $z$ of Eqs.~(\ref{eq:lubrication}) yields
\begin{equation}
u = \frac{1}{2\eta}\frac{\partial p}{\partial x}z(z-H).
\end{equation}

The flux along $x$ is thus of the Poiseuille form
\begin{equation}
Q = \int_0^H u dz = -\frac{H^3}{12\eta}\frac{\partial p}{\partial x}.
\label{eq:PoiseuilleFlux}
\end{equation}

Mass conservation writes
\begin{equation}
\frac{\partial Q}{\partial x} + \frac{\partial w}{\partial t} = 0.
\label{eq:conservation}
\end{equation}

Keeping only linear order terms
\begin{equation}
12\eta\frac{\partial w}{\partial t} = 3H^2\frac{\partial w}{\partial x}\frac{\partial p}{\partial x} + H^3\frac{\partial^2 p}{\partial x^2} \approx H^3\frac{\partial^2 p}{\partial x^2}.
\label{eq:conservationLin}
\end{equation}

In operator language
\begin{equation}
\frac{12\eta}{\tau} \frac{w}{H^3k^2} = -q
\end{equation}
which allows to identify $B_s = \frac{12\eta}{\tau H^3k^3}$. Since $B_s$ depends on $k$, we come back to Eq.~(\ref{eq:sigma0}) that we differentiate:
\begin{equation}
0 = \frac{\partial\sigma_0}{\partial k}
 = \frac{Bh^2 k}{6} - \frac{48}{hH^3}\frac{\eta}{\tau}\frac{1}{k^5},
\end{equation}
which yields 
\begin{equation}
k_P^6 = 2\times 12^2 \frac{1}{h^3H^3\Upsilon},
\end{equation}
with $\Upsilon = \frac{B\tau}{\eta}$ the viscoelastic factor. The dominant wavelength for Poiseuille model is thus
\begin{equation}
\lambda_P = \pi\sqrt{hH}\left(\frac{2}{9}\Upsilon\right)^{1/6}
\end{equation}

The scaling is then identical to the (much more involved) derivation by \citet{Huang2002}.

\section{Elastic film between two viscous layers}
We use once again the linearity of Eq.~(\ref{eq:mecheq}) to find the modulus of the effective substrate as 
\begin{equation}
B_s = \frac{12\eta}{\tau k^3} \left(\frac{1}{H_1^3}+\frac{1}{H_2^3}\right) = \frac{12\eta}{\tau H^3k^3}.
\end{equation}

We can use the results of the previous section using an effective $H$ so that
\begin{equation}
\frac{1}{H^3} = \frac{1}{H_1^3}+\frac{1}{H_2^3}.
\end{equation}

Note that $H$ goes to zero if any of the $H_1,H_2$ goes to zero.

\section{Porous elastic film with no possible flow}
The lateral load derives from the Darcy law through the film of permeability $\alpha$:
\begin{equation}
-q = p_2-p_1 = \frac{h\eta}{\alpha}\frac{\partial w}{\partial t} = \frac{\eta}{\tau}\frac{h}{\alpha k} \frac{\partial w}{\partial x},
\end{equation}
which allows to identify $B_s = \frac{\eta}{\tau}\frac{h}{\alpha k}$. Since $B_s$ depends on $k$, we come back to Eq.~(\ref{eq:sigma0}) that we differentiate:
\begin{equation}
0 = \frac{\partial\sigma_0}{\partial k}
 = \frac{Bh^2 k}{6} - \frac{2}{\alpha}\frac{\eta}{\tau}\frac{1}{k^3},
\end{equation}
which yields 
\begin{equation}
k_D^4 = \frac{12}{h^2\alpha\Upsilon}.
\end{equation}

The dominant wavelength for Darcy model is thus
\begin{equation}
\lambda_D = 2\pi h^{1/2}\alpha^{1/4}\left(\frac{\Upsilon}{12}\right)^{1/4}.
\end{equation}

\section{Porous elastic film between two viscous layers}
We write mass conservation over the whole height of the cell
\begin{equation}
\frac{\partial Q_1}{\partial x} + \frac{\partial Q_2}{\partial x} = 0.
\end{equation}

Since we only consider sinusoidal variations $Q_1 + Q_2 = 0$. Using the expression of the Poiseuille flux Eq.~(\ref{eq:PoiseuilleFlux}), we obtain
\begin{equation}
H_1^3 \frac{\partial p_1}{\partial x} + H_2^3 \frac{\partial p_2}{\partial x} = 0,
\end{equation}
and thus
\begin{equation}
H_1^3 p_1 + H_2^3 p_2 = 0.
\label{eq:pressures}
\end{equation}

We now write the mass conservation in the lower viscous layer
\begin{equation}
\frac{\partial Q_1}{\partial x} + \frac{\partial w}{\partial t} + v = 0,
\label{eq:conservationDarcy}
\end{equation}
which is the same as Eq.~(\ref{eq:conservation}) with an added leak $v$ due to the porosity. $v$ can be expressed by the Darcy law
\begin{equation}
v = \frac{\alpha}{\eta} \frac{p_1-p_2}{h}.
\label{eq:Darcy}
\end{equation}

Combining Eqs.~(\ref{eq:pressures}, \ref{eq:conservationDarcy}, \ref{eq:Darcy}) we obtain
\begin{align}
\frac{\partial w}{\partial t} &= \frac{H_1^3}{12\eta} \frac{\partial^2 p_1}{\partial x^2} - \frac{\alpha}{h\eta} \left(\frac{H_1}{H}\right)^3 p_1\\
\frac{\eta}{\tau}\left(\frac{H}{H_1}\right)^3 w &= -\left(\frac{H^3 k^2}{12} + \frac{\alpha}{h}\right)p_1.
\end{align}

The lateral load is
\begin{equation}
q = p_1-p_2 = \left(\frac{H_1}{H}\right)^3 p_1 = - \frac{\eta}{\tau} \frac{w}{\frac{H^3 k^2}{12} + \frac{\alpha}{h}},
\end{equation}
which allows to identify 
\begin{equation}
B_s = \frac{\eta}{\tau}\frac{1}{\frac{H^3 k^3}{12} + \frac{\alpha k}{h}}.
\end{equation}

Note that we recover a pure Poiseuille substrate when $\alpha \rightarrow 0$ and a pure Darcy substrate when $H \rightarrow 0$. Analytic minimisation of $\sigma_0$ knowing $B_s$ is possible but heavy. Instead we rewrite Eq.~(\ref{eq:sigma0}) in terms of previously calculated $\lambda_P$ and $\lambda_D$:
\begin{align}
\frac{12\sigma_0}{k_D^2 h^2 B} &= \left(\frac{k}{k_D}\right)^2 + \frac{1}{2\frac{k_D^2}{k_P^6}k^4  + \left(\frac{k}{k_D}\right)^2}\\
&= \left(\frac{\lambda_D}{\lambda}\right)^2 + \frac{1}{2\frac{\lambda_P^6}{\lambda_D^2\lambda^4}  + \left(\frac{\lambda_D}{\lambda}\right)^2}
\end{align}
and we minimise this expression numerically.



\bibliography{../../bib/Yaourt}

\end{document}